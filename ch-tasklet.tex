%\chapter{Overview of the {\XMP} 2.0 Model}
\chapter{DRAFT: Tasklet of upcoming {\XMP} 2.0}
\label{183724_5Nov18}

This chapter shows a draft of the specification of the tasklet
features from upcoming {\XMP} 2.0.
%
The models and any other rules of {\XMP} 2.0 follows those of {\XMP}
unless specified.

\section{{\XMP} Extended Execution Model}

When a node encounters a \verb|tasklets| construct at runtime, a set of
threads are created, an
implicit tasklet for {\it structured-block} is generated on the
node, and one of the threads begins execution of it.

When a thread encounters a \verb|tasklet| construct at runtime, a new
tasklet is generated on the node. Execution of generated
tasklets is assigned to one of the threads on the node, subject to the
thread's availability to execute work. Thus, execution of the new
tasklet could be immediate, or deferred until later according to the
tasklet scheduling constraint (Section \ref{115313_4Oct18}) and thread
availability.

At the end of the \verb|tasklets| construct, there is an implicit
\verb|taskletwait| construct to complete all of the tasklets generated
in the construct, after which the threads created at the beginning of the
construct are terminated, and the node resumes the execution.


\section{Glossary}\label{sec:tasklet-glossary}

\subsection{Node Terminology}

\begin{namelist}{predecessor taskletx}

\gitem{node}

  A logical entity managed by the XcalableMP runtime system, which has
  its own local memory and can communicate with each other, and on which
  one or more threads can execute inside the \verb|tasklets| region.

\end{namelist}

\subsection{Thread Terminology}

\begin{namelist}{predecessor taskletx}

\gitem{thread}

  An execution entity of tasklets, which execute on a node inside a
  \verb|tasklets| region.

\end{namelist}

\subsection{Tasklet Terminology}

\begin{namelist}{predecessor taskletx}

\gitem{tasklet}

  A specific instance of executable code and its data environment,
  generated when a thread encounters a \verb|tasklet| construct.

\gitem{tasklet dependence}

  An ordering relation between two sibling tasklets: the dependent
  tasklet and a previously generated predecessor tasklet. The tasklet
  dependence is fulfilled when the predecessor tasklet has completed.

\gitem{dependent tasklet}

  A tasklet that because of a tasklet dependence cannot be executed
  until its predecessor tasklets have completed.

\gitem{predecessor tasklet}

  A tasklet that must complete before its dependent tasklets can be
  executed.

\end{namelist}


\section{Directives}
\index{directive}

\subsection{Tasklet Constructs}

%%%%%%%%%%%%%%%%%%%%%%%%%%%%%%%%%%%%%%%%%%%%%
%
% tasklets
%

\subsubsection{{\tt tasklets} Construct}

\subsubsection*{Synopsis}

The {\tt \Directive{tasklets}} construct starts a region for tasklet
execution. See Section \ref{104936_4Oct18} for a general description of
the XMP's tasklet execution model.

\subsubsection*{Syntax}
\Syntax{tasklet}

\begin{tabular}{ll}
\verb![F]! & \verb|!$xmp| {\tt tasklets} {\openb}{\tt on} \{{\it nodes-ref} $\vert$ {\it template-ref}\}{\closeb} \\
& {\it structured-block} \\
& \verb|!$xmp| {\tt end tasklets} \\
& \\
\verb![C]! & \verb|#pragma xmp| {\tt tasklets} {\openb}{\tt on} \{{\it nodes-ref} $\vert$ {\it template-ref}\}{\closeb} \\
& {\it structured-block} \\
\end{tabular}

\subsubsection*{Description}

When a node encounters a \verb|tasklets| construct at runtime, an
implicit tasklet for {\it structured-block} is created and a thread on
the node begins execution of it. There is an implicit \verb|taskletwait|
construct at the end of the \verb|tasklets| construct.

In a \verb|tasklets| region, only directives listed below can be
specified.

\begin{itemize}
  \item \verb|tasklet|
  \item \verb|taskletyield|
  \item \verb|taskletwait|
  \item \verb|tasklet reflect|
  \item \verb|tasklet gmove|
  \item \verb|tasklet bcast|
  \item \verb|tasklet reduction|
  \item \verb|tasklet reduce_shadow|
\end{itemize}

Note that a \verb|tasklets| region may not contain another
\verb|tasklets| construct, that is, \verb|tasklets| cannot be nested.

In addition, no OpenMP directives can be specified in a \verb|tasklets|
region.

\subsubsection*{Restrictions}

\begin{itemize}
 \item The {\tt tasklets} construct is global, which means that it must
       be executed by all nodes in the current executing node set, and
       each local variable referenced in the construct must have the
       same value.
 \item The node set specified by the {\tt on} clause must be a subset of
       the executing node set.
\end{itemize}


%%%%%%%%%%%%%%%%%%%%%%%%%%%%%%%%%%%%%%%%%%%%%
%
% tasklet
%

\subsubsection{{\tt tasklet} Construct}

\subsubsection*{Synopsis}

The {\tt \Directive{tasklet}} construct defines a tasklet that will be
executed by a specified node set.

\subsubsection*{Syntax}
\Syntax{tasklet}

\begin{tabular}{ll}
\verb![F]! & \verb|!$xmp| {\tt tasklet} {\openb}{\tt on} \{{\it nodes-ref} $\vert$ {\it template-ref}\}{\closeb}
	  {\openb} {\it depend-clause} {\closeb}... \\
& {\it structured-block} \\
& \verb|!$xmp| {\tt end tasklet} \\
& \\
\verb![C]! & \verb|#pragma xmp| {\tt tasklet} {\openb}{\tt on} \{{\it nodes-ref} $\vert$ {\it template-ref}\}{\closeb}
	  {\openb} {\it depend-clause} {\closeb}... \\
& {\it structured-block} \\
\end{tabular}

\vspace{0.3cm}

where {\it depend-clause} is one of:

\vspace{0.3cm}

\begin{tabular}{ll}
 \hspace{0.5cm} & \verb|in (|    {\it variable} {\openb}, {\it variable}{\closeb}... \verb|)| \\
 \hspace{0.5cm} & \verb|out (|   {\it variable} {\openb}, {\it variable}{\closeb}... \verb|)| \\
 \hspace{0.5cm} & \verb|inout (| {\it variable} {\openb}, {\it variable}{\closeb}... \verb|)| \\
\end{tabular}

\vspace{0.3cm}

\begin{tabular}{ll}
 \hspace{0.5cm} & \verb|pro_post (| \{{\it nodes-ref} $\vert$ {\it template-ref}\} {\openb}, {\it tag}{\closeb} \verb|)|\\
 \hspace{0.5cm} & \verb|pro_wait| {\openb}{\tt (} \{{\it nodes-ref} $\vert$ {\it template-ref}\} {\openb},
	  {\it tag}{\closeb} {\tt )}{\closeb}\\
 \hspace{0.5cm} & \verb|epi_post (| \{{\it nodes-ref} $\vert$ {\it template-ref}\} {\openb}, {\it tag}{\closeb} \verb|)|\\
 \hspace{0.5cm} & \verb|epi_wait| {\openb}{\tt (} \{{\it nodes-ref} $\vert$ {\it template-ref}\} {\openb},
	  {\it tag}{\closeb} {\tt )}{\closeb}\\
\end{tabular}

\vspace{0.3cm}

\begin{tabular}{ll}
 \hspace{0.5cm} & \verb|remote_in| {\openb}\verb|(| {\it variable} {\openb}, \{{\it nodes-ref} $\vert$ {\it template-ref}\}{\closeb}
  {\openb}, {\it tag}{\closeb} \verb|)|{\closeb}\\
 \hspace{0.5cm} & \verb|remote_out| {\openb}\verb|(| {\it variable} {\openb}, \{{\it nodes-ref} $\vert$ {\it template-ref}\}{\closeb}
  {\openb}, {\it tag}{\closeb} \verb|)|{\closeb}\\
 \hspace{0.5cm} & \verb|accept_remote_in (|  {\it variable}, \{{\it nodes-ref} $\vert$ {\it template-ref}\} {\openb},
	  {\it tag}{\closeb} \verb|)|\\
 \hspace{0.5cm} & \verb|accept_remote_out (| {\it variable}, \{{\it nodes-ref} $\vert$ {\it template-ref}\} {\openb},
	  {\it tag}{\closeb} \verb|)|\\
\end{tabular}

\subsubsection*{Description}

When a thread encounters a {\tt tasklet} construct at runtime, it
generates a tasklet from the code for the associated block and put the
tasklet into the tasklet pool if it is included by the node set
specified by the {\tt on} clause; otherwise, it skips the block.

If the {\tt on} clause is omitted, it is assumed that the nodes that
owns a variable selected by the implementation from the {\it variables}
specified in the \verb|out| clause, if any, is specified in it;
otherwise it is assumed that the current executing node set is specified
in it.

The tasklet may have some scheduling constraints defined by {\it
depend-clause}. The specification of each {\it depend-clause} is
described in the following sections.

\subsubsection*{Restrictions}

\begin{itemize}
  \item The node set specified by {\it nodes-ref} or {\it template-ref}
		in the {\tt on} clause must be a subset of the parent node set.
\end{itemize}


%%%%%%%%%%%%%%%%%%%%%%%%%%%%%%%%%%%%%%%%%%%%%
%
% in/out/inout
%

\subsubsection{{\tt in}/{\tt out}/{\tt inout} Clauses}

\subsubsection*{Synopsis}

The \verb|in|, \verb|out|, and \verb|inout| clauses specify dependences
of tasklets within a node.

\subsubsection*{Syntax}
\Syntax{in}
\Syntax{out}
\Syntax{inout}

\begin{tabular}{ll}
 \hspace{0.5cm} & \verb|in (|    {\it variable} {\openb}, {\it variable}{\closeb}... \verb|)| \\
 \hspace{0.5cm} & \verb|out (|   {\it variable} {\openb}, {\it variable}{\closeb}... \verb|)| \\
 \hspace{0.5cm} & \verb|inout (| {\it variable} {\openb}, {\it variable}{\closeb}... \verb|)| \\
\end{tabular}

\subsubsection*{Description}

The \verb|in|, \verb|out|, and \verb|inout| clauses work in the same
way as the {\tt depend} clause of the {\tt task} directive in OpenMP 4.0
or later, with regard to dependences of tasklets within a node.

The \verb|in| clause. The generated tasklet will be a dependent tasklet
of all previously generated sibling tasklets that reference at least one
of the list items in an \verb|out| or \verb|inout| clause.

The \verb|out| and \verb|inout| clauses. The generated tasklet will be a
dependent tasklet of all previously generated sibling tasklets that
reference at least one of the list items in an \verb|in|, \verb|out|, or
\verb|inout| clause.

\subsubsection*{Restrictions}

\begin{itemize}
  \item {\it variables} used in these clauses of the same tasklet or sibling
		tasklets must indicate identical storage locations or disjoint
		storage locations.
  \item {\it variables} cannot be zero-length array sections.
  \item A variable that is part of another variable (such as an element
		of a structure) but is not an array element or an array section
		cannot appear in these clause.
  % \item For a vec element of sink dependence-type of the form xi + di or
  % 		xi 􀀀 di if the loop iteration variable xi has an integral or
  % 		pointer type, the expression xi + di or xi 􀀀 di for any value of
  % 		the loop iteration variable xi that can encounter the ordered
  % 		construct must be computable in the loop iteration variable’s
  % 		type without overflow.
  % \item For a vec element of sink dependence-type of the form xi + di or
  % 		xi 􀀀 di if the loop iteration variable xi is of a random access
  % 		iterator type other than pointer type, the expression ( xi - lb
  % 		i ) + di or ( xi - lbi ) 􀀀 di for any value of the loop
  % 		iteration variable xi that can encounter the ordered construct
  % 		must be computable in the type that would be used by
  % 		std::distance applied to variables of the type of xi without
  % 		overflow.
\end{itemize}


%%%%%%%%%%%%%%%%%%%%%%%%%%%%%%%%%%%%%%%%%%%%%
%
% pro_post
%

\subsubsection{{\tt pro\_post} Clause}

\subsubsection*{Synopsis}

The \verb|pro_post| clause, in combination with the \verb|pro_wait| or
\verb|epi_wait| construct, specifies a point-to-point synchronization
between tasklets.

\subsubsection*{Syntax}
\Syntax{pro\_post}

\begin{tabular}{ll}
\verb|pro_post (| \{{\it nodes-ref} $\vert$ {\it template-ref}\} {\openb}, {\it tag}{\closeb} \verb|)|\\
\end{tabular}

\subsubsection*{Description}

This clause ensures that the predecessor tasklets have completed on the
local node before tasklets that have a matching \verb|pro_wait| clause
are scheduled, or tasklets that have a matching \verb|epi_wait| clause
are completed on remote nodes.

A \verb|pro_post| clause having the arguments of {\it nodes-ref}/{\it
template-ref} and {\it tag}, of a tasklet on a node (called a {\it
posting node}) dynamically matches at most one \verb|pro_wait| or
\verb|epi_wait| clause having the arguments of the posting node (unless
omitted) and the same value as {\it tag} (unless omitted), of a tasklet
on the node specified by {\it nodes-ref}/{\it template-ref}.

\subsubsection*{Restrictions}

\begin{itemize}
  \item {\it tag} must be an expression of type default integer, in
   		{\XMPF}, or type {\tt int}, in {\XMPC}.
\end{itemize}


%%%%%%%%%%%%%%%%%%%%%%%%%%%%%%%%%%%%%%%%%%%%%
%
% pro_wait
%

\subsubsection{{\tt pro\_wait} Clause}

\subsubsection*{Synopsis}

The \verb|pro_wait| clause, in combination with the \verb|pro_post|
or \verb|epi_post| clause, specifies a point-to-point synchronization
between tasklets.

\subsubsection*{Syntax}
\Syntax{pro\_wait}

\begin{tabular}{ll}
\verb|pro_wait| {\openb}{\tt (} \{{\it nodes-ref} $\vert$ {\it template-ref}\} {\openb},
	  {\it tag}{\closeb} {\tt )}{\closeb}\\
%\verb|pro_wait| {\openb}{\tt (} {\it nodes-ref} {\openb}, {\it tag}{\closeb} {\tt )}{\closeb}\\
\end{tabular}

\subsubsection*{Description}

This clause prohibits the tasklet from being scheduled on the local node
until tasklets that have a matching \verb|pro_post|
clause are scheduled, or tasklets that have a matching \verb|epi_post|
clause are completed on remote nodes.

A \verb|pro_wait| clause having the arguments of {\it nodes-ref}/{\it
template-ref} and {\it tag}, of a tasklet on a node (called a {\it
waiting node}) dynamically matches a \verb|pro_post| clause having the 
arguments of the waiting node and the same value as {\it tag}, of a
tasklet on the node specified by {\it nodes-ref}/{\it template-ref}.

If {\it tag} is omitted, then the \verb|pro_wait| construct can
match a \verb|pro_post| or \verb|epi_post| clause having the
arguments of the waiting node and any tag, of tasklet on the node
specified by {\it nodes-ref}/{\it template-ref}.
%
If both {\it nodes-ref}/{\it template-ref} and {\it tag} are omitted, then the
\verb|pro_wait| clause can match a \verb|pro_post| or \verb|epi_post|
clause having the arguments of any node and any tag.

\subsubsection*{Restrictions}

\begin{itemize}
  \item {\it tag} must be an expression of type default integer, in
   		{\XMPF}, or type {\tt int}, in {\XMPC}.
\end{itemize}


%%%%%%%%%%%%%%%%%%%%%%%%%%%%%%%%%%%%%%%%%%%%%
%
% epi_post
%

\subsubsection{{\tt epi\_post} Clause}

\subsubsection*{Synopsis}

The \verb|epi_post| clause, in combination with the \verb|pro_wait| or
\verb|epi_wait| construct, specifies a point-to-point synchronization
between tasklets.

\subsubsection*{Syntax}
\Syntax{epi\_wait}

\begin{tabular}{ll}
\verb|epi_post (| \{{\it nodes-ref} $\vert$ {\it template-ref}\} {\openb}, {\it tag}{\closeb} \verb|)|\\
\end{tabular}

\subsubsection*{Description}

This clause ensures that the tasklet have completed on the
local node before tasklets that have a matching \verb|pro_wait| clause
are scheduled, or tasklets that have a matching \verb|epi_wait| clause
are completed on remote nodes.

A \verb|pro_post| clause having the arguments of {\it
nodes-ref}/{\it template-ref} and {\it tag}, of a tasklet on a node (called a {\it posting
node}) dynamically matches at most one \verb|pro_wait| or
\verb|epi_wait| clause having the arguments of the posting node (unless
omitted) and the same value as {\it tag} (unless omitted), of a tasklet
on the node specified by {\it nodes-ref}/{\it template-ref}.

\subsubsection*{Restrictions}

\begin{itemize}
  \item {\it tag} must be an expression of type default integer, in
   		{\XMPF}, or type {\tt int}, in {\XMPC}.
\end{itemize}


%%%%%%%%%%%%%%%%%%%%%%%%%%%%%%%%%%%%%%%%%%%%%
%
% epi_wait
%

\subsubsection{{\tt epi\_wait} Clause}

\subsubsection*{Synopsis}

The \verb|epi_wait| clause, in combination with the \verb|pro_post|
or \verb|epi_post| clause, specifies a point-to-point synchronization
between tasklets.

\subsubsection*{Syntax}
\Syntax{epi\_wait}

\begin{tabular}{ll}
\verb|epi_wait| {\openb}{\tt (} \{{\it nodes-ref} $\vert$ {\it template-ref}\} {\openb},
	  {\it tag}{\closeb} {\tt )}{\closeb}\\
\end{tabular}

\subsubsection*{Description}

This clause prohibits the tasklet from being completed on the local node 
until tasklets that have a matching \verb|pro_post|
clause are scheduled, or tasklets that have a matching \verb|epi_post|
clause are completed on remote nodes.

A \verb|pro_wait| clause having the arguments of {\it nodes-ref}/{\it template-ref} and
{\it tag}, of a tasklet on a node (called a {\it waiting node})
dynamically matches a \verb|pro_post| clause having the
arguments of the waiting node and the same value as {\it tag}, of a
tasklet on the node specified by {\it nodes-ref}/{\it template-ref}.

If {\it tag} is omitted, then the \verb|pro_wait| construct can
match a \verb|pro_post| or \verb|epi_post| clause having the
arguments of the waiting node and any tag, of tasklet on the node
specified by {\it nodes-ref}/{\it template-ref}.
%
If both {\it nodes-ref}/{\it template-ref} and {\it tag} are omitted, then the
\verb|pro_wait| clause can match a \verb|pro_post| or \verb|epi_post|
clause having the arguments of any node and any tag.

\subsubsection*{Restrictions}

\begin{itemize}
  \item {\it tag} must be an expression of type default integer, in
   		{\XMPF}, or type {\tt int}, in {\XMPC}.
\end{itemize}


%%%%%%%%%%%%%%%%%%%%%%%%%%%%%%%%%%%%%%%%%%%%%
%
% remote_in
%

\subsubsection{{\tt remote\_in} Clauses}

\subsubsection*{Synopsis}

The \verb|remote_in| clause specifies a predecessor tasklet with regard to a
data dependency derived from a remote-read (get) operation.

\subsubsection*{Syntax}
\Syntax{remote\_in}

\begin{tabular}{ll}
\verb|remote_in| {\openb}\verb|(| \{{\it variable} $\vert$ \verb|*|\}
  {\openb}, \{{\it nodes-ref} $\vert$ {\it template-ref}\}{\closeb}
  {\openb}, {\it tag}{\closeb} \verb|)|{\closeb}\\
\end{tabular}

\subsubsection*{Description}

A \verb|remote_in| clause can be regarded as a syntactic sugar for the
combination of {\tt in}, \verb|pro_wait|, and \verb|epi_post| clauses.

If \verb|*| is specified as the first argument or all of the arguments
are omitted, no {\tt in} clause is implied.

If the second argument is omitted, the implied \verb|pro_wait| clause
will match the first incoming \verb|pro_post| or \verb|epi_post| clause
from a node and the implied \verb|epi_post| will work as if the node
was specified in it.

\subsubsection*{Restrictions}

\begin{itemize}
  \item {\it variables} used in these clauses of the same tasklet or sibling
		tasklets must indicate identical storage locations or disjoint
		storage locations.
  \item {\it variables} cannot be zero-length array sections.
  \item A variable that is part of another variable (such as an element
		of a structure) but is not an array element or an array section
		cannot appear in these clause.
  \item {\it tag} must be an expression of type default integer, in
   		{\XMPF}, or type {\tt int}, in {\XMPC}.
\end{itemize}


%%%%%%%%%%%%%%%%%%%%%%%%%%%%%%%%%%%%%%%%%%%%%
%
% accept_remote_in
%

\subsubsection{{\tt accept\_remote\_in} Clause}

\subsubsection*{Synopsis}

The \verb|accept_remote_in| clause specifies a dependent tasklet with regard
to a data dependency derived from a remote-read (get) operation.

\subsubsection*{Syntax}
\Syntax{accept\_remote\_in}

\begin{tabular}{ll}
\verb|accept_remote_in (| {\it variable}, \{{\it nodes-ref} $\vert$ {\it template-ref}\} {\openb}, {\it tag}{\closeb} \verb|)|\\
\end{tabular}

\subsubsection*{Description}

A \verb|accept_remote_in| clause can be regarded as a syntactic sugar for the
combination of {\tt in}, \verb|pro_post|, and \verb|epi_wait| clauses.

\begin{quotation}
  (Advice to implementers) The action for the implied \verb|epi_wait|
  clause may be deferred until a dependent tasklet of the tasklet with
  respect to the the implied \verb|in| clause is scheduled.
\end{quotation}

\subsubsection*{Restrictions}

\begin{itemize}
  \item {\it variables} used in these clauses of the same tasklet or sibling
		tasklets must indicate identical storage locations or disjoint
		storage locations.
  \item {\it variables} cannot be zero-length array sections.
  \item A variable that is part of another variable (such as an element
		of a structure) but is not an array element or an array section
		cannot appear in these clause.
  \item {\it tag} must be an expression of type default integer, in
   		{\XMPF}, or type {\tt int}, in {\XMPC}.
\end{itemize}


%%%%%%%%%%%%%%%%%%%%%%%%%%%%%%%%%%%%%%%%%%%%%
%
% remote_out
%

\subsubsection{{\tt remote\_out} Clause}

\subsubsection*{Synopsis}

The \verb|remote_out| clause specifies a predecessor tasklet with regard to a
data dependency derived from a remote-write (put) operation.

\subsubsection*{Syntax}
\Syntax{remote\_out}

\begin{tabular}{ll}
\verb|remote_out| {\openb}\verb|(| \{{\it variable} $\vert$ \verb|*|\}
  {\openb}, \{{\it nodes-ref} $\vert$ {\it template-ref}\}{\closeb}
  {\openb}, {\it tag}{\closeb} \verb|)|{\closeb}\\
\end{tabular}

\subsubsection*{Description}

A \verb|remote_out| clause can be regarded as a syntactic sugar for the
combination of {\tt out}, {\tt pro\_wait}, and {\tt epi\_post} clauses.

If \verb|*| is specified as the first argument or all of the arguments
are omitted, no {\tt in} clause is implied.

If the second argument is omitted, the implied \verb|pro_wait| clause
will match the first incoming \verb|pro_post| or \verb|epi_post| clause
from a node and the implied \verb|epi_post| will work as if the node
was specified in it.

\subsubsection*{Restrictions}

\begin{itemize}
  \item {\it variables} used in these clauses of the same tasklet or sibling
		tasklets must indicate identical storage locations or disjoint
		storage locations.
  \item {\it variables} cannot be zero-length array sections.
  \item A variable that is part of another variable (such as an element
		of a structure) but is not an array element or an array section
		cannot appear in these clause.
  \item {\it tag} must be an expression of type default integer, in
   		{\XMPF}, or type {\tt int}, in {\XMPC}.
\end{itemize}


%%%%%%%%%%%%%%%%%%%%%%%%%%%%%%%%%%%%%%%%%%%%%
%
% accept_remote_out
%

\subsubsection{{\tt accept\_remote\_out} Clause}

\subsubsection*{Synopsis}

The \verb|accept_remote_out| clause specifies a dependent tasklet with regard
to a data dependency derived from a remote-write (put) operation.

\subsubsection*{Syntax}
\Syntax{accept\_remote\_out}

\begin{tabular}{ll}
\verb|accept_remote_out (| {\it variable}, \{{\it nodes-ref} $\vert$ {\it template-ref}\} {\openb},
	  {\it tag}{\closeb} \verb|)|\\
\end{tabular}

\subsubsection*{Description}

A \verb|accept_remote_out| clause can be regarded as a syntactic sugar for the
combination of {\tt out}, \verb|pro_post|, and \verb|epi_wait| clauses.

\begin{quotation}
  (Advice to implementers) The action for the implied \verb|pro_post|
  clause may be advanced after the scheduling constraint enforced by the
  implied \verb|out| clause is met.
\end{quotation}

\subsubsection*{Restrictions}

\begin{itemize}
  \item {\it variables} used in these clauses of the same tasklet or sibling
		tasklets must indicate identical storage locations or disjoint
		storage locations.
  \item {\it variables} cannot be zero-length array sections.
  \item A variable that is part of another variable (such as an element
		of a structure) but is not an array element or an array section
		cannot appear in these clause.
  \item {\it tag} must be an expression of type default integer, in
   		{\XMPF}, or type {\tt int}, in {\XMPC}.
\end{itemize}


%%%%%%%%%%%%%%%%%%%%%%%%%%%%%%%%%%%%%%%%%%%%%
%
% Examples
%

%\subsubsection{Examples of {\tt tasklet} Constructs}

% \Example{task}
% \Example{end task}

% \begin{description}

% \item[Example 1]

% In XcalableMP Fortran, copies of variables {\tt a} and {\tt b} are replicated on
% nodes {\tt nd(1)} through {\tt nd(8)}. 
% A task defined by the {\tt task} construct is executed only on {\tt nd(1)} and
% defines the copies of {\tt a} and {\tt b} on a node {\tt nd(1)}. 
% The copies on nodes {\tt nd(2)} through {\tt nd(8)} are not defined.

% In XcalableMP C, copies of variables {\tt a} and {\tt b} are replicated on
% nodes {\tt nd[0]} through {\tt nd[7]}. 
% A task defined by the {\tt task} construct is executed only on {\tt nd[0]} and
% defines the copies of {\tt a} and {\tt b} on a node {\tt nd[0]}.
% The copies on nodes {\tt nd[1]} through {\tt nd[7]} are not defined.

% \hspace{\hsize}

% \begin{minipage}{0.44\hsize}
% \begin{center}
% \begin{XFexample}
% !$xmp nodes nd(8)
% !$xmp template t(100)
% !$xmp distribute t(block) onto nd

%       real a, b;

% !$xmp task on nd(1)
%       read(*,*) a
%       b = a*1.e-6
% !$xmp end task
% \end{XFexample}
% \end{center}
% \end{minipage}
% %
% \begin{minipage}{0.51\hsize}
% \begin{center}
% \begin{XCexampleR}
% #pragma xmp nodes nd[8]
% #pragma xmp template t[100]
% #pragma xmp distribute t[block] onto nd

%     float a, b;

% #pragma xmp task on nd[0]
%     {
%         scanf ("%f", &a);
%         b = a*1.e-6;
%     }
% \end{XCexampleR}
% \end{center}
% \end{minipage}

% \vspace{1cm}

% \item[Example 2]

% According to the {\tt on} clause with a template reference,
% an assignment statement in the {\tt task} construct is
% executed by the owner of the array element {\tt a(:,j)} or {\tt a[j][:]}.

% \hspace{\hsize}

% \begin{minipage}{0.44\hsize}
% \begin{center}
% \begin{XFexample}
% !$xmp nodes nd(8)
% !$xmp template t(100)
% !$xmp distribute t(block) onto nd

%       integer i,j
%       real a(200,100)
% !$xmp align a(*,j) with t(j)

%       i = ...
%       j = ...

% !$xmp task on t(j)
%       a(i,j) = 1.0
% !$xmp end task
% \end{XFexample}
% \end{center}
% \end{minipage}
% %
% \begin{minipage}{0.51\hsize}
% \begin{center}
% \begin{XCexampleR}
% #pragma xmp nodes nd[8]
% #pragma xmp template t[100]
% #pragma xmp distribute t(block) onto nd

%     int i,j;
%     float a[100][200];
% #pragma align a[j][*] with t[j]

%     i = ...;
%     j = ...;

% #pragma xmp task on t[j]
%     a[j][i] = 1.0;
% }
% \end{XCexampleR}
% \end{center}
% \end{minipage}

% \end{description}


%%%%%%%%%%%%%%%%%%%%%%%%%%%%%%%%%%%%%%%%%%%%%
%
% taskletloop
%

% \subsection{{\tt taskletloop} Construct}

% \subsubsection*{Synopsis}

% The \verb|taskletloop| construct specifies that the iterations of one or
% more associated loops will be executed in parallel using tasklets. The
% iterations are distributed across tasklets created by the construct and
% scheduled to be executed.

% \subsubsection*{Syntax}
% \Syntax{taskletloop}

% \begin{tabular}{ll}
% \verb![F]! & \verb|!$xmp| {\tt taskletloop} {\openb} \verb|(| {\it loop-index}
%  {\openb}, {\it loop-index}{\closeb}... \verb|)| {\closeb} 
% 	  {\tt on} \{{\it nodes-ref} $\vert$ {\it template-ref}\} {\bsquare} \\
%  & \hspace{5cm}{\bsquare} 
% 	  {\openb} \verb|expand(| {\it expand-width} {\openb}, {\it
% 	  expand-width}{\closeb}... \verb|)| {\closeb} {\bsquare} \\
%  & \hspace{5cm}{\bsquare} 
% 	  {\openb} \verb|margin(| {\it margin-width} {\openb}, {\it
% 	  margin-width}{\closeb}... \verb|)| {\closeb} {\bsquare} \\
%  & \hspace{5cm}{\bsquare} 
% 	  {\openb} {\it reduction-clause} {\closeb}... \\
%  & {\it do-loops} \\
%  & \\
% \verb![C]! & \verb|#pragma xmp| {\tt taskletloop} {\openb} \verb|(| {\it
%      loop-index} {\openb}, {\it loop-index}{\closeb}... \verb|)|
%      {\closeb} {\tt on} \{{\it nodes-ref} $\vert$ {\it template-ref}\} {\bsquare} \\
%  & \hspace{5cm}{\bsquare} 
% 	  {\openb} \verb|expand(| {\it expand-width} {\openb}, {\it
% 	  expand-width}{\closeb}... \verb|)| {\closeb} {\bsquare} \\
%  & \hspace{5cm}{\bsquare} 
% 	  {\openb} \verb|margin(| {\it margin-width} {\openb}, {\it
% 	  margin-width}{\closeb}... \verb|)| {\closeb} {\bsquare} \\
%  & \hspace{5cm}{\bsquare}
% 	  {\openb} {\it reduction-clause} {\closeb}... \\
%  & {\it for-loops} \\
% \end{tabular}

% \subsubsection*{Description}


%%%%%%%%%%%%%%%%%%%%%%%%%%%%%%%%%%%%%%%%%%%%%
%
% taskletyield
%

\subsubsection{{\tt taskletyield} Construct}

\subsubsection*{Synopsis}

The \verb|taskletyield| construct specifies that the current tasklet can
be suspended in favor of execution of a different tasklet.
%The taskyield construct is a stand-alone directive.

\subsubsection*{Syntax}
\Syntax{taskletyield}

\begin{tabular}{ll}
\verb![F]! & \verb|!$xmp| {\tt taskletyield} \\
& \\
\verb![C]! & \verb|#pragma xmp| {\tt taskletyield} \\
\end{tabular}

\subsubsection*{Description}

The \verb|taskletyield| construct includes an explicit task scheduling
point in the current tasklet.

%%%%%%%%%%%%%%%%%%%%%%%%%%%%%%%%%%%%%%%%%%%%%
%
% taskletwait
%

\subsubsection{{\tt taskletwait} Construct}

\subsubsection*{Synopsis}

The \verb|taskletwait| construct specifies a wait on the completion of
child tasklets of the current tasklet.

\subsubsection*{Syntax}
\Syntax{taskletwait}

\begin{tabular}{ll}
\verb![F]! & \verb|!$xmp| {\tt taskletwait} \\
& \\
\verb![C]! & \verb|#pragma xmp| {\tt taskletwait} \\
\end{tabular}

\subsubsection*{Description}

The \verb|taskletwait| construct includes an implicit task scheduling
point in the current tasklet. The current tasklet is suspended at the
tasklet scheduling point until all child tasks that it generated before
the taskletwait construct complete execution.


% %%%%%%%%%%%%%%%%%%%%%%%%%%%%%%%%%%%%%%%%%%%%%
% %
% % taskletbarrier
% %

% \subsection{{\tt taskletbarrier} Construct}

% \subsubsection*{Synopsis}

% The \verb|taskletbarrier| construct specifies a wait on the completion
% of descendant tasklets of the current tasklet.

% \subsubsection*{Syntax}
% \Syntax{taskletwait}

% \begin{tabular}{ll}
% \verb![F]! & \verb|!$xmp| {\tt taskletbarrier} \\
% & \\
% \verb![C]! & \verb|#pragma xmp| {\tt taskletbarrier} \\
% \end{tabular}

% \subsubsection*{Description}


%%%%%%%%%%%%%%%%%%%%%%%%%%%%%%%%%%%%%%%%%%%%%
%
% Tasklet Scheduling
%

\subsection{Tasklet Scheduling}
\label{115313_4Oct18}

% tasklet scheduling points

Whenever a thread reaches a tasklet scheduling point, the implementation
may cause it to perform a tasklet switch, beginning or resuming
execution of a different tasklet. Tasklet scheduling points are implied
at the following locations:

\begin{itemize}
  \item the point immediately following the generation of a tasklet
  \item the point of completion of a tasklet that has one or more
		implicit or explicit \verb|epi_wait| clauses
  \item after the point of completion of a tasklet
  \item at a \verb|taskletyield| construct
  \item at an implicit and explicit \verb|taskletwait| construct
\end{itemize}

% actions at a tasklet scheduling point

When a thread encounters a tasklet scheduling point, it may do either of
the following, subject to the Tasklet Scheduling Constraints (below):

\begin{itemize}
  \item begin execution of a tasklet
  \item resume any suspended tasklet
\end{itemize}

If more than one of the above choices is available, it is unspecified as
to which will be chosen.

% task scheduling constraints

Tasklet Scheduling Constraints are as follows:

\begin{enumerate}
  % \item Scheduling of new tasklets is constrained by the set of tasklets
  % 		that are not suspended at a \verb|barrier| construct.
  % 		%
  % 		If this set is empty, any new tasklet may be scheduled.
  % 		%
  % 		Otherwise, a new tasklet may be scheduled only if it is a
  % 		descendent tasklet of every tasklet in the set.
  \item A dependent tasklet shall not be scheduled until its tasklet
		dependences are fulfilled.
  \item A tasklet that has one or more implicit or explicit
		\verb|pro_wait| clauses shall not be scheduled until all of the
		\verb|pro_wait| clauses are matched with \verb|pro_post| or
		\verb|epi_post| clauses.
\end{enumerate}

% A program relying on any other assumption about task scheduling is
% non-conforming.


%%%%%%%%%%%%%%%%%%%%%%%%%%%%%%%%%%%%%%%%%%%%%
%
% Global Communication Tasklet Constructs
%
%%%%%%%%%%%%%%%%%%%%%%%%%%%%%%%%%%%%%%%%%%%%%

\subsection{Communication Tasklet Constructs}

\subsubsection{Overview}

The communication tasklet constructs define inter-node interactions
between tasklets on the basis of XMP's global-view communication
constructs.

They are executed as a set of tasklets, each of which produces the same
effect on a node as the corresponding global-view communication
construct does.
% Each of the communication tasklet constructs works as a
% local construct, not global.

\begin{quotation}
  % (Advice to implementers) To implement the above feature of the
  % communication tasklet constructs, they should be based only on the 
  % remote read (get) operation. 
  (Advice to implementers) To implement the above feature of the
  communication tasklet constructs, they should be based on
  remote read (get) operations and point-to-point synchronizations.
\end{quotation}


% Each the construct can be ragarded as a syntactic sugar for a {\tt
% tasklet} construct that has a global-view communication construct in its
% associated block and appropriate {\it depend-clause}'s.

%%%%%%%%%%%%%%%%%%%%%%%%%%%%%%%%%%%%%%%%%%%%%
%
% tasklet reflect
%

\subsubsection{{\tt tasklet reflect} Construct}

\subsubsection*{Synopsis}

The \verb|tasklet reflect| construct defines tasklets that update the
shadow area of a global array like the \verb|reflect| directive.

\subsubsection*{Syntax}
\Syntax{tasklet reflect}

\begin{tabular}{ll}
 \verb![F]! & \verb|!$xmp tasklet| {\tt reflect} \verb|(| {\it array-name}
 {\openb}, {\it array-name}{\closeb}... \verb|)| {\bsquare} \\
 &\hspace{0.3cm} {\bsquare} {\openb}{\tt width (} {\it reflect-width}
     {\openb}, {\it reflect-width}{\closeb}... {\tt )}{\closeb}
     {\openb}{\tt orthogonal}{\closeb} {\bsquare} \\
 &\hspace{0.3cm} {\bsquare} {\openb}{\tt on} \{{\it nodes-ref} $\vert$
	  {\it template-ref}\}{\closeb} \\
\verb![C]! & \verb|#pragma xmp tasklet| {\tt reflect} \verb|(| {\it array-name}
     {\openb}, {\it array-name}{\closeb}... \verb|)| {\bsquare} \\
 &\hspace{0.3cm} {\bsquare} {\openb}{\tt width (} {\it reflect-width}
     {\openb}, {\it reflect-width}{\closeb}... {\tt )}{\closeb}
     {\openb}{\tt orthogonal}{\closeb} {\bsquare} \\
 &\hspace{0.3cm} {\bsquare} {\openb}{\tt on} \{{\it nodes-ref} $\vert$
	  {\it template-ref}\}{\closeb} \\
\end{tabular}

% #pragma xmp tasklet reflect (array-name[, array-name] ... )
% [chunksize (reflect-chunksize[, reflect-chunksize] ... ) ]

\subsubsection*{Description}

The \verb|tasklet reflect| construct generates a tasklet on each of the
nodes specified by the \verb|on| clause, which will produce the same
effect as the \verb|reflect| construct having the same clauses does. In
addition, for each of the arrays specified by the sequence of {\it
array-names}, an \verb|inout| clause for it implicitly added to the
generated tasklets.

Note that \verb|tasklet reflect| is a local construct, unlike
\verb|reflect|, and therefore the node set specified by the \verb|on|
clause need not include all of the nodes onto which the target arrays
are mapped.

% The chunksize clause can be matched the task dependency
% descriptions of users using the dependency generated by the
% \verb|tasklet reflect| construct.
% %
% When users calculate an array in block units, such as
% in the cache blocking technique for a node with data dependency, the
% user-specified task dependency and generated data dependency for halo
% exchange may not identically match.
% %
% By specifying the chunksize clause,
% the halo region is distributed logically to equal-sized contiguous chunks,
% and data dependencies for the halo exchange are generated
% automatically by the XMP runtime system based on the specified chunk
% size.

\subsubsection*{Restrictions}

\begin{itemize}
 % \item The arrays specified by the sequence of {\it array-names} must
 %       be mapped onto the executing node set.
 \item The reflect width of each dimension specified by the {\it
       reflect-width} must not exceed the shadow width of the arrays.
 \item The node set specified by the {\tt on} clause must be a subset of the
       executing node set.
\end{itemize}

\subsubsection*{Example}

% For example, when updating one side of a halo region for a one-dimensional
% global array on two nodes, these communications are separated into
% four tasks: the sender of the upper element on node 1, the receiver of
% the upper halo region on node 1, the sender of the lower element on node
% 2, and the receiver of the lower halo region on node 2.


%%%%%%%%%%%%%%%%%%%%%%%%%%%%%%%%%%%%%%%%%%%%%
%
% tasklet gmove
%

\subsubsection{{\tt tasklet gmove} Construct}

\subsubsection*{Synopsis}

The \verb|tasklet gmove| construct defines tasklets that copy the
variable from the right-hand side (rhs) into the left-hand side (lhs) of
the associated assignment statement like the \verb|gmove| construct.

\subsubsection*{Syntax}
\Syntax{tasklet gmove}

\begin{tabular}{ll}
\verb![F]! & \verb|!$xmp tasklet| {\tt gmove} {\openb}{\tt on} \{{\it
	  nodes-ref} $\vert$ {\it template-ref}\}{\closeb} \\
% {\it dest} {\tt =} {\it source} \\
\verb![C]! & \verb|#pragma xmp tasklet| {\tt gmove} {\openb}{\tt on} \{{\it
	  nodes-ref} $\vert$ {\it template-ref}\}{\closeb} \\
% {\it dest} {\tt =} {\it source} \\
\end{tabular}

\subsubsection*{Description}

The \verb|tasklet gmove| construct generates a tasklet on each of the
nodes specified by the \verb|on| clause, which will produce the same
effect as the \verb|gmove| construct having the same clauses does. In
addition, for the variables on the right-hand and left-hand side of the
associated assignment statement, \verb|in| and \verb|out| clauses are
implicitly added to the generated tasklets, respectively,

Note that \verb|tasklet gmove| is a local construct, unlike
\verb|gmove|, and therefore the node set specified by the \verb|on|
clause need not include all of the nodes onto which global arrays
appearing in the associated statement, if any, are mapped.

% The copy operation is basically performed on all
% execution nodes. However, if the distributed array is specified at the
% associated assignment statement, only nodes with the distributed array
% execute the operation in the task. The execution nodes can also be
% determined by the on clause. When the in, out, or inout clause is
% present on the tasklet gmove directive, the generated task has the
% corresponding data dependency in a node, similar to the tasklet
% directive. While the gmove directive also supports one-sided
% communication by the in and out clauses, which differ from those used
% for data dependencies, the tasklet gmove directive does not support
% these in this work.

\subsubsection*{Restrictions}

\begin{itemize}
 \item The \verb|tasklet gmove| construct must be followed by (i.e.,
       associated with) a simple assignment statement that contains
       neither arithmetic operations nor function calls.
 \item The node set specified by the {\tt on} clause must be a subset of the
       executing node set.
\end{itemize}

% %%%%%%%%%%%%%%%%%%%%%%%%%%%%%%%%%%%%%%%%%%%%%
% %
% % tasklet barrier
% %

% \subsection{{\tt tasklet barrier} Construct}

% The \verb|tasklet barrier| construct defines a tasklet that ???
% like the \verb|barrier| construct.

% \subsubsection*{Synopsis}


% \subsubsection*{Syntax}
% \Syntax{tasklet_barrier}

% \begin{tabular}{ll}
% \verb![F]! & \verb|!$xmp tasklet| {\tt barrier} {\openb}{\tt on} {\it nodes-ref}
%  $\vert${\it template-ref}{\closeb} \\
% \verb![C]! & \verb|#pragma xmp tasklet| {\tt barrier} {\openb}{\tt on} {\it
%      nodes-ref} $\vert$ {\it template-ref}{\closeb} \\
% \end{tabular}

% \subsubsection*{Description}


% \subsubsection*{Restriction}

% \begin{itemize}
% \item The node set specified by the {\tt on} clause must be a subset of the
%       executing node set.  
% \end{itemize}


%%%%%%%%%%%%%%%%%%%%%%%%%%%%%%%%%%%%%%%%%%%%%
%
% tasklet bcast
%

\subsubsection{{\tt tasklet bcast} Construct}

\subsubsection*{Synopsis}

The \verb|tasklet bcast| construct defines tasklets that perform
broadcast communication from a specified node like the \verb|bcast|
construct.

\subsubsection*{Syntax}
\Syntax{tasklet bcast}

\begin{tabular}{ll}
 \verb![F]! & \verb|!$xmp tasklet| {\tt bcast} \verb|(| {\it variable} 
 {\openb}, {\it variable}{\closeb}... \verb|)|
 {\openb}{\tt from} {\it nodes-ref} $\vert$ {\it template-ref}{\closeb}
 {\bsquare} \\
 & \hspace{5cm} {\bsquare} {\openb}{\tt on} {\it nodes-ref}{\closeb}
     $\vert$ {\it template-ref}{\closeb} \\

 \verb![C]! & \verb|#pragma xmp tasklet| {\tt bcast} \verb|(| {\it variable} 
 {\openb}, {\it variable}{\closeb}... \verb|)|
 {\openb}{\tt from} {\it nodes-ref}  $\vert$ {\it
     template-ref}{\closeb} {\bsquare} \\
 & \hspace{5cm} {\bsquare} {\openb}{\tt on} {\it nodes-ref} $\vert$ {\it
     template-ref}{\closeb} \\
\end{tabular}

\subsubsection*{Description}

The \verb|tasklet bcast| construct generates a tasklet on each of the
nodes specified by the \verb|on| clause, which will produce the same
effect as the \verb|bcast| construct having the same clauses does. In
addition, for each of the variables specified by the sequence of {\it
variables}, an \verb|inout| clause for it implicitly added to the
generated tasklets.

Note that \verb|tasklet bcast| is a local construct, unlike
\verb|bcast|.

\subsubsection*{Restrictions}

\begin{itemize}
 \item The variables specified by the sequence of {\it variables} must
       either not be aligned or must be replicated among nodes of the
       node set specified by the {\tt on} clause.
 \item The node set specified by the {\tt on} clause must be a subset of
       the executing node set.
 \item The source node specified by the {\tt from} clause must belong to
       the node set specified by the {\tt on} clause.
 \item The source node specified by the {\tt from} clause must be one
	   node.
\end{itemize}


%%%%%%%%%%%%%%%%%%%%%%%%%%%%%%%%%%%%%%%%%%%%%
%
% tasklet reduction
%

\subsubsection{{\tt tasklet reduction} Construct}

\subsubsection*{Synopsis}

The \verb|tasklet reduction| construct defines tasklets that perform a
reduction operation like the \verb|reduction| construct.

\subsubsection*{Syntax}
\Syntax{tasklet reduction}

\begin{tabular}{ll}
\verb![F]! & \verb|!$xmp tasklet| {\tt reduction (} {\it reduction-kind} {\it
  :} {\it variable} {\openb}, {\it variable} {\closeb}... {\tt )}
 {\bsquare} \\
 & \hspace{6cm} {\bsquare} {\openb}{\tt on} {\it node-ref} $\vert$ {\it
     template-ref}{\closeb} \\
\verb![C]! & \verb|#pragma xmp tasklet| {\tt reduction (} {\it reduction-kind} {\it
  :} {\it variable} {\openb}, {\it variable} {\closeb}... {\tt )}
 {\bsquare} \\
 & \hspace{6cm} {\bsquare} {\openb}{\tt on} {\it node-ref} $\vert$ {\it
     template-ref}{\closeb} \\
\end{tabular}

\subsubsection*{Description}

The \verb|tasklet reduction| construct generates a tasklet on each of
the nodes specified by the \verb|on| clause, which will produce the same
effect as the \verb|reduction| construct having the same clauses
does. In addition, for each of the variables specified by the sequence
of {\it variables}, an \verb|inout| clause for it implicitly added to
the generated tasklets.

Note that \verb|tasklet reduction| is a local construct, unlike
\verb|reduction|.

\subsubsection*{Restrictions}

\begin{itemize}
 \item The variables specified by the sequence of {\it variables} must
       either not be aligned or must be replicated among nodes of the node
       set specified by the {\tt on} clause.
 \item The node set specified by the {\tt on} clause must be a subset of the
       executing node set.
\end{itemize}

%%%%%%%%%%%%%%%%%%%%%%%%%%%%%%%%%%%%%%%%%%%%%
%
% tasklet reduce_shadow
%

\subsubsection{{\tt tasklet reduce\_shadow} Construct}

\subsubsection*{Synopsis}

The \verb|tasklet reduce_shadow| construct defines tasklets that add
values of shadow objects to their reflection source like the
\verb|reduce_shadow| construct.

\subsubsection*{Syntax}
\Syntax{tasklet reduce\_shadow}

\begin{tabular}{ll}
 \verb![F]! & \verb|!$xmp tasklet| {\tt reduce\_shadow} \verb|(| {\it array-name}
 {\openb}, {\it array-name}{\closeb}... \verb|)| {\bsquare} \\
 &\hspace{0.3cm} {\bsquare} {\openb}{\tt width (} {\it reflect-width}
     {\openb}, {\it reflect-width}{\closeb}... {\tt )}{\closeb}
     {\openb}{\tt orthogonal}{\closeb} {\bsquare} \\
 &\hspace{0.3cm} {\bsquare} {\openb}{\tt on} \{{\it nodes-ref} $\vert$
	  {\it template-ref}\}{\closeb} \\
\verb![C]! & \verb|#pragma xmp tasklet| {\tt reduce\_shadow} \verb|(| {\it array-name}
     {\openb}, {\it array-name}{\closeb}... \verb|)| {\bsquare} \\
 &\hspace{0.3cm} {\bsquare} {\openb}{\tt width (} {\it reflect-width}
     {\openb}, {\it reflect-width}{\closeb}... {\tt )}{\closeb}
     {\openb}{\tt orthogonal}{\closeb} {\bsquare} \\
 &\hspace{0.3cm} {\bsquare} {\openb}{\tt on} \{{\it nodes-ref} $\vert$
	  {\it template-ref}\}{\closeb} \\
\end{tabular}

\subsubsection*{Description}

The \verb|tasklet reduce_shadow| construct generates a tasklet on each
of the nodes specified by the \verb|on| clause, which will produce the
same effect as the \verb|reduce_shadow| construct having the same
clauses does. In addition, for each of the arrays specified by the
sequence of {\it array-names}, an \verb|inout| clause for it implicitly
added to the generated tasklets.

Note that \verb|tasklet reduce_shadow| is a local construct, unlike
\verb|reduce_shadow|, and therefore the node set specified by the
\verb|on| clause need not include all of the nodes onto which the target
arrays are mapped.

\subsubsection*{Restrictions}

\begin{itemize}
 % \item The arrays specified by the sequence of {\it array-names} must
 %       be mapped onto the executing node set.
 \item The width of each dimension specified by {\it
       reflect-width} must not exceed the shadow width of the arrays.
 \item The node set specified by the {\tt on} clause must be a subset of the
       executing node set.
\end{itemize}
