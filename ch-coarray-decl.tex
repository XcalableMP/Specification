%-------------------------------------------------------%
\section{Declaration of Coarrays}
\label{sec:Declaration of Coarrays}
%-------------------------------------------------------%



%============================================
\subsection{Declaration Statement of Coarray}
\label{sec:Declaration Statement of Coarray}
%============================================

\subsubsection*{Synopsis}
Declarations of a variable are extended to add the codimension attribute.

{\onlyF}
The {\it type-declaration-stmt} is extended with {\it coarray-spec}.
An {\it object-name} declared with a {\it coarray-spec} is a coarray.

{\onlyC} 
the {\it init-declarator} of the {\it declaration} and 
the {\it parameter-declaration} are extended with a {\it coarray-spec}.
If the {\it declarator} of them does not start with {\tt *},
the {\it identifier} declared with a {\it coarray-spec} is a coarray.
Else, 
the {\it identifier} declared with a {\it coarray-spec} is a coarray pointer
that may point an unnamed coarray.


\subsubsection*{Syntax \onlyF}
{\it coarray-spec} is adopted in the type declaration statement, as shown below.
As a kind of {\it attr-spec}, the CODIMENSION attribute specifier is added.
Besides, {\it entry-decl} is extended with {\it coarray-spec}.
Underlined parts are addition to the Fortran 90 standard.

\begin{center}
 \begin{tabular}{lll}
  {\it type-declaration-stmt} &  {\bf is} & 
  {\it declaration-type-spec} 
  {\openb}\/{\openb}\/, {\it attr-spec}\/{\closeb}\/{\tt ...} {\tt ::} {\closeb}\/
  {\it entity-decl-list}\\
  \\
  {\it attr-spec} & {\bf is} & {\tt ...}\\
  & \multicolumn{2}{l}{\tt ...}\\
  & \underline{\bf or} & 
    \underline{{\tt CODIMENSION} {\tt [} {\it coarray-spec} {\tt ]}}\\
  \\
  {\it entity-decl} & {\bf is} & {\it object-name}
  {\openb}\/{\tt (} {\it array-spec}\/ {\tt )}{\closeb}\/
  \underline{{\openb}\/{\tt [} {\it coarray-spec}\/ {\tt ]}{\closeb}\/} ~~{\bsquare}
  \\
  \multicolumn{3}{r}{\hfill{\bsquare}~~
  {\openb}\/{\tt *} {\it char-length} {\closeb}\/
  {\openb}\/{\it initialization} {\closeb}}
  \\
  & {\bf or} & {\it function-name} {\openb}\/{\tt *} {\it char-length} {\closeb}
 \end{tabular}
\end{center}
%
{\it coarray-spec} is defined as follows:
%
\begin{center}
 \begin{tabular}{lll}
  {\it coarray-spec} & {\bf is} & {\it explicit-coshape-spec} \\
                     & {\bf or} & {\it deferred-coshape-spec}
 \end{tabular}
\end{center}
%
{\it explicit-coshape-spec} and {\it deferred-coshape-spec} is defined in 
Section~\ref{sec:Explicit coshape} and
Section~\ref{sec:Deferred coshape}, respectively.


\subsubsection*{Syntax \onlyC}
{\it coarray-spec} is adopted in {\it declaration},
as an underlined part shown below:
%
\begin{center}
 \begin{tabular}{lll}
  {\it declaration} & {\bf is} & {\it declaration-specifiers} 
    {\openb}\/{\it init-declarator-list}\/{\closeb} {\tt ;} \\
  {\it init-declarator} & {\bf is} & {\it declarator}
    \underline{{\openb}\/{\tt :} {\it coarray-spec} {\closeb}}
    {\openb} {\tt =} {\it initializer} {\closeb} \\
  \\
  {\it parameter-declaration} & {\bf is} & {\it declaration-specifiers}
    {\it declarator}\/
    \underline{{\openb}\/{\tt :} {\it coarray-spec} {\closeb}}\\
  & {\bf or} & {\it declaration-specifiers}
    {\openb} {\it abstract-declarator} {\closeb}\\
 \\
  {\it declarator} & {\bf is} &
    {\openb} {\it pointer} {\closeb} {\it direct-declarator} \\
  \\
  {\it pointer} & {\bf is} &
    {\tt *} {\openb} {\it type-qualifier} {\closeb}\/{\tt ...}\\
  & {\bf or} &
    {\tt *} {\openb} {\it type-qualifier} {\closeb}\/{\tt ...} {\it pointer}\\
 \end{tabular}
\end{center}
%
\begin{center}
 \begin{tabular}{lll}
  {\it direct-declarator} & {\bf is} & {\it identifier} \\
  & {\bf or} & {\tt (} {\it declarator}\/ {\tt )} \\
  & {\bf or} & {\it direct-declarator} 
      {\tt [} 
         {\openb}\,{\it type-qualifier}\,{\closeb}\/{\tt ...}
         {\openb}\,{\it assignment-expression}\,{\closeb}\/
      {\tt ]} \\
  & {\bf or} & {\it direct-declarator} 
      {\tt [}
         {\tt static} {\openb}\,{\it type-qualifier}\,{\closeb}\/{\tt ...}
         {\it assignment-expression}
      {\tt ]} \\
  & {\bf or} & {\it direct-declarator} 
      {\tt [}
         {\it type-qualifier}\,{\tt ...} {\tt static}
         {\it assignment-expression}
      {\tt ]} \\
  & {\bf or} & {\it direct-declarator} 
      {\tt [}
         {\openb}\,{\it type-qualifier}\,{\closeb}\/{\tt ...} {\tt *}
      {\tt ]} \\
  & {\bf or} & {\it direct-declarator} 
      {\tt (} {\it parameter-type-list}\/ {\tt )} \\
  & {\bf or} & {\it direct-declarator} 
      {\tt (} {\openb} {\it identifier-list} {\closeb}\/ {\tt )}\\
 \end{tabular}
\end{center}
%
{\it coarray-spec} is defined as follows:
%
\begin{center}
 \begin{tabular}{lll}
  {\it coarray-spec} & {\bf is} & {\it explicit-coshape-spec}\\
                     & {\bf or} & {\it deferred-coshape-spec}\\
 \end{tabular}
\end{center}
%

\begin{Constraints F}
\item A coarray shall be a dummy argument or have the ALLOCATABLE or SAVE 
attribute.\footnote
{In other words, a local coarray to the procedure should be a SAVE'd or allocatable
variable unless it is a dummy argument.}

\item A coarray shall not be a function result.

\item A coarray shall not be a named constant or a pointer. 

\item A coarray shall not be a {\it common-block-object}
or a {\it equivalence-object}.

\item The VOLATILE attribute shall not be specified for a coarray that is 
associated by use or host association. %-- C560

\item Within a BLOCK construct, the VOLATILE attribute shall not be specified
for a coarray that is not a construct entity of that construct.  %-- C561

\end{Constraints F}

\begin{Constraints C}
\item A coarray shall be a dummy argument or have the {\tt static} or {\tt extern} 
storage class.\footnote
{Conversely, a coarray may not have the {\tt auto} or {\tt register} 
storage class.}

\item A coarray shall not be a function. 
A coarray pointer shall not be a pointer to a function.

\item A coarray shall not be of an {\tt enum} or {\tt union} type.

\item A {\it declaration-specifiers} of a {\it declaration} or 
{\it parameter-declaration} shall not contain the {\tt volatile} type qualifier.\footnote
{Because it is difficult to allow the access to the coarray from outside of the 
language system.
The {\tt const} qualifier here asserts the coarray data is not be modified 
like INTENT(IN) of Fortran.}

\item For a coarray pointer, the most right {\it pointer} of a {\it declarator} 
shall not contain the {\tt volatile} type qualifier.\footnote
{It is difficult to allow coarrays to be disturbed from the outside of the language.}

\item For an array coarray, the outermost (the most left) brakets shall not contain the {\tt volatile} type qualifier.

\end{Constraints C}


\subsubsection*{Description}

{\onlyF} The entity declared with a {\it coarray-spec} is a coarray, and
the {\it coarray-spec} specifies the coshape of the coarray.
A coarray is a scalar or an array data object and is of a basic or derived type.
The specification of the {\it coarray-spec} in the {\it entity-decl} 
overrides the specification of the {\it coarray-spec} in the {\it attr-spec}
if both are specified.

{\onlyC} The entity of a basic or structure type declared with a {\it coarray-spec} 
is a coarray, and the {\it coarray-spec} specifies the coshape of the coarray.
The entity of a pointer type declared with a {\it coarray-spec} is a coarray pointer, 
and the {\it coarray-spec} specifies the coshape of the coarray that the coarray pointer points.

A coarray can be initialized. 
Each image can initialize coarrays on the image and cannot initialize
any coarrays on the other images.

A declaration of a coarray has either an {\it explicit-coshape-spec} 
or a {\it deferred-coshape-spec} as the {\it coarray-spec}.
A static coarray and a dummy coarray are declarad with an 
{\it explicit-coshape-spec} (Section~\ref{sec:Explicit coshape}).
An allocatable coaray and a coarray pointer is declarad with a 
{\it deferred-coshape-spec} (Section~\ref{sec:Deferred coshape}).



%============================================
\subsection{Explicit coshape}
\label{sec:Explicit coshape}
%============================================

\subsubsection*{Synopsis}

A static coarray and a dummy coarray (Section~\ref{sec:Introduction for Coarrays})
are declared with a {\it coarray-spec} that is an {\it explicit-coshape-spec}.


\subsubsection*{Syntax \onlyF}

\begin{quote}
 \begin{tabular}{lll}
  {\it explicit-coshape-spec} & {\bf is} & 
     {\openb} 
       {\openb} {\it lower-cobound} {\tt :}{\closeb} {\it upper-cobound}{\tt ,}
     {\closeb}{\tt ...}
     {\openb} {\it lower-cobound} {\tt :}{\closeb} {\tt *} \\
  {\it lower-cobound} & {\bf is} & {\it specification-expr} \\
  {\it upper-cobound} & {\bf is} & {\it specification-expr}
 \end{tabular}
\end{quote}


\subsubsection*{Syntax \onlyC}

\begin{quote}
 \begin{tabular}{lll}
  {\it explicit-coshape-spec} & {\bf is} & 
     {\tt [}\,{\tt *}\,{\tt ]}
     {\openb\/} {\tt [}\,{\it coextent\/}\,{\tt ]} {\closeb\/}{\tt ...} \\
  {\it coextent} & {\bf is} & {\it assignment-expression} \\
 \end{tabular}
\end{quote}

\begin{Constraints F}
\item 
A nonallocatable coarray shall have a {\it coarray-spec} that is an 
{\it explicit-coshape-spec}.

\item
A lower-cobound or upper-cobound that is not a constant expression shall appear 
only in a subprogram, BLOCK construct, or interface body.

\item
The upper cobound shall not be less than the lower cobound.

\end{Constraints F}

\begin{Constraints C}
\item
If a {\it coarray-spec} appearing in an {\it init-declarator} or 
a {\it parameter-declaration} is an {\it explicit-coshape-spec}, 
the {\it declarator} followed by the {\it coarray-spec} shall be
in the following format:\footnote
{Keyword {\tt static} and {\it assignment-expression}s in ``{\tt [ ]}''
are not yet taken into account!}

\begin{quote}
 \begin{tabular}{lll}
  {\it declarator} & {\bf is} & {\it identifier} 
    {\openb\/} {\tt [} {\it assignment-expression\/} {\tt ]} {\closeb\/}{\tt ...}
 \end{tabular}
\end{quote}

\item
A {\it coextent} that is not a constant expression shall appear 
only in a block scope or function prototype scope.

\end{Constraints C}


\subsubsection*{Description}

An explicit coshape is valid for the scope it is specified.
Non-constant values in the explicit coshape are 
evaluated at the entry of the block in {\XMPF} or at the 
executon of the declarattion in {\XMPC}.

{\onlyF}
An explicit coshape specifies the corank and the cobounds, 
except the upper cobound of the final (outermost) dimension.
If the lower cobound is omitted, the default value is 1.
For each dimension, \\
\(($coextent$) = ($upper cobound$) - ($lower cobound$) + 1.\)

{\onlyC}
An explicit coshape specifies the corank and the coextents,
except the extent of the final (outermost) dimension.
The lower cobound is always zero and the upper cobound is the same as 
the coextent for each dimension.



\subsubsection*{Example}

{\onlyF} The type declaration statement shown below declares a 
two-dimensional array coarray of real type. 
The {\it coarray-spec} is the notation ``{\tt [4,0:*]}''.
The corank is 2, the lower and upper cobounds of the first dimension 
are 1 (default) and 4, and the lower cobound of the second dimension is 0.

{\onlyC} The declaration shown below declares a two-dimensional array
coarray of float type. 
The {\it coarray-spec} is the notation ``{\tt [*][4]}''.
The corank is 2 and the coextents of the first dimension is 4.
Because the lower cobound is always 0, the lower and upper cobounds of 
the first dimension is 0 and 3, respectively.

\begin{center}
 \begin{minipage}{0.48\hsize}
  \begin{XFexample}
  real, save :: z(10,20)[4,0:*]
  \end{XFexample}
 \end{minipage}
%
 \begin{minipage}{0.48\hsize}
  \begin{XCexampleR}
  static float z[20][10]:[*][4];
  \end{XCexampleR}
 \end{minipage}
\end{center}


%============================================
\subsection{Deferred coshape}
\label{sec:Deferred coshape}
%============================================

\subsubsection*{Synopsis}

An allocatable coarray and a coarray pointer 
(Section~\ref{sec:Introduction for Coarrays}) are declared with 
a {\it coarray-spec} that is an {\it deferred-coshape-spec}.


\subsubsection*{Syntax}

\begin{quote}
 \begin{tabular}{llll}
  {\onlyF} & {\it deferred-coshape-spec} & {\bf is} & 
    {\openb}\/{\tt :,} {\closeb}\/{\tt ...} {\tt :} \\
  \\
  {\onlyC} & {\it deferred-coshape-spec} & {\bf is} &
%%    \{{\tt [ ]}\}{\tt ...}
    {\tt [ ]} {\tt ...}
 \end{tabular}
\end{quote}

\begin{Constraints F}
\item 
A coarray with the ALLOCATABLE attribute shall have a {\it coarray-spec}
that is a {\it deferred-coshape-spec}.

\end{Constraints F}


\begin{Constraints C}
\item 
If a {\it coarray-spec} appearing in an {\it init-declarator} or 
a {\it parameter-declaration} is an {\it deferred-coshape-spec}, 
the {\it declarator} followed by the {\it coarray-spec} shall be
a pointer.

\end{Constraints C}


\subsubsection*{Description}

A deferred coshape specifies the corank of the coarray but does not
specify the cobounds or the coextents, which are
determined by allocation or argument association.
If the coarray or the coarray pointer is a dummy argument,
the coshape is inherited from the corresponding actual argument.



\subsubsection*{Example}

{\onlyF} The type declaration statement shown below declares a 
two-dimensional array coarray of real type. 
The {\it coarray-spec} of {\tt z} is the notation ``{\tt [:,:]}'',
and the corank is 2.
The {\it coarray-spec} of {\tt x} is the notation ``{\tt [:]}'',
and the corank is 1.
All upper and lower cobounds are always unknown in deferred coshape.

{\onlyC} The declaration shown below declares a two-dimensional array
coarray of float type. 
The {\it coarray-spec} of {\tt z} is the notation ``{\tt [][]}'',
and the corank is 2.
The {\it coarray-spec} of {\tt x} is the notation ``{\tt []}'',
and the corank is 1.
All coextents are always unknown in deferred coshape.

\begin{center}
 \begin{minipage}{0.48\hsize}
  \begin{XFexample}
  real, allocatable :: z(:,:)[:,:]
  type(my_t), allocatable :: x[:]
  \end{XFexample}
 \end{minipage}
%
 \begin{minipage}{0.48\hsize}
  \begin{XCexampleR}
  float (*z)[10]:[][];
  struct my_t *x:[];
  \end{XCexampleR}
 \end{minipage}
\end{center}


%============================================
\subsection{Coarray Container}
\label{sec:Coarray Container}
%============================================

{\onlyF}
A derived type that has a scalar coarray component is called 
a {\bf coarray container type}.
A derived type that has a scalar component of a coarray container type 
is also called a coarray container type.
A scalar object of a coarray container type is called 
a {\bf coarray container}.

{\onlyC}
A {\tt struct} type that has a coarray component is called 
a {\bf coarray container type}.
A {\tt struct} type that has a component of a coarray container type 
is also called a coarray container type.
A object of a coarray container type is called 
a {\bf coarray container}.

\begin{Constraints F}
\item A coarray container shall be a dummy argument or have 
the ALLOCATABLE or SAVE attribute.\footnote
{In other words, a local coarray container to the procedure should be a SAVE'd or 
allocatable structure unless it is a dummy argument.}

\item A coarray container shall be a nonpointer nonallocatable scalar, 
shall not be a coarray, and shall not be a function result. 
\end{Constraints F}

\begin{Constraints C}
\item A coarray container shall be a dummy argument or have the {\tt static} 
or {\tt extern} storage class.\footnote
{Conversely, a coarray container may not have {\tt auto} storage class.}

\item A coarray container shall not be a coarray.

\end{Constraints C}


\subsubsection*{Example}

{\onlyF} Variable {\tt cc} is a structure container
and has a coarray structure component {\tt cc\%a}.
Variable {\tt dd} is also a structure container
and has ultimate coarray structure components {\tt dd\%x\%a}
and {\tt dd\%y\%a}.

{\onlyC} Variable {\tt cc} is a structure container
and has a coarray structure component {\tt cc.a}.
Variable {\tt dd} is also a structure container
and has ultimate coarray structure components {\tt dd.x.a}
and {\tt dd.y.a}.

\begin{center}
 \begin{minipage}{0.48\hsize}
  \begin{XFexample}
  type cc_t
    integer size
    real, allocate :: a(:)[:,:]
  end type
  type dd_t
    type(cc_t) :: x,y
  end type
  type(cc_t),save :: cc
  type(dd_t),save :: dd
  \end{XFexample}
 \end{minipage}
%
 \begin{minipage}{0.48\hsize}
  \begin{XCexampleR}
  typedef struct {
    int size;
    double (*a)[10]:[][];
  } cc_t;
  typedef struct {
    cc_t x, y;
  } dd_t;
  static cc_t cc;
  static dd_t dd;
  \end{XCexampleR}
 \end{minipage}
\end{center}


