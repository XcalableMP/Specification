%-------------------------------------------------------%
\section{Declaration of Coarrays}
\label{sec:Declaration of Coarrays}
%-------------------------------------------------------%

%============================================
\subsection{Declaration statement of coarray}
\label{sec:Declaration statement of coarray}
%============================================

\subsubsection*{Synopsis}
Declarations of a variable are extended to add the codimension attribute.

{\onlyF}
The {\it type-declaration-stmt} is extended with {\it coarray-spec}.
An {\it object-name} declared with a {\it coarray-spec} is a coarray.

{\onlyC} 
the {\it init-declarator} of the {\it declaration} and 
the {\it parameter-declaration} are extended with a {\it coarray-spec}.
If the {\it declarator} of them does not start with {\tt *},
the {\it identifier} declared with a {\it coarray-spec} is a coarray.
Else, 
the {\it identifier} declared with a {\it coarray-spec} is a coarray pointer
that may point an unnamed coarray.


\subsubsection*{Syntax \onlyF}
{\it coarray-spec} is adopted in the type declaration statement, as shown below.
As a kind of {\it attr-spec}, the CODIMENSION attribute specifier is added.
Besides, {\it entry-decl} is extended with {\it coarray-spec}.
Underlined parts are addition to the Fortran 90 standard.

\begin{center}
 \begin{tabular}{lll}
  {\it type-declaration-stmt} &  {\bf is} & 
  {\it declaration-type-spec} 
  {\openb}\/{\openb}\/, {\it attr-spec}\/{\closeb}\/{\tt ...} {\tt ::} {\closeb}\/
  {\it entity-decl-list}\\
  \\
  {\it attr-spec} & {\bf is} & {\tt ...}\\
  & {\bf or} & {\tt ...}\\
  & \multicolumn{2}{l}{\tt ...}\\
  & \underline{\bf or} & 
    \underline{{\tt CODIMENSION} {\tt [} {\it coarray-spec} {\tt ]}}\\
  \\
  {\it entity-decl} & {\bf is} & {\it object-name}
  {\openb}\/{\tt (} {\it array-spec}\/ {\tt )}{\closeb}\/
  \underline{{\openb}\/{\tt [} {\it coarray-spec}\/ {\tt ]}{\closeb}\/} ~~{\bsquare}
  \\
  \multicolumn{3}{r}{\hfill{\bsquare}~~
  {\openb}\/{\tt *} {\it char-length} {\closeb}\/
  {\openb}\/{\it initialization} {\closeb}}
  \\
  & {\bf or} & {\it function-name} {\openb}\/{\tt *} {\it char-length} {\closeb}
 \end{tabular}
\end{center}
%
{\it coarray-spec} is defined as follows:
%
\begin{center}
 \begin{tabular}{lll}
  {\it coarray-spec} & {\bf is} & {\it explicit-coshape-spec} \\
                     & {\bf or} & {\it deferred-coshape-spec}
 \end{tabular}
\end{center}
%
{\it explicit-coshape-spec} and {\it deferred-coshape-spec} is defined in 
Section~\ref{sec:Static coarray} and
Section~\ref{sec:Allocatable coarray}, respectively.


\subsubsection*{Syntax \onlyC}
{\it coarray-spec} is adopted in {\it declaration},
as an underlined part shown below:
%
\begin{center}
 \begin{tabular}{lll}
  {\it declaration} & {\bf is} & {\it declaration-specifiers} 
    {\openb}\/{\it init-declarator-list}\/{\closeb} {\tt ;} \\
  {\it init-declarator} & {\bf is} & {\it declarator}
    \underline{{\openb}\/{\tt :} {\it coarray-spec} {\closeb}}
    {\openb} {\tt =} {\it initializer} {\closeb} \\
  \\
  {\it parameter-declaration} & {\bf is} & {\it declaration-specifiers}
    {\it declarator}\/
    \underline{{\openb}\/{\tt :} {\it coarray-spec} {\closeb}}\\
  & {\bf or} & {\it declaration-specifiers}
    {\openb} {\it abstract-declarator} {\closeb}\\
 \\
  {\it declarator} & {\bf is} &
    {\openb} {\it pointer} {\closeb} {\it direct-declarator} \\
  \\
  {\it pointer} & {\bf is} &
    {\tt *} {\openb} {\it type-qualifier} {\closeb}\/{\tt ...}\\
  & {\bf or} &
    {\tt *} {\openb} {\it type-qualifier} {\closeb}\/{\tt ...} {\it pointer}\\
 \end{tabular}
\end{center}
%
\begin{center}
 \begin{tabular}{lll}
  {\it direct-declarator} & {\bf is} & {\it identifier} \\
  & {\bf or} & {\tt (} {\it declarator}\/ {\tt )} \\
  & {\bf or} & {\it direct-declarator} 
      {\tt [} 
         {\openb}\,{\it type-qualifier}\,{\closeb}\/{\tt ...}
         {\openb}\,{\it assignment-expression}\,{\closeb}\/
      {\tt ]} \\
  & {\bf or} & {\it direct-declarator} 
      {\tt [}
         {\tt static} {\openb}\,{\it type-qualifier}\,{\closeb}\/{\tt ...}
         {\it assignment-expression}
      {\tt ]} \\
  & {\bf or} & {\it direct-declarator} 
      {\tt [}
         {\it type-qualifier}\,{\tt ...} {\tt static}
         {\it assignment-expression}
      {\tt ]} \\
  & {\bf or} & {\it direct-declarator} 
      {\tt [}
         {\openb}\,{\it type-qualifier}\,{\closeb}\/{\tt ...} {\tt *}
      {\tt ]} \\
  & {\bf or} & {\it direct-declarator} 
      {\tt (} {\it parameter-type-list}\/ {\tt )} \\
  & {\bf or} & {\it direct-declarator} 
      {\tt (} {\openb} {\it identifier-list} {\closeb}\/ {\tt )}\\
 \end{tabular}
\end{center}
%
{\it coarray-spec} is defined as follows:
%
\begin{center}
 \begin{tabular}{lll}
  {\it coarray-spec} & {\bf is} & {\it explicit-coshape-spec}\\
                     & {\bf or} & {\it deferred-coshape-spec}\\
 \end{tabular}
\end{center}
%

\begin{Constraints F}
\item A coarray shall be a dummy argument or have the ALLOCATABLE or SAVE 
attribute.\footnote
{In other words, a local coarray to the procedure should be a SAVE'd or allocatable
variable unless it is a dummy argument.}

\item A coarray shall not be a function result.

\item A coarray shall not be a named constant or a pointer. 

\item A coarray shall not be a {\it common-block-object}
or a {\it equivalence-object}.

%-- COMMENT OUT. This is not in F90.
% \item The VOLATILE attribute shall not be specified for a coarray that is 
% associated by use or host association. %-- C560
%
% \item Within a BLOCK construct, the VOLATILE attribute shall not be specified
% for a coarray that is not a construct entity of that construct.
%

\end{Constraints F}

\begin{Constraints C}
\item A coarray shall be a dummy argument or have the {\tt static} or {\tt extern} 
storage class.\footnote
{Conversely, a coarray may not have the {\tt auto} or {\tt register} 
storage class.}

\item A coarray shall not be a function. 
A coarray pointer shall not be a pointer to a function.

\item A coarray shall not be of an {\tt enum} or {\tt union} type.

\item A {\it declaration-specifiers} of a {\it declaration} or 
{\it parameter-declaration} shall not contain the {\tt volatile} type qualifier.\footnote
{Because it is difficult to allow the access to the coarray from outside of the 
language system.
The {\tt const} qualifier here asserts the coarray data is not be modified 
like INTENT(IN) of Fortran.}

\item For a coarray pointer, the most right {\it pointer} of a {\it declarator} 
shall not contain the {\tt volatile} or {\tt restrict} type qualifier.\footnote
{Because it is difficult to permit .......}

\item For an array coarray, the {\it type-qualifier}s and the {\tt static} keyword
apearing inside the outermost (the most left) brackets shall meeet the following conditions.
 \begin{itemize}
 \item The {\tt const} qualifier
 \end{itemize}

\end{Constraints C}


\subsubsection*{Description}

The entity that is declared
by a {\it type-declaration-statement} (in {\XMPF}) or 
by a {\it declaration} (in {\XMPC})
with {\it coarray-spec} is a coarray variable.

{\onlyF} A coarray variable is a coarray itself.
A coarray is a scalar or an array data object and is of a basic or derived type.
The specification of {\it coarray-spec} in {\it entity-decl} 
overrides the specification of {\it coarray-spec} in {\it attr-spec}
if both are specified.

{\onlyC} \NEW{A coarray in {\XMPC} is a single or sequence of a data object 
each of which is of a basic type or of a {\tt struct} type.}
A coarray variable of a basic type or a {\tt struct} type is a coarray 
as a single data object.
A coarray variable of an array or of a pointer is a (nested) pointer 
to a coarray.

A coarray can be initialized. 
Each image can initialize coarrays on the image and cannot initialize
any coarrays on the other images.


{\onlyF} A type declaration statement 
with the specification of the codimension attribute
defines the coshape of the specified variable.

{\onlyC} If the specified variable is of a basic type or of a {\tt struct} type,
a declaration with the specification of the codimension attribute
defines the coshape of the specified variable.
If the specified variable is of an array or of a pointer,
a declaration with the specification of the codimension attribute
defines the coshape of the coarray corresponding to the specified variable.


A declaration of a coarray has either an {\it explicit-coshape-spec} 
or a {\it deferred-coshape-spec} as the {\it coarray-spec}.
A static coarray (Section~\ref{sec:Static coarray}) is declarad 
with an {\it explicit-coshape-spec} and
an allocatable coaray (Section~\ref{sec:Allocatable coarray}) is declarad 
with a {\it deferred-coshape-spec}.



{\onlyF} 
An explicit coshape specifies the corank and the cobounds, 
except the upper cobound of the final (outermost) dimension.
For each dimension, 
\(($coextent$) = ($upper cobound$) - ($lower cobound$) + 1.\)
A deffered coshape specifies the corank and does not specify cobounds.

{\onlyC}
An explicit coshape specifies the corank and the coextents,
except the extent of the final (outermost) dimension.
The lower cobound is always zero and the upper cobound is the same as 
the coextent for each dimension.
A deffered coshape specifies the corank and does not specify coextents.


The corank and coextents of an explicit coshape is explicitly declared
for the scope.
An explicit coshape of a coarray has a constant corank and 
constant coextents in the scope.
A deferred coshape of a coarray has a constant corank and
unknown coextents 

%A coarray is either an {\bf explicit-coshape coarray} or a 
%{\bf deferred-coshape coarray}.

A coshape of a coarray is either explicit or deferred in the scope.
%
The corank and coextents of the explicit coshape
are specified in the declaration of the coarray.
They are determined at compile 

The coextents of the explicit coshape are specified in the declaration of
the coarray and determined at compile time or at the begining of the scope.
%
The coextents of the deferred coshape are determined when the coarray 
is allocated. 


explicitly declared 
and is constant in the scope.
The coextents of the deferred coshape are 
It is 


The corank of the deferred coshape is 

An explicit-coshape coarray has a coshape whose corank and coextents 

The coextents of a coarray are explicitly declarad 

 in the scope, and
a differed coshape has 





\paragraph*{NOTE}
In {\XMPC}, after the following declaration:
\begin{verbatim}
            static double a1[10]:[*][4];
\end{verbatim}
{\tt a1} is not a coarray but a pointer to the coarray on this image.
The coarray is referred as {\tt a1[:]} using the notation defined in 
Chapter~\ref{chap:Base Language Extensions in {\XMPC}}.
In contrast, for a basic type variable {\tt v1} and a {\tt struct} 
variable {\tt s1}:
\begin{verbatim}
            static int b1:[*][4];
            static struct {int n; double a;} s1:[*][4];
\end{verbatim}
variables {\tt b1} and {\tt s1} are the coarrays.



\begin{Restriction}
\item 
The sum of the rank\footnote{the number of dimensions} and 
corank\footnote{the number of codimensions} of a coarray 
shall not exceed fifteen.

\end{Restriction}

\subsubsection*{Example}
See Examples in Sections~\ref{sec:Static coarray} and 
\ref{sec:Allocatable coarray}.


%============================================
\subsection{Static coarray}
\label{sec:Static coarray}
%============================================

\subsubsection*{Synopsis}

A static coarray is a coarray that has a static data and coshape during execution of the program.

An explicit-coshape coarray is a coarray that has its corank and cobounds,
which are declared with a {\it coarray-spec} that is an {\it explicit-coshape-spec}
for the coarray variable.

\subsubsection*{Syntax \onlyF}

\begin{quote}
 \begin{tabular}{lll}
  {\it explicit-coshape-spec} & {\bf is} & 
     {\openb} 
       {\openb} {\it lower-cobound} {\tt :}{\closeb} {\it upper-cobound}{\tt ,}
     {\closeb}{\tt ...}
     {\openb} {\it lower-cobound} {\tt :}{\closeb} {\tt *} \\
  {\it lower-cobound} & {\bf is} & {\it specification-expr} \\
  {\it upper-cobound} & {\bf is} & {\it specification-expr}
 \end{tabular}
\end{quote}


\subsubsection*{Syntax \onlyC}

\begin{quote}
 \begin{tabular}{lll}
  {\it explicit-coshape-spec} & {\bf is} & 
     {\tt [}\,{\tt *}\,{\tt ]}
     {\openb\/} {\tt [}\,{\it coextent\/}\,{\tt ]} {\closeb\/}{\tt ...} \\
  {\it coextent} & {\bf is} & {\it assignment-expression} \\
 \end{tabular}
\end{quote}

\begin{Constraints F}
\item 
A nonallocatable coarray shall have a {\it coarray-spec} that is an 
{\it explicit-coshape-spec}.

\item
A lower-cobound or upper-cobound that is not a constant expression shall appear 
only in a subprogram, BLOCK construct, or interface body.

\end{Constraints F}

\begin{Constraints C}
\item
If a {\it coarray-spec} appearing in an {\it init-declarator} or 
a {\it parameter-declaration} is an {\it explicit-coshape-spec}, 
the {\it declarator} followed by the {\it coarray-spec} shall be
in the following format:\footnote
{Keyword {\tt static} and {\it assignment-expression}s in ``{\tt [ ]}''
are not yet taken into account!}

\begin{quote}
 \begin{tabular}{lll}
  {\it declarator} & {\bf is} & {\it identifier} 
    {\openb\/} {\tt [} {\it assignment-expression\/} {\tt ]} {\closeb\/}{\tt ...}
 \end{tabular}
\end{quote}

\item
A {\it coextent} that is not a constant expression shall appear 
only in a block scope or function prototype scope.

\end{Constraints C}


\subsubsection*{Description}

{\onlyF}
Cobounds of an explicit-coshape coarray are determined on entry to the procedure
if they are not constant expressions.
If the lower cobound is omitted, the default value is 1.
The upper cobound shall not be less than the lower cobound.
A cosubscript of the coarray in that codimension shall not be less than the lower
cobound or greater than the upper cobound.

{\onlyC}
Coextents of an explicit-coshape coarray are determined on entry to the scope 
if they are not constant expressions.
A cosubscript of the coarray in that codimension shall be less than the coextent
and shall not be less than 0.


%============================================
\subsection{Allocatable coarray}
\label{sec:Allocatable coarray}
%============================================

\subsubsection*{Synopsis}

A deferred-coshape coarray is a coarray that has its corank 
declared by a {\it deferred-coshape-spec}.
Its cobounds are determined by allocation or argument association.

\subsubsection*{Syntax}

\begin{quote}
 \begin{tabular}{llll}
  {\onlyF} & {\it deferred-coshape-spec} & {\bf is} & 
    {\openb}\/{\tt :,} {\closeb}\/{\tt ...} {\tt :} \\
  \\
  {\onlyC} & {\it deferred-coshape-spec} & {\bf is} &
%%    \{{\tt [ ]}\}{\tt ...}
    {\tt [ ]} {\tt ...}
 \end{tabular}
\end{quote}

\begin{Constraints F}
\item 
A coarray with the ALLOCATABLE attribute shall have a {\it coarray-spec}
that is a {\it deferred-coshape-spec}.

\end{Constraints F}


\begin{Constraints C}
\item 
If a {\it coarray-spec} appearing in an {\it init-declarator} or 
a {\it parameter-declaration} is an {\it deferred-coshape-spec}, 
the {\it declarator} followed by the {\it coarray-spec} shall be
a pointer.

\end{Constraints C}


\subsubsection*{Description}

{\onlyF}
The cobounds of an unallocated allocatable coarray are undefined.
The cobounds of an allocated allocatable coarray are those specified 
when the coarray is allocated.

{\onlyC}


%============================================
\subsection{Volatile {\onlyF} Attribute/{\onlyC} Type Qualifier}
\label{sec:Volatile}
%============================================

C560 The VOLATILE attribute shall not be specified for a coarray that is accessed by use (11.2.2) or host (16.5.1.4) association.  %-- 5.3.19 VOLATILE attribute

C561 Within a BLOCK construct (8.1.4), the VOLATILE attribute shall not be specified for a coarray that is not a construct entity (16.4) of that construct.  %-- 5.3.19 VOLATILE attribute

If the target of a pointer is a coarray, the pointer shall have the VOLATILE attribute if and only if the coarray has the VOLATILE attribute.  %-- 7.2.2.3-8

