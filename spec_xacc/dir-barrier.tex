\section{{\tt barrier} Construct}

\subsection*{Synopsis}

The {\tt \Directive{barrier}} construct specifies an explicit barrier
at the point at which the construct appears.

\subsection*{Syntax}
\Syntax{barrier}

\begin{tabular}{ll}
\verb![F]! & \verb|!$xmp| {\tt barrier} {\openb}{\tt on} {\it nodes-ref}
 $\vert${\it template-ref}{\closeb} \\
\verb![C]! & \verb|#pragma xmp| {\tt barrier} {\openb}{\tt on} {\it
     nodes-ref} $\vert$ {\it template-ref}{\closeb} \\
\end{tabular}

\subsection*{Description}

The barrier operation is performed among the node set specified by
the {\tt on} clause. If no {\tt on} clause is specified, then it is
assumed that the current executing node set is specified in it.

Note that an {\tt on} clause may represent multiple node sets. In such a
case, a barrier operation is performed in each node set.

%The barrier construct also has the function of ensuring that all of the
%remote copy operations that are invoked by gmove in/out constructs
%executed by the node set specified by the {\tt on} clause are finished.

\subsection*{Restriction}

\begin{itemize}
\item The node set specified by the {\tt on} clause must be a subset of the
      executing node set.  
\end{itemize}
