\section{\Directive{device} Directive}

\subsubsection*{Synopsis}

The {\tt device} directive declares a named node array.

\subsubsection*{Syntax}
\Syntax{nodes}

\begin{tabular}{ll}
\verb![F]!&\verb|!$xmp| {\tt nodes} {\it nodes-decl} {\openb},
 {\it nodes-decl} {\closeb}...\\
& \\
\verb![C]!&\verb|#pragma xmp| {\tt nodes} {\it nodes-decl} {\openb},
 {\it nodes-decl} {\closeb}...\\
\end{tabular}

\vspace{0.3cm}

where {\it nodes-decl} is one of:

\vspace{0.3cm}

\begin{tabular}{ll}
 \hspace{0.5cm} & {\it nodes-name} \verb|(| {\it nodes-spec} {\openb},
 {\it nodes-spec} {\closeb}... \verb|)| \\
 \hspace{0.5cm} & {\it nodes-name} \verb|(| {\it nodes-spec} {\openb},
     {\it nodes-spec} {\closeb}... \verb|)| {\tt =} {\it nodes-ref}
\end{tabular}

\vspace{0.3cm}

and {\it nodes-spec} must be one of:

\vspace{0.3cm}

\begin{tabular}{ll}
 \hspace{0.5cm} & {\it int-expr} \\
 \hspace{0.5cm} & {\tt *} \\
\end{tabular}


\subsubsection*{Description}

The {\tt nodes} directive declares a node array that corresponds to a
node set.

The first form of the {\tt nodes} directive is used to declare a node
array that corresponds to the entire node set.
The second form is used to declare a node array, each node of which is
assigned to a node of the node set specified by {\it nodes-ref} at the
corresponding position in Fortran's array element order, as if the node
set were a one-dimensional node array.

If {\it node-size} in the last dimension is ``{\tt *}'', then the size
of the node array is automatically adjusted according to the total size
of the entire node set in the first form, the executing node set in the
second form, or the referenced node set in the third form.

\subsubsection*{Restrictions}

\begin{itemize}
\item {\it nodes-spec} can be ``{\tt *}'' only in the last dimension.
\item {\it nodes-ref} must not reference {\it nodes-name} either
      directly or indirectly.
\item If no {\it nodes-spec} is ``{\tt *}'', then the product
      of all {\it nodes-spec} must be equal to the total size of the
      entire node set in the first form, the executing node set in the
      second form, or the referenced node set in the third form.
\item {\it nodes-subscript} in {\it nodes-ref} must not be ``{\tt *}''.
\end{itemize}

\subsubsection*{Examples}
\Example{nodes}

The following are examples of the first and the third forms appeared in
the main program. Since the node array {\tt p}, which corresponds to the
entire node set, is declared to be of size 16, this program must be
executed by 16 nodes.

\vspace{0.5cm}

\begin{minipage}{0.45\hsize}
\begin{center}
\begin{XFexample}
      program main
!$xmp nodes p(16)
!$xmp nodes q(4,*)
!$xmp nodes r(8)=p(3:10)
!$xmp nodes z(2,3)=p(1:6)
      ...       
      end program 
\end{XFexample}
\end{center}
\end{minipage}
%
\begin{minipage}{0.45\hsize}
\begin{center}
\begin{XCexampleR}
int main() {
#pragma xmp nodes p(16)
#pragma xmp nodes q(4,*)
#pragma xmp nodes r(8)=p(3:10)
#pragma xmp nodes z(2,3)=p(1:6)
    ...
}
\end{XCexampleR}
\end{center}
\end{minipage}

\vspace{0.5cm}

The following is an example of a node declaration in a procedure.
Since {\tt p} is declared in the second form to be of size 16 and
corresponds to the executing node set, the invocation of the {\tt foo}
function must be executed by 16 nodes.
%
The node array {\tt q} is declared in the first form and corresponds to
the entire node set. The node array {\tt r} is declared as a subset of
{\tt p}, and {\tt x} as a subset of {\tt q}.

\begin{XFexample}
      function foo()
!$xmp nodes p(16)=*
!$xmp nodes q(4,*)
!$xmp nodes r(8)=p(3:10)
!$xmp nodes x(2,3)=q(1:2,1:3)
      ...
      end function
\end{XFexample}


\subsection{Device Reference}

\subsubsection*{Synopsis}

The \Term{node reference} is used to reference a node set.

\subsubsection*{Syntax}
\index{node reference}
\index{Syntax!node reference}

A node reference {\it nodes-ref} is specified by either the name of a
node array, the ``{\tt *}'' symbol or ``{\tt **}''.

\begin{center}
\begin{tabular}{lll}
{\it nodes-ref} & {\bf is} & {\it nodes-name} {\openb}\verb|(| {\it nodes-subscript}
	 {\openb}, {\it nodes-subscript} {\closeb}... \verb|)|{\closeb} \\
                & {\bf or} & {\tt *} \\
                & {\bf or} & {\tt **}
\end{tabular}
\end{center}
%
\vspace{0.3cm}
%
where {\it nodes-subscript} must be one of:

\hspace{\hsize}

\begin{tabular}{ll}
 \hspace{0.5cm} & {\it int-expr} \\
 \hspace{0.5cm} & {\it triplet} \\
 \hspace{0.5cm} & {\tt *} \\
\end{tabular}

\subsubsection*{Description}

A node reference by {\it nodes-name} represents a node set corresponding
to the node array specified by the name or its subarray, which is
totally ordered in Fortran's array element order.
%
A node reference by ``{\tt *}''
represents the executing node set. A node reference by ``{\tt **}''
represents the primary node set.

Specifically, the ``{\tt *}'' symbol appeared as {\it nodes-subscript}
in a dimension of {\it nodes-ref} is interpreted by each node at runtime
as its position (coordinate) in the dimension of the referenced node
array.
%
Thus, a node reference {\tt p($s_1$, ..., $s_{k-1}$, *, $s_{k+1}$, ...,
$s_n$)} is interpreted as {\tt p($s_1$, ..., $s_{k-1}$, $j_k$,
$s_{k+1}$, ..., $s_n$)} on the node {\tt p($j_1$, ..., $j_{k-1}$, $j_k$,
$j_{k+1}$, ..., $j_n$)}.

Note that ``{\tt *}'' can be used only as the node reference in
the {\tt on} clause of some executable directives.

\subsubsection*{Examples}
\index{node reference}
\index{Example!node reference}

Assume that {\tt p} is the name of a node array and that {\tt m} is an
integer variable.

\begin{itemize}
\item As a target node array in the {\tt distribute} directive,

\Example{distribute}
\begin{tabular}{l}
\verb|!$xmp distribute a(block) onto p| \\
\end{tabular}%$

\item To specify a node set to which the declared node array corresponds
      in the second form of the {\tt nodes} directive,

\Example{nodes}
\begin{tabular}{l}
\verb|!$xmp nodes r(2,2,4) = p(1:4,1:4)| \\
\verb|!$xmp nodes r(2,2,4) = p(1:16)| \\
\end{tabular}

\item To specify a node array that corresponds to the executing node set
      of a task in the {\tt task} directive,

\Example{task}
\begin{tabular}{l}
\verb|!$xmp task on p(1:4,1:4)| \\
\verb|!$xmp task on p(1:16)| \\
\verb|!$xmp task on p(:,*)| \\
\verb|!$xmp task on (m)| \\
\end{tabular}

\item To specify a node array with which iterations of a loop are
      aligned in the {\tt loop} directive,

\Example{loop}
\begin{tabular}{l}
\verb|!$xmp loop (i) on p(lb(i):lb(i+1)-1)| \\
\end{tabular}%$

\item To specify a node array that corresponds to the executing node set
      in the {\tt barrier} and the {\tt reduction} directive,

%In {\tt barrier} directive and the {\tt reduction} directive,
%executing nodes are specified. 

\Example{barrier}
\Example{reduction}
\begin{tabular}{l}
\verb|!$xmp barrier on p(5:8)| \\
\verb|!$xmp reduction (+:a) on p(*,:)| \\
\end{tabular}

\item To specify the source node and the node array that corresponds to
      the executing node set in the {\tt bcast} directive,

%In the {\tt bcast} directive, a source node and executing nodes are specified.

\Example{bcast}
\begin{tabular}{l}
\verb|!$xmp bcast (b) from p(k) on p(:)| \\
\end{tabular}
\end{itemize}
