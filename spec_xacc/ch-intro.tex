\setvruler[][][][3][0][1.2\textwidth]

\chapter{Introduction}
\pagenumbering{arabic}
\setcounter{page}{1}

This document defines the specification of {\XMP}, a directive-based
language extension of {\Fort} and {\C} for scalable and
performance-aware parallel programming.
%
The specification includes a collection of compiler directives and
intrinsic and library procedures, and provides a model of parallel
programming for distributed memory multiprocessor systems.

%This document specifies a collection of compiler directives and runtime
%library routines that can be used to write distributed-memory parallel
%programs in {\C} and {\Fort}.These compiler directives define the
%specifications of the {\XMP} Application 
%Program Interface ({\XMP} API). These specifications provide a
%model of parallel programming for distributed memory multiprocessor
%systems. The directives extend the {\C} and {\Fort} base languages to
%describe distributed memory parallel programs.

\section{Features of {\XMP}}

The features of {\XMP} are summarized as follows:

\begin{itemize}

 \item {\XMP} supports typical parallelization based on the
       data-parallel paradigm and work mapping under ``global-view''
       programming model, and enables parallelizing the original
       sequential code using minimal modification with simple
       directives, like {\OMP} \cite{omp}. Many ideas on ``global-view''
       programming are inherited from High Performance Fortran ({\HPF})
       \cite{hpf}.

 \item The important design principle of {\XMP} is
       ``performance-awareness.'' All actions of communication and
       synchronization are taken by directives (and coarray features),
       which is different from automatic parallelizing compilers. The
       user should be aware of what happens by the {\XMP} directives in
       the execution model on the distributed memory architecture.

 \item {\XMP} also includes features from Partitioned Global Address
       Space (PGAS) languages, such as coarray of the {\Fort} 2008
       standard, for the ``local-view'' programming.

 \item Extension of existing base languages with directives is useful to
       reduce code-rewriting and education costs. The {\XMP} language
       specification is defined on {\Fort} or {\C} as a base
       language.

 \item For flexibility and extensibility, the execution model allows to
       combine with explicit Message Passing Interface ({\MPI})
       \cite{mpi} coding for more complicated and tuned parallel codes
       and libraries.

 \item For multi-core and SMP clusters, {\OMP} directives can be
       combined into {\XMP} for thread programming inside each node as a
       hybrid programming model.

% \item {\bf Language extensions} for familiar languages, such as {\C}
%       and Fortran, which can reduce code-rewriting and educational
%       costs.
%
% \item {\XMP} supports typical parallelization based on the {\bf data
%       parallel paradigm} and work sharing under {\it global view} and
%       enables parallelization of the original sequential code with
%       minimal modification using simple {\bf directives}, such as
%       {\OMP}.
%
% \item {\XMP} also includes a CAF-like Partitioned Global Address Space
%       (PGAS) feature as {\it local-view} programming.
%
% \item {\bf Explicit communication and synchronization}. All actions are
%       taken by directives for being ``easy-to-understand'' for
%       performance-aware programming
%
% \item For flexibility and extensibility, the execution model allows
%       {\bf combination with explicit {\MPI} coding} for more
%       complicated and tuned parallel codes and libraries.
%
% \item For multi-core and SMP clusters, {\bf {\OMP} directives can be
%       combined} into {\XMP} for thread programming inside each node as
%       a hybrid programming model.

\end{itemize}

{\XMP} is being designed based on experiences obtained in the
development of HPF, HPF/JA \cite{hpfja}, Fujitsu XPF (VPP
FORTRAN) \cite{XPF,VPPFORTRAN}, and OpenMPD \cite{OpenMPD}.

\section{Scope}

The {\XMP} specification covers only user-directed parallelization,
wherein the user explicitly specifies the behavior of the compiler and
the runtime system in order to execute the program in parallel in a
distributed-memory system.
%
{\XMP}-compliant implementations are not required to automatically
lay out data, detect parallelism and parallelize loops, or generate
communications and synchronizations.

%The {\XMP} is defined by following items:
%
%\begin{itemize}
%\item A set of directives
%\item Minimum language extension on base languages ({\C} and {\Fort})
%\item Runtime libraries
%\item Environment Variables
%\end{itemize}

\section{Organization of this Document}

The remainder of this document is structured as follows:

\begin{itemize}
 \item Chapter 2: Overview of the {\XMP} Model and Language
 \item Chapter 3: Directives 
 \item Chapter 4: Support for the Local-view Programming
 \item Chapter 5: Base Language Extensions in {\XMPC}
 \item Chapter 6: Procedure Interface
 \item Chapter 7: Intrinsic and Library Procedures
 \item Chapter 8: OpenMP in XcalableMP Programs
\end{itemize}
%
In addition, the following appendices are included in this document as
proposals.
%
\begin{itemize}
 \item Appendix A: Programming Interface for MPI
% \item Appendix B: Directive for Thread Parallelism
 \item Appendix B: Interface to Numerical Libraries
 \item Appendix C: Memory-layout Model
 \item Appendix D: XcalableMP I/O
\end{itemize}
