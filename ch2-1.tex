\chapter{Directives}
\index{directive}

This chapter describes the syntax and behavior of {\XMP} directives.
In this document, the following notation is used to describe {\XMP}
directives. 

\vspace{0.5cm}%

\begin{tabular}{ll}
{\tt xxx} & {\tt type-face} characters are used to indicate literal type characters. \\
{\it xxx...} & If the line is followed by ``...'', then xxx can be
repeated. \\
{\it [xxx]} & {\it xxx} is optional. \\
{\bsquare} & The syntax rule continues. \\
\verb![F]! & The following lines are effective only in {\XMPF}. \\
\verb![C]! & The following lines are effective only in {\XMPC}. \\
\end{tabular}

\section{Directive Format}

\subsection{General Rule}

In {\XMPF}, {\XMP} directives are specified using special comments that
are identified by unique sentinels {\tt\verb|!$xmp|}. An {\XMP}
directive follows the rules for comment lines of either the Fortran free
or fixed source form, depending on the source form of the surrounding
program unit\footnote{Consequently, the rules of comment lines that an
{\XMP} directive follows is the same as the ones that an {\OMP}
directive follows.}. {\XMPF} directives are case-insensitive.

\vspace{0.5cm}

\Syntax{directive}
\begin{tabular}{ll}
\verb![F]! & \verb|!$xmp| {\it directive-name clause} \\
\end{tabular}

\vspace{0.5cm}

In {\XMPC}, {\XMP} directives are specified using the \verb|#pragma|
mechanism provided by the {\C} standards. {\XMPC} directives are
case-sensitive.

\vspace{0.5cm}

\Syntax{directive}
\begin{tabular}{ll}
\verb![C]! & \verb|#pragma xmp| {\it directive-name clause} \\
\end{tabular}

\vspace{0.5cm}

%Additionally, in {\Fort}, directives of the {\it attribute form}
%analogous to type declaration statements in Fortran using the ``{\tt
%::}'' punctuation can also be used.

Directives are classified as declarative directives and executable
directives.

The declarative directive is a directive that may only be
placed in a declarative context. A declarative directive has no
associated executable user code. The scope rule of declarative
directives obeys that of the declaration statements in the base
language.
%
For example, in {\XMPF}, a node array declared by a {\tt nodes}
directive is visible only within either the program unit, the
derived-type declaration or the interface body that immediately
surrounds the directives, unless overridden in the inner blocks or use
or host associated,
%
and, in {\XMPC}, a node array declared by a {\tt nodes} directive is
visible only in the range from the declaring point to the end of 
the block when placed within a block, or of the file when
placed outside any blocks, unless overridden in the inner blocks.
%


The following directives are declarative directives.

\begin{itemize}
 \item {\tt nodes}
 \item {\tt template}
 \item {\tt distribute}
 \item {\tt align}
 \item {\tt shadow}
 \item {\tt coarray}
\end{itemize}

The executable directives are placed in an executable context. A
stand-alone directive is an executable directive that has no associated
user code, such as a {\tt barrier} directive.
%
An executable directive and its associated user code make up an
{\XMP} construct, as in the following format:

\vspace{0.5cm}

\begin{tabular}{ll}
\verb![F]! & \verb|!$xmp| {\it directive-name clause} ...\\
 & \hspace{0.5cm} {\it structured-block} \\
\end{tabular}

\vspace{0.3cm}

\begin{tabular}{ll}
\verb![C]! & \verb|#pragma xmp| {\it directive-name clause} ...\\
 & \hspace{0.5cm} {\it structured-block} \\
\end{tabular}

\vspace{0.5cm}

Note that, in {\XMPF}, a corresponding {\tt end} directive is required
for some executable directives such as {\tt task} and {\tt tasks} and,
in {\XMPC}, the associated statement can be compound.

%\vspace{0.5cm}
%
%\begin{tabular}{ll}
%\verb![F]! & \verb|!$xmp tasks| \\
% & \hspace{0.5cm} {\it structured-block} \\
% & \hspace{0.5cm} ... \\
% & \verb|!$xmp| {\tt end tasks}\\
%\end{tabular}

The following directives are executable directives.

\begin{itemize}
 \item {\tt template\_fix}
 \item {\tt task}
 \item {\tt tasks}
 \item {\tt loop}
 \item {\tt array}
 \item {\tt reflect}
 \item {\tt gmove}
 \item {\tt barrier}
 \item {\tt reduction}
 \item {\tt bcast}
 \item {\tt wait\_async}
\end{itemize}


\subsection{Combined Directive}

\subsubsection*{Synopsis}

For {\XMP} {\Fort}, multiple attributes can be specified
in one combined declarative directive, which is analogous to type
declaration statements in Fortran using the ``{\tt ::}'' punctuation.

\subsubsection*{Syntax}

\begin{center}
\begin{tabular}{llll}
\verb![F]! & \verb|!$xmp| {\it combined-directive} & {\bf is} & {\it
 combined-attribute} {\openb}, {\it combined-attribute}
 {\closeb}... {\tt ::} \\
 & & & {\it combined-decl} {\openb}, {\it combined-decl}
 {\closeb}...
\end{tabular}
\end{center}

{\it combined-attribute} is one of:

\vspace{0.3cm}

\begin{tabular}{ll}
 \hspace{0.5cm} & {\tt nodes} \\
 & {\tt template} \\
 & {\tt distribute} \verb|(|{\it dist-format} {\openb}, {\it
     dist-format}{\closeb}... \verb|)| {\tt onto} {\it nodes-name} \\
 & {\tt align} \verb|(| {\it align-source} {\openb}, {\it
     align-source}{\closeb}... \verb|)| {\bsquare} \\
 & \hspace{4cm}{\bsquare} {\tt with} {\it template-name} \verb|(|{\it
     align-subscript} {\openb}, {\it
     align-subscript}{\closeb}... \verb|)| \\
 & {\tt shadow} \verb|(| {\it shadow-width} {\openb},
     {\it shadow-width}{\closeb}... \verb|)| \\
% & {\tt coarray} {\tt on} {\it nodes-ref} \\
 & {\tt dimension} \verb|(| {\it explicit-shape-spec} {\openb},
     {\it explicit-shape-spec}{\closeb}... \verb|)|
\end{tabular}

\vspace{0.3cm}

and {\it combined-decl} is one of:

\vspace{0.3cm}

\begin{tabular}{ll}
 \hspace{0.5cm} & {\it nodes-decl} \\
 & {\it template-decl} \\
 & {\it array-name}
\end{tabular}

\subsubsection*{Description}

A combined directive is interpreted as if an object corresponding to
each {\it combined-decl} is declared in a directive corresponding to
each {\it combined-attribute}, where all restrictions of each directive,
in addition to the following ones, are applied.

\subsubsection*{Restrictions}

\begin{itemize}
 \item The same kind of {\it combined-attribute} must not appear more
       than once in a given {\it combined-directive}.
 \item If the {\tt nodes} attribute appears in a {\it
       combined-directive}, each {\it combined-decl} must be a {\it
       nodes-decl}.
 \item If the {\tt template} or {\tt distribute} attribute appears in a
       {\it combined-directive}, each {\it combined-decl} must be a {\it
       template-decl}.
 \item If the {\tt align} or {\tt shadow} attribute appears in a
       {\it combined-directive}, each {\it combined-decl} must be an
       {\it array-name}.
 \item If the {\tt dimension} attribute appears in a {\it
       combined-directive}, any object to which it applies must be
       declared with either the {\tt template} or the {\tt nodes}
       attribute.
\end{itemize}
