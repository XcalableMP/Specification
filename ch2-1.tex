\chapter{Directives}\index{directive}

This chapter describes the syntax and behavior of \XMP directives.
In this document, the following notations are used to describe \XMP
directives. 

\vspace{0.5cm}

\begin{tabular}{ll}
{\tt xxx} & {\tt type-face} characters are used for literal. \\
{\it xxx ...} & if the line is followed by ``...'' then xxx can be
repeated. \\
{\it [xxx]} & {\it xxx} is optional. \\
\verb![F]! & The following lines are effective only in \Fort. \\
\verb![C]! & The following lines are effective only in \C. \\
\end{tabular}

\vspace{0.5cm}

In \C, \XMP
directives are specified by using \verb|#pragma| mechanism provided by \C
standards. In \Fort, \XMP directives are specified by using
special comments that are identified by unique sentinels
{\tt\verb|!$xmp|}. The rules of \Fort directives in fixed source format
and free source format follow these in \OMP and HPF.

\vspace{0.5cm}

\Syntax{directive}
\begin{tabular}{ll}
\verb![F]! & \verb|!$xmp| {\it directive-name clause} \\
& \\
\verb![C]! & \verb|#pragma xmp| {\it directive-name clause} \\
\end{tabular}

\vspace{0.5cm}

Directives are classified into declarative directives and
executable directives. The \Directive{declarative directive} is one that my
only be placed in a declarative context. A declarative directive has no
associated executable user code. The scope rule of declarative
directives obeys one of declaration statements in the base
language. For example, node declarations by {\tt node} directives are
effective from declared point to the end of the file in C, and within
subprogram in \Fort.

{\Directive{Executable} directives are placed in executable
context. A stand-alone directive is an executable directive which has
no associated user code, such as a {\tt barrier} directive. Some executable
directives compose directive construct with associated user code, as
in following format:

\vspace{0.5cm}

\begin{tabular}{ll}
\verb![F]! & \verb|!$xmp| {\it directive-name clause} ...\\
 & \hspace{0.5cm} {\it statement} \\
 & \hspace{0.5cm} ... \\
 & \verb|!$xmp| {\tt end} {\it directive-name} \\
\end{tabular}
\vspace{0.3cm}

Note that in {\tt loop} construct, {\tt end} can be omitted.

\vspace{0.5cm}

\begin{tabular}{ll}
\verb![F]! & \verb|!$xmp| {\it directive-name clause} ...\\
 & \hspace{0.5cm} {\it do-loop-construct} \\
\end{tabular}

\vspace{0.5cm}

\begin{tabular}{ll}
\verb![C]! & \verb|#pragma xmp| {\it directive-name clause} ...\\
 & \hspace{0.5cm} {\it statement} \\
\end{tabular}

In C, statement can be a compound-statement.

