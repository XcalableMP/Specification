%\chapter{Procedure call and data mapping for procedure argument}
\chapter{Subprogram Interfaces}

This chapter describes the subprogram interfaces, that is, how
subprograms are invoked and arguments are passed, in {\XMP}.

In order to achieve high composability of {\XMP} programs, it is one of
the most important requirement that {\XMP} subprograms can invoke
subprograms written in the base language with as a few restrictions as
possible. The subprogram interfaces of {\XMP} described in this chapter
is designed so as to satisfy the requirement.


\section{General Rule}

A subprogram invocation itself is a local operation in {\XMP}. Thus, a
node can invoke any subprogram, whether written in {\XMP} or in the base
language, at any point of the execution.
%
There is no restriction on the characteristics of subprograms invoked by
an {\XMP} subprogram, except for a few ones on its argument, which is
explained below.

A local data or part of it (i.e. an array element or a Fortran's array
section) in the actual or dummy argument list (referred to as a {\it
local actual argument} and a {\it local dummy argument}, respectively)
are treated by the {\XMP} compiler in the same way as by the compiler of
the base language.
%
This rule makes it possible that a local actual argument in a subprogram
written in {\XMP} is associated with a dummy argument of a subprogram
written in the base language.

\paragraph{Implementation.}

The {\XMP} compiler does not transform local actual and dummy arguments,
so that the backend compiler of the base language can process them in
the same way as usual data.

\vspace{1.5zw}

\hspace{-1.2\parindent}
The rest of this chapter specifies how global data in the actual and
dummy argument list (referred to as a {\it global actual argument} and a
{\it global dummy argument}, respectively) are processed by the {\XMP}
compiler.


\section{Argument Passing Mechanism in {\XMP} Fortran}

A section (except for an element) of a global data cannot be put in the
actual argument list. Therefore only the name or an element of it can be
put in the actual argument list.

There are two kinds of argument association for global data in {\XMP}
Fortran: one is {\it sequence association}, which is for a global dummy
that is an explicit-shape or assumed-size array, and the other is
{\it descriptor association}, which is for all other global dummy.


\subsection{Sequence Association of Global Data}

If the actual argument is the name of a global data, it represents a
{\it local element sequence} consisting of the elements of its local
section in array element order on each node.
%
Also, if the actual argument is an element of a global data, it
represents an {\it local element sequence} consisting of the
corresponding element of its local section and each element that follows
it in array element order on each node.

An actual argument that represents a local element sequence and
corresponds to a global dummy argument is {\it sequence associated} with
the local section of the dummy argument if the dummy argument is an
explicit-shape or assumed-size array.

The sequence association is the default manner of association for global
actual arguments and therefore are applied unless it is obvious from the
interface of the invoked procedure that the corresponding dummy argument
is neither an explicit-shape nor assumed-size array.

%The rank and shape of the actual argument need not agree with the rank
%and shape of the dummy argument, but the number of elements in the dummy
%argument shall not exceed the number of elements in the element sequence
%of the actual argument. If the dummy argument is assumed-size, the
%number of elements in the dummy argument is exactly the number of
%elements in the element sequence.

\paragraph{Implementation.}

In order to implement the sequence association, the name or an element
of a global data appearing as an actual argument is treated by the
{\XMP} compiler as the base address of its local section on each node,
and the global data appearing as the corresponding dummy argument is
initialized at runtime so as to be composed of the local sections each
of which starts at the received address.
%
Note that on a node that does not own the local section, an unspecified
value (e.g. null) is passed as the base address.


\subsection{Descriptor Association of Global Data}

When the actual argument is a global data and it is obvious from the
interface of the invoked procedure that the corresponding dummy argument
is neither an explicit-shape nor assumed-size array, the actual
argument is {\it descriptor associated} with the dummy argument. That
is, the dummy argument inherits its shape and storage from the actual
argument via a runtime data structure.

\paragraph{Implementation.}

In order to implement the descriptor association, a global actual
argument is treated by the {\XMP} compiler as:

\begin{itemize}
 \item its {\it global-data descriptor}, which is an internal data
       structure managed by the {\XMP} runtime system to hold
       information on a global data, if the dummy is a global data; or
 \item the array representing its local section, which is to be processed
       by the backend Fortran compiler in the same way as usual local
       data, if the dummy is a local data.
\end{itemize}

For the first case, a global dummy is initialized at runtime with a copy
of the global-data descriptor received.


\section{Argument Passing Mechanism in {\XMP} C}

\subsection{Actual Argument}

When the actual argument is a global data, it is passed by the address
of its local section.

\paragraph{Implementation.}

The name of a global data appearing in the actual argument list is
treated by the {\XMP} compiler as the pointer to the first element of
its local section on each node.
%
On a node onto which no part of the global data is mapped, the pointer
is set to an unspecified value (e.g. null).

Note that an element of a global data in the actual argument list is
treated in the same way as those in other usual statements because an
array element is passed by value as in C.

\subsection{Dummy Argument}

When the dummy argument is a global data, an address is received to the
argument as the base address of each its local section.

\paragraph{Implementation.}

The name of a global data appearing in the dummy argument list is
treated by the {\XMP} compiler as the pointer to the first element of
its local section on each node.
%
As a result, the global data is initialized at runtime so as to be
composed of the local sections on the executing node set.