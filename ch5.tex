\chapter{Intrinsic and Library Procedures}
\label{chap:Intrinsic and library procedures}

This specification defines various procedures for system inquiry,
synchronization, computations, etc. The procedures are provided as
intrinsic procedures in {\XMPF} and library procedures in {\XMPC}.

\section{{\tt [F]} Intrinsic Functions}

\subsection{{\tt xmp\_desc\_of}}
\label{subsec: xmp_desc_of}
\Intrinsic{xmp\_desc\_of}
\index{descriptor-of operator}
\index{xmp\_desc\_of@{\tt xmp\_desc\_of}}

\subsubsection*{Format}

\begin{tabular}{lll}

\verb![F]!&  {\tt type(xmp\_desc)}& {\tt xmp\_desc\_of(xmp\_entity)}\\

\end{tabular}

\vspace{0.3cm}

Note that {\tt xmp\_desc\_of} is an intrinsic function in {\XMPF} or
a built-in operator in {\XMPC}. For the {\tt xmp\_desc\_of} operator,
refer to section \ref{sec:Descriptor of Global Data in C}.

\subsubsection*{Synopsis}

{\tt xmp\_desc\_of} returns a descriptor to retrieve informations of the
specified global array, template, or node array. The resulting
descriptor can be used as an input argument of mapping inquiry functions.

The type of the descriptor, {\tt type(xmp\_desc)}, is
implementation-dependent, and defined in a Fortran module named {\tt
xmp\_lib} or a Fortran {\tt include} file named {\tt xmp\_lib.h}.

\subsubsection*{Arguments}

The argument or operand {\tt xmp\_entity} is the name of either a global
array, a template or a node array.

%\subsection{\tt xmp\_desc\_of}
%\label{subsec: xmp_desc_of}
%\Intrinsic{xmp\_desc\_of}
%\index{descriptor-of operator}
%\index{xmp\_desc\_of@{\tt xmp\_desc\_of}}
%
%\subsubsection*{Format}
%
%\begin{tabular}{lll}
%
%\verb![F]!&  {\tt type(xmp\_desc)}& {\tt xmp\_desc\_of(xmp\_entity)}\\
%
%\verb![C]!&  {\tt xmp\_desc\_t}& {\tt xmp\_desc\_of(xmp\_entity)}
%
%\end{tabular}
%
%\vspace{0.3cm}
%
%Note that {\tt xmp\_desc\_of} is an intrinsic function in {\XMPF} or
%a built-in operator in {\XMPC}.
%
%\subsubsection*{Synopsis}
%
%%    A {\tt xmp\_desc\_of} is an input argument of query functions.
%%    When Query functions get descriptor information of array
%%    from XMP compiler, they must set a {\tt xmp\_desc\_of} as an
%%    input argument.
%
%{\tt xmp\_desc\_of} returns, in {\XMPF}, or is evaluated to, in {\XMPC},
%a descriptor to retrieve informations of the specified global array,
%template, or node array. The resulting descriptor can be used as an
%input argument of the inquiry functions which is described in appendix
%\ref{chap:Interface to Numerical Libraries}.
%
%The type of the descriptor, {\tt type(xmp\_desc)}, in {\XMPF} 
%is implementation-dependent, and defined in
%a Fortran module named {\tt xmp\_lib} or a Fortran {\tt include} file
%named {\tt xmp\_lib.h}.
%
%The type of the descriptor, {\tt xmp\_desc\_t}, in {\XMPC} is
%implementation-dependent, and defined in a header file named {\tt xmp.h}
%in {\XMPC}.
%
%%The type of the descriptor, {\tt xmp\_desc\_t}, is
%%implementation-dependent, and defined in a Fortran module named {\tt
%%xmp\_lib} or a Fortran {\tt include} file named {\tt xmp\_lib.h} in
%%{\XMPF}, or a header file named {\tt xmp.h} in {\XMPC}.
%
%\subsubsection*{Arguments}
%
%The argument or operand {\tt xmp\_entity} is the name of either a global
%array, a template or a node array.
%
%%   The argument of {\tt xmp\_desc\_of} is an {\it array-name}, a {\it template-name}, or a {\it nodes-name}.
%%   In the case of {\it array-name}, return value d is a pointer to
%%   be able to access array descriptor information.
%%   In the case of {\it template-name}, return value d is a pointer to
%%   be able to access template descriptor information.
%%   In the case of {\it nodes-name}, return value d is a pointer to
%%   be able to access node descriptor information.
%%   If the argument of {\tt xmp\_desc\_of} is a local data, it is unchanged.


\section{System Inquiry Functions}

\begin{itemize}
% \item {\tt [F] xmp\_desc\_of}
 \item {\tt xmp\_all\_node\_num}
 \item {\tt xmp\_all\_num\_nodes}
 \item {\tt xmp\_node\_num}
 \item {\tt xmp\_num\_nodes}
% \item {\tt xmp\_mpi\_comm}
 \item {\tt xmp\_wtime}
 \item {\tt xmp\_wtick}
\end{itemize}

\subsection{\tt xmp\_all\_node\_num}
\Intrinsic{xmp\_all\_node\_num}

\subsubsection*{Format}

\begin{tabular}{lll}

\verb![F]!&  {\tt integer function}& {\tt xmp\_all\_node\_num()}\\

\verb![C]!&  {\tt int}& {\tt xmp\_all\_node\_num(void)}

\end{tabular}

\subsubsection*{Synopsis}

     The {\tt xmp\_all\_node\_num} routine returns the node number,
     within the primary node set, of the node that calls {\tt
     xmp\_all\_node\_num}.

\subsubsection*{Arguments}

    none.


\subsection{\tt xmp\_all\_num\_nodes}
\Intrinsic{xmp\_all\_num\_nodes}

\subsubsection*{Format}

\begin{tabular}{lll}

\verb![F]!&  {\tt integer function}& {\tt xmp\_all\_num\_nodes()}\\

\verb![C]!&  {\tt int}& {\tt xmp\_all\_num\_nodes(void)}

\end{tabular}

\subsubsection*{Synopsis}

     The {\tt xmp\_all\_num\_nodes} routine returns the number of nodes
     in the entire node set.

\subsubsection*{Arguments}

    none.


\subsection{\tt xmp\_node\_num}
\Intrinsic{xmp\_node\_num}

\subsubsection*{Format}

\begin{tabular}{lll}

\verb![F]!&  {\tt integer function}& {\tt xmp\_node\_num()}\\

\verb![C]!&  {\tt int}& {\tt xmp\_node\_num(void)}

\end{tabular}

\subsubsection*{Synopsis}

     The {\tt xmp\_node\_num} routine returns the node number,
     within the current executing node set, of the node that calls {\tt
     xmp\_node\_num}.

\subsubsection*{Arguments}

none.


\subsection{\tt xmp\_num\_nodes}
\Intrinsic{xmp\_num\_nodes}

\subsubsection*{Format}

\begin{tabular}{lll}

\verb![F]!&  {\tt integer function}& {\tt xmp\_num\_nodes()}\\

\verb![C]!&  {\tt int}& {\tt xmp\_num\_nodes(void)}

\end{tabular}

\subsubsection*{Synopsis}

     The {\tt xmp\_num\_nodes} routine returns the number of the
     executing nodes.

\subsubsection*{Arguments}

none.

%\subsection{\tt xmp\_mpi\_comm}
%
%\subsubsection*{Format}
%
%\begin{tabular}{lll}
%
%\verb![F]!&  {\tt integer function}& {\tt xmp\_mpi\_comm({\it nodes-name})}\\
%
%\verb![C]!&  {\tt int}& {\tt xmp\_mpi\_comm({\it nodes-name})}
%
%\end{tabular}
%
%\subsubsection*{Synopsis}
%     The {\tt xmp\_mpi\_comm} routine returns the integer value with associated communicator
%     to which {\it nodes-name} belongs. If non {\it nodes-name}, {\tt xmp\_mpi\_comm} returns 
%     the value of MPI\_Comm\_World.
%
%\subsubsection*{Arguments}
%    The argument of {\tt xmp\_mpi\_comm} is a {\it nodes-name} of the executing node set.
%
%

\subsection{\tt xmp\_wtime}
\Intrinsic{xmp\_wtime}

\subsubsection*{Format}

\begin{tabular}{lll}

\verb![F]!&  {\tt double precision function}& {\tt xmp\_wtime()}\\

\verb![C]!&  {\tt double}& {\tt xmp\_wtime(void)}

\end{tabular}

\subsubsection*{Synopsis}
    The {\tt xmp\_wtime} routine returns elapsed wall clock time in seconds 
    since some time in the past. The ``time in the past'' is guaranteed
    not to change during the life of the process.
    There is no requirement that different nodes return ``the same time.''

\subsubsection*{Arguments}
    none.


\subsection{\tt xmp\_wtick}
\Intrinsic{xmp\_wtick}

\subsubsection*{Format}

\begin{tabular}{lll}

\verb![F]!&  {\tt double precision function}& {\tt xmp\_wtick()}\\

\verb![C]!&  {\tt double}& {\tt xmp\_wtick(void)}

\end{tabular}

\subsubsection*{Synopsis}
    The {\tt xmp\_wtick} routine returns the resolution of the timer
    used by {\tt xmp\_wtime}. 
    It returns a double precision value equal to the number of seconds 
    between successive clock ticks.

\subsubsection*{Arguments}
    none.


%\subsection{\tt xmp\_barrier}
%
%\subsubsection*{Format}
%
%\begin{tabular}{lll}
%
%\verb![F]!&  & {\tt xmp\_barrier({\it nodes-name})}\\
%
%\verb![C]!&  {\tt void}& {\tt  xmp\_barrier({\it nodes-name})}
%
%\end{tabular}
%
%\subsubsection*{Synopsis}
%    The {\tt xmp\_barrier} routine blocks the caller until all nodes in the executing node set 
%    indicated by {\it nodes-name} have called it.
%    The call returns at any process only after all {\it nodes-name} member's nodes
%    have entered the call.
%
%\subsubsection*{Arguments}
%    The argument of {\tt xmp\_barrier} is a {\it nodes-name} of the executing node set.
%

%\section{Computational Intrinsic Procedures}
%
%\subsection{Fortran}
%
%\begin{itemize}
% \item {\tt x = xmp\_scatter(a, idx1, idx2, ...)}
% \item {\tt x = xmp\_gather(a, idx1, idx2, ...)}
% \item {\tt v = xmp\_pack(a, mask)}
% \item {\tt a = xmp\_unpack(v, mask)}
% \item {\tt x = xmp\_sort\_up(a)}
% \item {\tt x = xmp\_sort\_down(a)}
% \item {\tt x = xmp\_cshift(a, shift, dim)}
% \item {\tt x = xmp\_eoshift(a, shift, b, dim)}
% \item {\tt m = xmp\_transpose(m)}
%\end{itemize}
%
%
%\subsection{C}
%
%
%\begin{itemize}
% \item {\tt x = xmp\_scatter(desc\_a, idx1, idx2, ...)}
% \item {\tt x = xmp\_gather(desc\_a, idx1, idx2, ...)}
% \item {\tt v = xmp\_pack(desc\_a, mask)}
% \item {\tt a = xmp\_unpack(desc\_v, mask)}
% \item {\tt x = xmp\_sort\_up(desc\_a)}
% \item {\tt x = xmp\_sort\_down(desc\_a)}
% \item {\tt x = xmp\_cshift(desc\_a, shift, dim)}
% \item {\tt x = xmp\_eoshift(desc\_a, shift, b, dim)}
% \item {\tt m = xmp\_transpose(desc\_m)}
%\end{itemize}


\section{Synchronization Functions}

\subsection{\tt xmp\_test\_async}
\Intrinsic{xmp\_test\_async}

\begin{tabular}{lll}

\verb![F]!& {\tt logical function} & {\tt xmp\_test\_async(async\_id)}\\
          & {\tt integer} & {\tt async\_id}\\
          & & \\
\verb![C]!&  {\tt int} & {\tt  xmp\_test\_async(int async\_id)}

\end{tabular}

\subsubsection*{Synopsis}

The {\tt xmp\_test\_async} routine returns {\tt .true.}, in {\XMPF}, or
{\tt 1}, in {\XMPC}, if an asynchronous communication specified by the
argument {\tt async\_id} is complete; otherwise, it returns {\tt .false.}
or {\tt 0}.

\subsubsection*{Arguments}

The argument {\tt async\_id} is an integer expression that specifies an
asynchronous communication initiated by a global communication construct
with the {\tt async} clause.


\section{Memory Allocation Functions}

\subsection{\tt [C] xmp\_malloc}
\label{subsec: xmp_malloc}
\Intrinsic{xmp\_malloc}

\begin{tabular}{ll}

{\tt void*} & {\tt xmp\_malloc(xmp\_desc\_t d, size\_t size)}

\end{tabular}

\subsubsection*{Synopsis}

The {\tt xmp\_malloc} routine allocates a storage for the local section
of a one-dimensional global array of size {\tt size} that is associated
with the descriptor specified by {\tt d}, and returns the pointer to it on
each node. For an example of {\tt xmp\_malloc}, refer to section 
\ref{sec:Dynamic Allocation of Global Data in C}.

\subsubsection*{Arguments}

\begin{itemize}
 \item {\tt d} is the descriptor associated with a pointer to the
       one-dimensional global array to be allocated.
 \item {\tt size} is the size of the global array to be allocated.
\end{itemize}


\section{Mapping Inquiry Functions}

All mapping inquiry functions are specified as integer functions.
These functions return zero on success and an implementation-dependent
negative integer value on failure.

\subsection{\tt xmp\_nodes\_ndims}
\index{xmp\_nodes\_ndims@{\tt xmp\_nodes\_ndims}}

\subsubsection*{Format}

\begin{tabular}{lll}

\verb![F]!& {\tt integer function}& {\tt xmp\_nodes\_ndims(d, ndims)}\\
          & {\tt type(xmp\_desc)} & {\tt d}\\
          & {\tt integer} & {\tt ndims}\\

\verb![C]!&  {\tt int}& {\tt xmp\_nodes\_ndims(xmp\_desc\_t d, int *ndims)}\\

\end{tabular}

\subsubsection*{Synopsis}

The {\tt xmp\_nodes\_ndims} function provides the rank of the target node
array.

\subsubsection*{Input Arguments}
\begin{itemize}
 \item {\tt d} is a descriptor of a node array.
\end{itemize}

\subsubsection*{Output Arguments}
\begin{itemize}
 \item {\tt ndims} is the rank of the node array specified by {\tt d}.
\end{itemize}


\subsection{\tt xmp\_nodes\_index}
\index{xmp\_nodes\_index@{\tt xmp\_nodes\_index}}

\subsubsection*{Format}

\begin{tabular}{lll}

\verb![F]!& {\tt integer function}& {\tt xmp\_nodes\_index(d, dim, index)}\\
          & {\tt type(xmp\_desc)} & {\tt d}\\
          & {\tt integer} & {\tt dim}\\
          & {\tt integer} & {\tt index}\\

\verb![C]!&  {\tt int}& {\tt xmp\_nodes\_index(xmp\_desc\_t d, int dim, int *index)}\\

\end{tabular}

\subsubsection*{Synopsis}

The {\tt xmp\_nodes\_index} function provides the indices of the
executing node in the target node array.

\subsubsection*{Input Arguments}

\begin{itemize}
 \item {\tt d} is a descriptor of a node array.
 \item {\tt dim} is the target dimension of the node array.
\end{itemize}

\subsubsection*{Output Arguments}

\begin{itemize}
 \item {\tt index} is an index of the target dimension of the node array
       specified by {\tt d}.
\end{itemize}


\subsection{\tt xmp\_nodes\_size}
\index{xmp\_node\_size@{\tt xmp\_nodes\_size}}

\subsubsection*{Format}

\begin{tabular}{lll}

\verb![F]!& {\tt integer function}& {\tt xmp\_nodes\_size(d, dim, size)}\\
          & {\tt type(xmp\_desc)} & {\tt d}\\
          & {\tt integer} & {\tt dim}\\
          & {\tt integer} & {\tt size}\\

\verb![C]!&  {\tt int}& {\tt xmp\_nodes\_size(xmp\_desc\_t d, int dim, int *size)}\\

\end{tabular}

\subsubsection*{Synopsis}

The {\tt xmp\_nodes\_size} function provides the size of each dimension
of the target node array.

\subsubsection*{Input Arguments}

\begin{itemize}
 \item {\tt d} is a descriptor of a node array.
 \item {\tt dim} is the target dimension of the node array.
\end{itemize}

\subsubsection*{Output Arguments}

\begin{itemize}
 \item {\tt size} is an extent of the target dimension of the node array
       specified by {\\t d}.
\end{itemize}


\subsection{\tt xmp\_nodes\_attr}
\index{xmp\_nodes\_attr@{\tt xmp\_nodes\_attr}}

\subsubsection*{Format}

\begin{tabular}{lll}

\verb![F]!& {\tt integer function}& {\tt xmp\_nodes\_attr(d, attr)}\\
          & {\tt type(xmp\_desc)} & {\tt d}\\
          & {\tt integer} & {\tt attr}\\

\verb![C]!&  {\tt int}& {\tt xmp\_nodes\_attr(xmp\_desc\_t d, int *attr)}\\

\end{tabular}

\subsubsection*{Synopsis}

The {\tt xmp\_nodes\_attr} function provides the attribute of the target
node array. The output value of the argument {\tt attr} is one of:

\begin{tabular}{lll}
  \hspace{2.5cm} & {\tt XMP\_ENTIRE\_NODES} & (Entire nodes)\\
                 & {\tt XMP\_EXECUTING\_NODES}  & (Executing nodes) \\
                 & {\tt XMP\_PRIMARY\_NODES} & (Primary nodes) \\
                 & {\tt XMP\_EQUIVALENCE\_NODES} & (Equivalence nodes) \\
\end{tabular}

These are named constants defined in module {\tt xmp\_lib} 
and in include file {\tt xmp\_lib.h} in {\XMPF}, and symbolic constants
defined in header file {\tt xmp.h} in {\XMPC}.

\subsubsection*{Input Arguments}
\begin{itemize}
 \item {\tt d} is a descriptor of a node array.
\end{itemize}

\subsubsection*{Output Arguments}
\begin{itemize}
 \item {\tt attr} is an attribute of the target node array specified by
       {\tt d}.
\end{itemize}


\subsection{\tt xmp\_nodes\_equiv}
\index{xmp\_nodes\_equiv@{\tt xmp\_nodes\_equiv}}

\subsubsection*{Format}

\begin{tabular}{lll}

\verb![F]!& {\tt integer function}& {\tt xmp\_nodes\_equiv(d, dn, lb,  ub, st)}\\
          & {\tt type(xmp\_desc)} & {\tt d}\\
          & {\tt type(xmp\_desc)} & {\tt dn}\\
          & {\tt integer}         & {\tt lb(*)}\\
          & {\tt integer}         & {\tt ub(*)}\\
          & {\tt integer}         & {\tt st(*)}\\

\verb![C]!&  {\tt int}& {\tt xmp\_nodes\_equiv(xmp\_desc\_t d, xmp\_desc\_t *dn,}\\
          &           & \hspace{3.1cm}{\tt int lb[], int ub[], int st[])}\\

\end{tabular}

\subsubsection*{Synopsis}

The {\tt xmp\_nodes\_equiv} function provides the descriptor of a node
array and a subscript list that represent a node set that is 
assigned to the target node array in the {\tt nodes} directive. This
function returns with failure when the target node array is not declared
as equivalenced.

\subsubsection*{Input Arguments}
\begin{itemize}
 \item {\tt d} is a descriptor of a node array.
\end{itemize}

\subsubsection*{Output Arguments}
\begin{itemize}
 \item {\tt dn} is the descriptor of the referenced node array
       if the target node array is declared as equivalenced; otherwise
       {\tt dn} is set to undefined.
 \item {\tt lb} is a one-dimensional integer array the extent of which
       must be more than or equal to the rank of the referenced node
       array. The i-th element of {\tt lb} is set to the lower bound of
       the i-th subscript of the node reference unless it is ``{\tt *}'',
       or to undefined otherwise.
 \item {\tt ub} is a one-dimensional integer array the extent of which
       must be more than or equal to the rank of the referenced node
       array. The i-th element of {\tt ub} is set to the upper bound of
       the i-th subscript of the node reference unless it is ``{\tt *}'',
       or to undefined otherwise.
 \item {\tt st} is a one-dimensional integer array the extent of which
       must be more than or equal to the rank of the referenced node
       array. The i-th element of {\tt st} is set to the stride of
       the i-th subscript of the node reference unless it is ``{\tt *}'',
       or to zero otherwise.
\end{itemize}


\subsection{\tt xmp\_template\_fixed}
\index{xmp\_template\_fixed@{\tt xmp\_template\_fixed}}

\subsubsection*{Format}

\begin{tabular}{lll}

\verb![F]!& {\tt integer function}& {\tt xmp\_template\_fixed(d, fixed)}\\
          & {\tt type(xmp\_desc)} & {\tt d}\\
          & {\tt logical} & {\tt fixed}\\

\verb![C]!&  {\tt int}& {\tt xmp\_template\_fixed(xmp\_desc\_t d, int *fixed)}\\

\end{tabular}

\subsubsection*{Synopsis}

The {\tt xmp\_template\_fixed} function provides the logical value which
shows whether the template is fixed or not.


\subsubsection*{Input Arguments}
\begin{itemize}
 \item {\tt d} is a descriptor of a template.
\end{itemize}

\subsubsection*{Output Arguments}
\begin{itemize}
 \item {\tt fixed} is set to true in {\XMPF} and an
       implementation-dependent non-zero integer value in {\XMPC} if the
       template specified by {\tt d} is fixed; otherwise to false in
       {\XMPF} and zero in {\XMPC}.
\end{itemize}

\subsection{\tt xmp\_template\_ndims}
\index{xmp\_template\_ndims@{\tt xmp\_template\_ndims}}

\subsubsection*{Format}

\begin{tabular}{lll}

\verb![F]!& {\tt integer function}& {\tt xmp\_template\_ndims(d, ndims)}\\
          & {\tt type(xmp\_desc)} & {\tt d}\\
          & {\tt integer} & {\tt ndims}\\

\verb![C]!&  {\tt int}& {\tt xmp\_template\_ndims(xmp\_desc\_t d, int *ndims)}\\

\end{tabular}

\subsubsection*{Synopsis}

The {\tt xmp\_template\_ndims} function provides the rank of the target
template.


\subsubsection*{Input Arguments}
\begin{itemize}
 \item {\tt d} is a descriptor of a template.
\end{itemize}

\subsubsection*{Output Arguments}
\begin{itemize}
 \item {\tt ndims} is the rank of the template specified by {\tt d}.
\end{itemize}


\subsection{\tt xmp\_template\_lbound}
\index{xmp\_template\_lbound@{\tt xmp\_template\_lbound}}

\subsubsection*{Format}

\begin{tabular}{lll}

\verb![F]!& {\tt integer function}& {\tt xmp\_template\_lbound(d, dim, lbound)}\\
          & {\tt type(xmp\_desc)} & {\tt d}\\
          & {\tt integer} & {\tt dim}\\
          & {\tt integer} & {\tt lbound}\\

\verb![C]!&  {\tt int}& {\tt xmp\_template\_lbound(xmp\_desc\_t d, int dim, int *lbound)}\\

\end{tabular}

\subsubsection*{Synopsis}

The {\tt xmp\_template\_lbound} function provides the lower bound of each
dimension of the template. This function returns with failure when the
lower bound is not fixed.

\subsubsection*{Input Arguments}
\begin{itemize}
 \item {\tt d} is a descriptor of a template.
 \item {\tt dim} is the target dimension of the template.
\end{itemize}

\subsubsection*{Output Arguments}
\begin{itemize}
 \item {\tt lbound} is the lower bound of the target dimension of the
       template specified by {\tt d}.  When the lower bound is not
       fixed, it is set to undefined.
\end{itemize}


\subsection{\tt xmp\_template\_ubound}
\index{xmp\_template\_ubound@{\tt xmp\_template\_ubound}}

\subsubsection*{Format}

\begin{tabular}{lll}

\verb![F]!& {\tt integer function}& {\tt xmp\_template\_ubound(d, dim, ubound)}\\
          & {\tt type(xmp\_desc)} & {\tt d}\\
          & {\tt integer} & {\tt dim}\\
          & {\tt integer} & {\tt ubound}\\

\verb![C]!&  {\tt int}& {\tt xmp\_template\_ubound(xmp\_desc\_t d, int dim, int *ubound)}\\

\end{tabular}

\subsubsection*{Synopsis}

The {\tt xmp\_template\_ubound} function provides the upper bound of each
dimension of the template. This function returns with failure when the
upper bound is not fixed.

\subsubsection*{Input Arguments}
\begin{itemize}
 \item {\tt d} is a descriptor of a template.
 \item {\tt dim} is the target dimension of the template.
\end{itemize}

\subsubsection*{Output Arguments}
\begin{itemize}
 \item {\tt ubound} is a upper bound of the target dimension of the
       template specified by {\tt d}. When the upper bound is not fixed,
       it is set undefined.
\end{itemize}


\subsection{\tt xmp\_dist\_format}
\index{xmp\_dist\_format@{\tt xmp\_dist\_format}}

\subsubsection*{Format}

\begin{tabular}{lll}

\verb![F]!& {\tt integer function}& {\tt xmp\_dist\_format(d, dim, format)}\\
          & {\tt type(xmp\_desc)} & {\tt d}\\
          & {\tt integer} & {\tt dim}\\
          & {\tt integer} & {\tt format}\\

\verb![C]!&  {\tt int}& {\tt xmp\_dist\_format(xmp\_desc\_t d, int dim, int *format)}\\

\end{tabular}

\subsubsection*{Synopsis}

The {\tt xmp\_dist\_format} function provides the distribution format of
a dimension of a template. The output value of the argument {\tt format}
is one of:

\begin{tabular}{lll}
       \hspace{2.5cm} & {\tt XMP\_NOT\_DISTRIBUTED} & (not distributed)\\
                      & {\tt XMP\_BLOCK}  & (block distribution) \\
                      & {\tt XMP\_CYCLIC} & (cyclic distribution) \\
                      & {\tt XMP\_GBLOCK} & (gblock distribution) \\
\end{tabular}

These symbolic constants are defined in ``xmp.h''.

\subsubsection*{Input Arguments}
\begin{itemize}
 \item {\tt d} is a descriptor of a template.
 \item {\tt dim} is the target dimension of the template.
\end{itemize}

\subsubsection*{Output Arguments}
\begin{itemize}
 \item {\tt format} is a distribution format of the target dimension of
       the template specified by {\tt d}.
\end{itemize}


\subsection{\tt xmp\_dist\_blocksize}
\index{xmp\_dist\_blocksize@{\tt xmp\_dist\_blocksize}}

\subsubsection*{Format}

\begin{tabular}{lll}

\verb![F]!& {\tt integer function}& {\tt xmp\_dist\_blocksize(d, dim, blocksize)}\\
          & {\tt type(xmp\_desc)} & {\tt d}\\
          & {\tt integer} & {\tt dim}\\
          & {\tt integer} & {\tt blocksize}\\

\verb![C]!&  {\tt int}& {\tt xmp\_dist\_blocksize(xmp\_desc\_t d, int dim, int *blocksize)}\\

\end{tabular}

\subsubsection*{Synopsis}

The {\tt xmp\_dist\_blocksize} function provides the block width of
a dimension of a template.


\subsubsection*{Input Arguments}
\begin{itemize}
 \item {\tt d} is a descriptor of a template.
        \item {\tt dim} is the target dimension of the template.
\end{itemize}

\subsubsection*{Output Arguments}
\begin{itemize}
 \item {\tt blocksize} is the block width of the target dimension of
       the template specified by {\tt d}.
\end{itemize}


\subsection{\tt xmp\_dist\_gblockmap}
\index{xmp\_dist\_blocksize@{\tt xmp\_dist\_gblockmap}}

\subsubsection*{Format}

\begin{tabular}{lll}

\verb![F]!& {\tt integer function}& {\tt xmp\_dist\_gblockmap(d, dim, map)}\\
          & {\tt type(xmp\_desc)} & {\tt d}\\
          & {\tt integer} & {\tt dim}\\
          & {\tt integer} & {\tt map(N)}\\

\verb![C]!&  {\tt int}& {\tt xmp\_dist\_gblockmap(xmp\_desc\_t d, int dim, int map[])}\\

\end{tabular}

\subsubsection*{Synopsis}

The {\tt xmp\_dist\_gblockmap} function provides the mapping array of the
{\tt gblock} distribution.

When {\tt dim} dimension of the global array is distributed by {\tt
gblock} and its mapping array is fixed, this function returns zero;
otherwise it returns an implementation-dependent negative integer value.

\subsubsection*{Input Arguments}
\begin{itemize}
 \item {\tt d} is a descriptor of a template.
 \item {\tt dim} is the target dimension of the template.
\end{itemize}

\subsubsection*{Output Arguments}
\begin{itemize}
 \item {\tt map} is a one-dimensional integer array the extent of which
       is more than the size of the corresponding
       dimension of the node array onto which the template is
       distributed.

       The i-th element of {\tt map} is set to the value of the i-th
       element of the target mapping array.
\end{itemize}


\subsection{\tt xmp\_dist\_nodes}
\index{xmp\_dist\_nodes@{\tt xmp\_dist\_nodes}}

\subsubsection*{Format}

\begin{tabular}{lll}

\verb![F]!& {\tt integer function}& {\tt xmp\_dist\_nodes(d, dn)}\\
          & {\tt type(xmp\_desc)} & {\tt d}\\
          & {\tt type(xmp\_desc)} & {\tt dn}\\

\verb![C]!&  {\tt int}& {\tt xmp\_dist\_nodes(xmp\_desc\_t d, xmp\_desc\_t *dn)}\\

\end{tabular}

\subsubsection*{Synopsis}

The {\tt xmp\_dist\_nodes} function provides the descriptor of the node
array onto which a template is distributed.


\subsubsection*{Input Arguments}
\begin{itemize}
 \item {\tt d} is a descriptor of a template.
\end{itemize}

\subsubsection*{Output Arguments}
\begin{itemize}
 \item {\tt dn} is the descriptor of the node array.
\end{itemize}


\subsection{\tt xmp\_dist\_axis}
\index{xmp\_dist\_axis@{\tt xmp\_dist\_axis}}

\subsubsection*{Format}

\begin{tabular}{lll}

\verb![F]!& {\tt integer function}& {\tt xmp\_dist\_axis(d, dim, axis)}\\
          & {\tt type(xmp\_desc)} & {\tt d}\\
          & {\tt integer} & {\tt dim}\\
          & {\tt integer} & {\tt axis}\\

\verb![C]!&  {\tt int}& {\tt xmp\_dist\_axis(xmp\_desc\_t d, int dim, int *axis)}\\

\end{tabular}

\subsubsection*{Synopsis}

The {\tt xmp\_dist\_axis} function provides the dimension of the node
array onto which a dimension of a template is distributed. This function
returns with failure when the dimension of the template is not
distributed.

\subsubsection*{Input Arguments}
\begin{itemize}
 \item {\tt d} is a descriptor of a template.
 \item {\tt dim} is the target dimension of the template.
\end{itemize}

\subsubsection*{Output Arguments}
\begin{itemize}
 \item {\tt axis} is a dimension of the node array onto which 
       the target dimension of the template specified by {\tt d} is
       distributed.  When the dimension of the template is not
       distributed, it is set to undefined.
\end{itemize}


\subsection{\tt xmp\_align\_axis}
\index{xmp\_align\_axis@{\tt xmp\_align\_axis}}

\subsubsection*{Format}

\begin{tabular}{lll}

\verb![F]!& {\tt integer function}& {\tt xmp\_align\_axis(d, dim, axis)}\\
          & {\tt type(xmp\_desc)} & {\tt d}\\
          & {\tt integer} & {\tt dim}\\
          & {\tt integer} & {\tt axis}\\

\verb![C]!&  {\tt int}& {\tt xmp\_align\_axis(xmp\_desc\_t d, int dim, int *axis)}\\

\end{tabular}

\subsubsection*{Synopsis}

The {\tt xmp\_align\_axis} function provides the dimension of the
template with which a dimension of a global array is aligned. This
function returns with failure when the dimension of the global array is
not aligned.

\subsubsection*{Input Arguments}
\begin{itemize}
 \item {\tt d} is a descriptor of a global array.
 \item {\tt dim} is the target dimension of the global array.
\end{itemize}

\subsubsection*{Output Arguments}
\begin{itemize}
 \item {\tt axis} is the dimension of the template with which the target
       dimension of the global array specified by {\tt d} is
       aligned. When the dimension of the global array is not aligned,
       or collapsed, it is set to undefined.
\end{itemize}


\subsection{\tt xmp\_align\_offset}
\index{xmp\_align\_offset@{\tt xmp\_align\_offset}}

\subsubsection*{Format}

\begin{tabular}{lll}

\verb![F]!& {\tt integer function}& {\tt xmp\_align\_offset(d, dim, offset)}\\
          & {\tt type(xmp\_desc)} & {\tt d}\\
          & {\tt integer} & {\tt dim}\\
          & {\tt integer} & {\tt offset}\\

\verb![C]!&  {\tt int}& {\tt xmp\_align\_offset(xmp\_desc\_t d, int dim, int *offset)}\\

\end{tabular}

\subsubsection*{Synopsis}

The {\tt xmp\_align\_offset} function provides the align offset for a
dimension of a global array. This function returns with failure when
there is no offset.

\subsubsection*{Input Arguments}
\begin{itemize}
 \item {\tt d} is a descriptor of a global array.
 \item {\tt dim} is the target dimension of the global array.
\end{itemize}

\subsubsection*{Output Arguments}
\begin{itemize}
 \item {\tt offset} is the align offset for the target dimension of the
       global array specified by {\tt d}. When there is no offset, it is
       set to undefined.
\end{itemize}


\subsection{\tt xmp\_align\_replicated}
\index{xmp\_align\_replicated@{\tt xmp\_align\_replicated}}

\subsubsection*{Format}

\begin{tabular}{lll}

\verb![F]!& {\tt integer function}& {\tt xmp\_align\_replicated(d, dim, replicated)}\\
          & {\tt type(xmp\_desc)} & {\tt d}\\
          & {\tt integer} & {\tt dim}\\
          & {\tt logical} & {\tt replicated}\\

\verb![C]!&  {\tt int}& {\tt xmp\_align\_replicated(xmp\_desc\_t d, int dim, \_Bool *replicated)}\\

\end{tabular}

\subsubsection*{Synopsis}

The {\tt xmp\_align\_replicated} function provides the logical value
which shows whether the dimension of the template with which a global
array is aligned is replicated or not. 


\subsubsection*{Input Arguments}
\begin{itemize}
 \item {\tt d} is a descriptor of a global array.
 \item {\tt dim} is the target dimension of the template with which the
       global array is aligned.
\end{itemize}

\subsubsection*{Output Arguments}
\begin{itemize}
 \item {\tt replicated} is a logical scalar, which is set to true if the
       dimension of the template is replicated.

\end{itemize}


\subsection{\tt xmp\_align\_template}
\index{xmp\_align\_template@{\tt xmp\_align\_template}}

\subsubsection*{Format}

\begin{tabular}{lll}

\verb![F]!& {\tt integer function}& {\tt xmp\_align\_template(d, dt)}\\
          & {\tt type(xmp\_desc)} & {\tt d}\\
          & {\tt type(xmp\_desc)} & {\tt dt}\\

\verb![C]!&  {\tt int}& {\tt xmp\_align\_template(xmp\_desc\_t d, xmp\_desc\_t *dn)}\\

\end{tabular}

\subsubsection*{Synopsis}

The {\tt xmp\_align\_template} function provides the descriptor of the
template with which a global array is aligned.


\subsubsection*{Input Arguments}
\begin{itemize}
 \item {\tt d} is a descriptor of a global array.
\end{itemize}

\subsubsection*{Output Arguments}
\begin{itemize}
 \item {\tt dt} is the descriptor of the template.
\end{itemize}


\subsection{\tt xmp\_array\_ndims}
\index{xmp\_array\_ndims@{\tt xmp\_array\_ndims}}

\subsubsection*{Format}

\begin{tabular}{lll}

\verb![F]!& {\tt integer function}& {\tt xmp\_array\_ndims(d, ndims)}\\
          & {\tt type(xmp\_desc)} & {\tt d}\\
          & {\tt integer} & {\tt ndims}\\

\verb![C]!&  {\tt int}& {\tt xmp\_array\_ndims(xmp\_desc\_t d, int *ndims)}\\

\end{tabular}

\subsubsection*{Synopsis}

The {\tt xmp\_array\_ndims} function provides the rank of a global
array.


\subsubsection*{Input Arguments}
\begin{itemize}
 \item {\tt d} is a descriptor of a global array.
\end{itemize}

\subsubsection*{Output Arguments}
\begin{itemize}
 \item {\tt ndims} is the rank of the global array specified by {\tt d}.
\end{itemize}


\subsection{\tt xmp\_array\_lshadow}
\index{xmp\_array\_lshadow@{\tt xmp\_array\_lshadow}}

\subsubsection*{Format}

\begin{tabular}{lll}

\verb![F]!& {\tt integer function}& {\tt xmp\_array\_lshadow(d, dim, lshadow)}\\
          & {\tt type(xmp\_desc)} & {\tt d}\\
          & {\tt integer} & {\tt dim}\\
          & {\tt integer} & {\tt lshadow}\\

\verb![C]!&  {\tt int}& {\tt xmp\_array\_lshadow(xmp\_desc\_t d, int dim, int *lshadow)}\\

\end{tabular}

\subsubsection*{Synopsis}

The {\tt xmp\_array\_lshadow} function provides the size of lower shadow
of a dimension of a global array.


\subsubsection*{Input Arguments}
\begin{itemize}
 \item {\tt d} is a descriptor of a global array.
 \item {\tt dim} is the target dimension of the global array.
\end{itemize}

\subsubsection*{Output Arguments}
\begin{itemize}
 \item {\tt lshadow} is the size of the lower shadow of the target
       dimension of the global array specified by {\tt d}.
\end{itemize}


\subsection{\tt xmp\_array\_ushadow}
\index{xmp\_array\_ushadow@{\tt xmp\_array\_ushadow}}

\subsubsection*{Format}

\begin{tabular}{lll}

\verb![F]!& {\tt integer function}& {\tt xmp\_array\_ushadow(d, dim, ushadow)}\\
          & {\tt type(xmp\_desc)} & {\tt d}\\
          & {\tt integer} & {\tt dim}\\
          & {\tt integer} & {\tt ushadow}\\

\verb![C]!&  {\tt int}& {\tt xmp\_array\_ushadow(xmp\_desc\_t d, int dim, int *ushadow)}\\

\end{tabular}

\subsubsection*{Synopsis}

The {\tt xmp\_array\_ushadow} function provides the size of upper shadow
of a dimension of a global array.


\subsubsection*{Input Arguments}
\begin{itemize}
 \item {\tt d} is a descriptor of a global array.
 \item {\tt dim} is the target dimension of the global array.
\end{itemize}

\subsubsection*{Output Arguments}
\begin{itemize}
 \item {\tt ushadow} is the size of the upper shadow of the target
       dimension of the global array specified by {\tt d}.
\end{itemize}


\subsection{\tt xmp\_array\_lbound}
\index{xmp\_array\_lbound@{\tt xmp\_array\_lbound}}

\subsubsection*{Format}

\begin{tabular}{lll}

\verb![F]!& {\tt integer function}& {\tt xmp\_array\_lbound(d, dim, lbound)}\\
          & {\tt type(xmp\_desc)} & {\tt d}\\
          & {\tt integer} & {\tt dim}\\
          & {\tt integer} & {\tt lbound}\\

\verb![C]!&  {\tt int}& {\tt xmp\_array\_lbound(xmp\_desc\_t d, int dim, int *lbound)}\\

\end{tabular}

\subsubsection*{Synopsis}

The {\tt xmp\_array\_lbound} function provides the lower bound of a
dimension of a global array. This function returns with failure when the
lower bound is not fixed.

\subsubsection*{Input Arguments}
\begin{itemize}
 \item {\tt d} is a descriptor of a global array.
 \item {\tt dim} is the target dimension of the global array.
\end{itemize}

\subsubsection*{Output Arguments}
\begin{itemize}
 \item {\tt lbound} is the lower bound of the target dimension of the
       global array specified by {\tt d}. When the lower bound is not
       fixed, it is set to undefined.
\end{itemize}


\subsection{\tt xmp\_array\_ubound}
\index{xmp\_array\_ubound@{\tt xmp\_array\_ubound}}

\subsubsection*{Format}

\begin{tabular}{lll}

\verb![F]!& {\tt integer function}& {\tt xmp\_array\_ubound(d, dim, ubound)}\\
          & {\tt type(xmp\_desc)} & {\tt d}\\
          & {\tt integer} & {\tt dim}\\
          & {\tt integer} & {\tt ubound}\\

\verb![C]!&  {\tt int}& {\tt xmp\_array\_ubound(xmp\_desc\_t d, int dim, int *ubound)}\\

\end{tabular}

\subsubsection*{Synopsis}

The {\tt xmp\_array\_ubound} function provides the upper bound of a
dimension of a global array. This function returns with failure when the
upper bound is not fixed.

\subsubsection*{Input Arguments}
\begin{itemize}
 \item {\tt d} is a descriptor of a global array.
 \item {\tt dim} is the target dimension of the global array.
\end{itemize}

\subsubsection*{Output Arguments}
\begin{itemize}
 \item {\tt ubound} is the upper bound of the target dimension of the
       global array specified by {\tt d}. When the upper bound is not
       fixed, it is set to undefined.
\end{itemize}

