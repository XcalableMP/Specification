\chapter{Intrinsic and Library Procedures}
\label{chap:Intrinsic and library procedures}

This specification defines various procedures for system inquiry,
synchronization, computations, etc. The procedures are provided as
intrinsic procedures in Fortran and library procedures in C.

\section{Miscellaneous Intrinsic Procedures}

\begin{itemize}
 \item {\tt xmp\_desc\_t d = xmp\_dec\_of(XMP object)}
 \item {\tt xmp\_all\_node\_num}
 \item {\tt xmp\_all\_num\_nodes}
 \item {\tt xmp\_node\_num}
 \item {\tt xmp\_num\_nodes}
 \item {\tt xmp\_mpi\_comm}
 \item {\tt xmp\_wtime}
 \item {\tt xmp\_test\_async}
 \item {\tt xmp\_gtol(xmp\_desc\_t d, int g\_dims[], int l\_dims[])}
\end{itemize}

\vspace{0.3cm}

\subsection{\tt xmp\_desc\_t d = xmp\_dec\_of(XMP object)}

\subsubsection*{Format}

\begin{tabular}{lll}

\verb![C]!&  {\tt xmp\_desc\_t}& {\tt xmp\_desc\_of(array-name} $\vert$ {\tt template-name);}\\

\verb![F]!&  {\tt xmp\_desc\_t}& {\tt xmp\_desc\_of(array-name} $\vert$ {\tt template-name)}
\end{tabular}

             Note that, in C, xmp\_desc\_of is an operator,
             in Fortran, xmp\_desc\_of is a built-in function.

\subsubsection*{Synopsis}

    A xmp\_desc\_of is an input argument of query functions.
    When Query functions get descriptor information of array
    from XMP compiler, they must set a xmp\_desc\_of as an
    input argument.

\subsubsection*{Type}

    Formally, a name of type of xmp\_desc\_of is xmp\_desc\_t.
    Specification of xmp\_desc\_t depends on the XMP compiler implementation.

\subsubsection*{Arguments}

   The argument of xmp\_desc\_of is an array-name or a template-name.
   In the case of array-name, return value d is a pointer to
   be able to access array descriptor information.
   In the case of template-name, return value d is a pointer to
   be able to access template descriptor information.
   If the argument of xmp\_desc\_of is a local data, it is unchanged.

\vspace{0.3cm}

\subsection{\tt xmp\_all\_node\_num}

\subsubsection*{Format}

\begin{tabular}{lll}

\verb![C]!&  {\tt int}& {\tt xmp\_all\_num\_nodes(void);}\\

\verb![F]!&  {\tt integer function}& {\tt xmp\_all\_num\_nodes()}

\end{tabular}

\subsubsection*{Synopsis}
     The {\tt xmp\_all\_node\_num} routine returns the current node number in all available nodes.


\subsubsection*{Arguments}
    none.

\vspace{0.3cm}

\subsection{\tt xmp\_all\_num\_nodes}

\subsubsection*{Format}

\begin{tabular}{lll}

\verb![C]!&  {\tt int}& {\tt xmp\_all\_num\_nodes(void);}\\

\verb![F]!&  {\tt integer function}& {\tt xmp\_all\_num\_nodes()}

\end{tabular}

\subsubsection*{Synopsis}
     The {\tt xmp\_all\_num\_nodes} routine returns the number of all available nodes.

\subsubsection*{Arguments}
    none.


\vspace{0.3cm}

\subsection{\tt xmp\_node\_num}

\subsubsection*{Format}

\begin{tabular}{lll}

\verb![C]!&  {\tt int}& {\tt xmp\_node\_num({\it nodes-name});}\\

\verb![F]!&  {\tt integer function}& {\tt xmp\_all\_num\_nodes({\it nodes-name})}

\end{tabular}

\subsubsection*{Synopsis}
     The {\tt xmp\_node\_num} routine returns the current node number 
     in the executing node set indicated by {\it nodes-name}.

\subsubsection*{Arguments}
     The argument of {\tt xmp\_node\_num} is a {\it nodes-name} of the executing node set.


\vspace{0.3cm}

\subsection{\tt xmp\_num\_nodes}

\subsubsection*{Format}

\begin{tabular}{lll}

\verb![C]!&  {\tt int}& {\tt xmp\_num\_nodes({\it nodes-name});}\\

\verb![F]!&  {\tt integer function}& {\tt xmp\_num\_nodes({\it nodes-name})}

\end{tabular}

\subsubsection*{Synopsis}
     The {\tt xmp\_num\_nodes} routine returns the number of nodes 
     in the executing node set indicated by {\it nodes-name}.

\subsubsection*{Arguments}
    The argument of {\tt xmp\_num\_nodes} is a {\it nodes-name} of the executing node set.


\vspace{0.3cm}

\subsection{\tt xmp\_mpi\_comm}

\subsubsection*{Format}

\begin{tabular}{lll}

\verb![C]!&  {\tt int}& {\tt xmp\_mpi\_comm({\it nodes-name});}\\

\verb![F]!&  {\tt integer function}& {\tt xmp\_mpi\_comm({\it nodes-name})}

\end{tabular}

\subsubsection*{Synopsis}
     The {\tt xmp\_mpi\_comm} routine returns the integer value with associated communicator
     to which nodes-name belongs. If non {\it nodes-name}, {\tt xmp\_mpi\_comm} returns 
     the vlue of MPI\_Comm\_World.

\subsubsection*{Arguments}
    The argument of {\tt xmp\_mpi\_comm} is a{\it nodes-name} of the executing node set.


\vspace{0.3cm}

\subsection{\tt xmp\_wtime}

\subsubsection*{Format}

\begin{tabular}{lll}

\verb![C]!&  {\tt double}& {\tt xmp\_wtime(void);}\\

\verb![F]!&  {\tt double precision function}& {\tt xmp\_wtime()}

\end{tabular}

\subsubsection*{Synopsis}
    The {\tt xmp\_wtime} routine returns elapsed wall clock time in seconds 
    since some time in the past. The "time in the past" is guaranteed not to change
    during the life of the process.
    There is no requirement that different nodes return "the same time".

\subsubsection*{Arguments}
    none.


\vspace{0.3cm}

\subsection{\tt xmp\_wtick}

\subsubsection*{Format}

\begin{tabular}{lll}

\verb![C]!&  {\tt double}& {\tt xmp\_wtick(void);}\\

\verb![F]!&  {\tt double precision function}& {\tt xmp\_wtick()}

\end{tabular}

\subsubsection*{Synopsis}
    The {\tt xmp\_wtick} routine returns the resolution of the timer used by {\tt xmp\_wtime}.
    It returns a double precision value equal to the number of seconds 
    between successive clock ticks.

\subsubsection*{Arguments}
    none.


\vspace{0.3cm}

\subsection{\tt xmp\_barrier}

\subsubsection*{Format}

\begin{tabular}{lll}

\verb![C]!&  {\tt void}& {\tt  xmp\_barrier({\it nodes-name})}\\

\verb![F]!&  & {\tt xmp\_barrier({\it nodes-name})}

\end{tabular}

\subsubsection*{Synopsis}
    The {\tt xmp\_barrier} routine blocks the caller until all nodes in the executing node set 
    indicated by{\it nodes-name} have called it.
    The call returns at any process only after all {\it nodes-name} member's nodes
    have entered the call.

\subsubsection*{Arguments}
    The argument of {\tt xmp\_barrier} is a {\it nodes-name} of the executing node set.

\vspace{0.3cm}

\subsection{\tt xmp\_test\_async}

\begin{tabular}{lll}

\verb![F]!& {\tt logical function} & {\tt xmp\_test\_async(async\_id)}\\
\verb![F]!& {\tt integer} & {\tt async\_id}\\
          & & \\
\verb![C]!&  {\tt int} & {\tt  xmp\_test\_async({\tt async\_id})}

\end{tabular}

\subsubsection*{Synopsis}

The {\tt xmp\_test\_async} routine returns {\tt .true.}, in {\Fort}, or
{\tt 1}, in {\C}, if an asynchronous communication specified by the
argument {\tt async\_id} is complete; otherwise, it returns {\tt .false.}
or {\tt 0}.

\subsubsection*{Arguments}

The argument {\tt async\_id} is an integer expression that specifies an
asynchronous communication initiated by a global communication construct
with the {\tt async} clause or a coarray reference preceded by the {\tt
async} directive.

\vspace{0.3cm}

\subsection{\tt xmp\_gtol}

\begin{tabular}{lll}

\verb![F]!& & {\tt xmp\_gtol(d, g\_dims, l\_dims)} \\
\verb![F]!& {\tt xmp\_desc\_t} & {\tt d}\\
\verb![F]!& {\tt integer} & {\tt g\_dims[NDIMS]}\\
\verb![F]!& {\tt integer} & {\tt l\_dims[NDIMS]}\\
          & & \\
\verb![C]!&  {\tt void} & {\tt xmp\_gtol(xmp\_desc\_t d, int g\_dims[], int l\_dims[])}

\end{tabular}


%\section{Computational Intrinsic Procedures}
%
%\subsection{Fortran}
%
%\begin{itemize}
% \item {\tt x = xmp\_scatter(a, idx1, idx2, ...)}
% \item {\tt x = xmp\_gather(a, idx1, idx2, ...)}
% \item {\tt v = xmp\_pack(a, mask)}
% \item {\tt a = xmp\_unpack(v, mask)}
% \item {\tt x = xmp\_sort\_up(a)}
% \item {\tt x = xmp\_sort\_down(a)}
% \item {\tt x = xmp\_cshift(a, shift, dim)}
% \item {\tt x = xmp\_eoshift(a, shift, b, dim)}
% \item {\tt m = xmp\_transpose(m)}
%\end{itemize}
%
%
%\subsection{C}
%
%
%\begin{itemize}
% \item {\tt x = xmp\_scatter(desc\_a, idx1, idx2, ...)}
% \item {\tt x = xmp\_gather(desc\_a, idx1, idx2, ...)}
% \item {\tt v = xmp\_pack(desc\_a, mask)}
% \item {\tt a = xmp\_unpack(desc\_v, mask)}
% \item {\tt x = xmp\_sort\_up(desc\_a)}
% \item {\tt x = xmp\_sort\_down(desc\_a)}
% \item {\tt x = xmp\_cshift(desc\_a, shift, dim)}
% \item {\tt x = xmp\_eoshift(desc\_a, shift, b, dim)}
% \item {\tt m = xmp\_transpose(desc\_m)}
%\end{itemize}
