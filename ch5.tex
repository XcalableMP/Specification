\chapter{Support for the Local-view Programming}
\label{chap:Support for the Local-view Programming}

This chapter describes the coarray features in XcalableMP, which are based on
that of Fortran 2008. Note that they are also available
in {\XMPC}. In addition, this chapter describes some directives for the local-view programming.

%{\XMP} adopts coarray features for the local-view
%programming. Particularly in {\XMPF}, the features are compatible with
%that of Fortran 2008.
%We refer to the ISO/IEC 1539-1:2010 as the Fortran 2008 standard.

The coarray features in Fortran 2008 are extended and integrated into
XcalableMP. The specifications in this chapter are designed to achieve
the following purposes:

\begin{itemize}
 \item Upward compatibility to the Fortran 2008 coarray features\\
If an {\XMPF} program does not contain any XMP directives,
any standard-conforming Fortran 2008 program
remains standard conforming under XcalableMP.
In this sense, the interpretations and extensions defined in this
 chapter are upward compatible with the Fortran International Standard, ISO/IEC 1539-1:2010 (Fortran 2008).

 \item Support for task parallelism \\
       XcalableMP makes it possible to construct a task parallel program by
       combining multiple Fortran 2008 codes, which may be developed
       independently, with minimum modifications.

 \item Integration of global-view style programming and local-view style programming\\
       In XcalableMP, users can appropriately use global-view style
       programming of 
       XcalableMP or local-view style programming, which is typically
       used in MPI or Fortran 2008 programs, according to
       the characteristics of the code in a program.

 \item Possibility of support for multiple topologies of a computing system\\
       An XcalableMP processor may allow users to specify the correspondence between 
       node arrays and the topologies of a computing system, and to
       exploit the full potential of a particular system.

\end{itemize}


\section{Rules Determining Image Index}
%An image is an instance of an XcalableMP program.
%A node array is a set of nodes, whose individual elements are arranged
%in a rectangular pattern.
%Each node or each element of an instance of a node array
%uniquely corresponds to an image.
This section defines how the image index of an image in a set of images
is determined in association with a node array and a {\tt task} construct.



\subsection{Primary Image Index}
Every image has a default image index in all of the images at the
invocation of a program. 
In XcalableMP, the default image index is the primary image index,
and is an integer value that ranges from one to the number of images
at the invocation of a program. 

%\subsection{Primary node array}
%A primary node array is a node array declared with ``=**'' in a NODES directive.
A primary node array corresponds to all of the images at the invocation of
a program, and it also corresponds to all of the nodes at the invocation of
a program.
The primary image index of an image is the (Fortran) subscript order value of 
the uniquely corresponding element of a primary node array.
%The primary node number of a node is also the (Fortran) subscript order value of 
%the uniquely corresponding element of a primary node array.


\subsection{Image Index Determined by a {\tt task} Directive}

The execution of a {\tt task} directive determines that a set of nodes (and
the corresponding set of images) forms an executing node set.
If a name of a node array or a subobject of a node array 
appears in the {\tt task} directive, the nodes and the corresponding 
images in the executing node set are ordered in (Fortran) array element order in
the node array or the subobject of the node array.
If a name of a template array or a subobject of a template array
appears in the {\tt task} directive, the nodes and the corresponding images 
in the executing node set 
are ordered in (Fortran) array element order in the corresponding 
subobject of the node array.
The image index of an image in the determined set of images 
is the integer order value in the range one to the cardinality of the set of images.
%The node number of a node in the determined set of nodes
%is the integer order value in the range one to the cardinality of the set of nodes.


\subsection{Current Image Index}

%The current set of images is a set of images 
%determined by the most lately executed {\itshape task-directive} in the
%TASK directive constructs that are not completed
%if any TASK directive constructs are being executed.
The image index of an image in the current set of images is
the current image index.

A current executing node array corresponds to the 
current set of images and also the current executing node set 
at the evaluation of the declaration of the node array.
Each image in the current set of images corresponds 
to the element of an executing node array whose
subscript order value is the same as the current image index
of the image when the evaluation of the declaration of the executing
node array is being evaluated.
In particular, when all {\tt task} directive constructs are 
completed, the current image index of an image is the same as the
primary image index.

%The current set of images corresponds to 
%primary node arrays and all the nodes at the invocation of a program
%if there are no TASK directive constructs that are not completed.


\subsection{Image Index Determined by a Non-primary Node Array}
%\subsection{Non-primary node array}

%A non-primary node array is a node array declared without ``={\it node-ref}'', ``=**'', or ``=*''
%in a NODES directive.

A non-primary node array corresponds to all of the images at the
invocation of a program, and it also corresponds to all the nodes at the
invocation of 
a program. The correspondence between each image and 
each element of a non-primary node array is processor dependent.
A processor may support any means to specify the correspondence.

The image index of an image in all of the images at the invocation 
of a program is the subscript order value of the corresponding element
of a non-primary node array. This is the case if and only if the current
set of images 
corresponds to the non-primary node whole array in which
the nodes in the executing node set are ordered in (Fortran)
array-element order in the non-primary node whole array.
The image index is a non-primary image index. 

The correspondence between the primary image index and a non-primary image
index of the same image is processor dependent.
Between any two distinct non-primary node arrays, 
the correspondence between a non-primary image index 
and the other non-primary image index of the same image is
processor dependent unless they have the same shape. 
If two non-primary node arrays have the same shape,
the corresponding elements of the node arrays correspond
to the same image.


%\subsection{Executing node array}
%An executing node array is a node array declared with ``=*''
%in a NODES directive.
%A current executing node array corresponds to the executing node set
%and also the current set of images
%at the evaluation of the declaration of the node array.
%
%Each image in the current set of images corresponds 
%to the element of an executing node array whose
%subscript order value is the same as the current image index
%of the image at the evaluation of the declaration of the executing node array.


\subsection{Image Index Determined by an Equivalenced Node Array}

A {\tt nodes} directive with ``={\it node-ref}'' that is not ``=*'' or
``=**''
specifies that
each element of the declared node array corresponds in (Fortran)
array-element order
to that of the {\it node-ref},
which is the name of a node array or a subobject of a node array.
The nodes in the declared node array and the corresponding images 
are ordered in (Fortran) array-element order in the {\it node-ref}.
The image index of an image in the set of images
corresponding to the declared node array
is the integer order value ranging from one to the cardinality of the
set of images.
%The node number of a node in the set of nodes corresponding to the declared node array
%is the integer order value in the range one to the cardinality of the set of nodes.


\subsection{On-node Image Index}

XcalableMP supports the {\tt coarray} directive and the {\tt image}
directive to   %%%%%% HARD TO READ. The reviewer misunderstood this sentence.
specify that an image index indicates the image corresponding to
the element of a particular node array whose subscript order value is 
the same as the image index.
The image index is an on-node image index for the specified node array. 
Because an evaluation of the declaration of a node array 
determines a set of images corresponding to the node array,
the directives specify that the set of images 
is the ``all images'' for the image indices affected by the directives.
In particular, the on-node image index for a primary node array
is the primary image index. 


\section{Basic Concepts}


In XcalableMP, ``all images'' in Fortran 2008 changes coupled with the 
execution of {\tt task} constructs, and
refers to the current set of images.
In particular,
when an {\tt allocate} statement is executed for which an {\it
allocate-object} is a coarray, there is an implicit synchronization of 
all the images in the current set of images.
On each image in the current set of images, execution of the segment 
following the statement is delayed until
all other images in the set have executed the same
statement the same number of times.
When a {\tt deallocate} statement is executed for which an {\it
allocate-object} is a coarray, there is an implicit synchronization of
all the images in the current set of images.
On each image in the current set of images, execution of the segment following the
statement is delayed until all other images in the set
have executed the same statement the same number of times.



%\subsection{A restriction on allocatable coarrays}

\begin{itemize}
 \item 
When an allocatable coarray is allocated during the execution of
       {\tt task} constructs,
       the coarray shall be subsequently deallocated before the completion of 
       the {\tt task} construct whose
       {\tt task} directive is the most recently executed one 
       in the {\tt task} constructs that are not completed at the allocation.

\end{itemize}


The image index determined by an image selector 
indicates the current image index by default.
Coarrays are visible within the range of the ``all images,'' and are 
accessed using the current image index by default.
The image index that appears in an executable statement
indicates the current image index by default.


\subsection{Examples}

\begin{itemize}
 
 \item In the following code fragment, the value of a coarray {\ttfamily b} on 
the images 1, 2, 3, and 4, which constitute the executing node set
and correspond to {\ttfamily node(5)}, {\ttfamily node(6)}, {\ttfamily node(7)}, and {\ttfamily node(8)} respectively,
is defined with the value of the coarray {\ttfamily a} on {\ttfamily node(5)}.

\begin{XFexample}
      program xmpcoarray
!$xmp nodes node(8)=**   ! A primary node array.
!$xmp task on node(5:8)  ! The executing node set
        call sub         ! corresponds to node(5:8).
!$xmp end task
      end

      subroutine sub           
      real, save :: a[*], b[*] ! The images 1, 2, 3,
         :                     ! and 4 correspond to node(5:8),
      b = a[1]                 ! respectively.
\end{XFexample}

 \item In the following code fragment, an allocatable coarray {\ttfamily a}
is allocated on the images 1, 2, 3, and 4, 
which constitute the executing node set
and correspond to {\ttfamily node(5)}, {\ttfamily node(6)}, {\ttfamily node(7)}, and {\ttfamily node(8)}, respectively.

\begin{XFexample}
      program xmpcoarray
!$xmp nodes node(8)=**
!$xmp task on node(5:8)  ! The executing node set
        call sub2        ! corresponds to node(5:8).
!$xmp end task
      end

      subroutine sub2           
      real, allocatable :: a(:)[:]
        :                         
      allocate(a(0:99)[*])        
\end{XFexample}
\end{itemize}



\paragraph*{Note}
\begin{itemize}
 \item 
       The result of {\tt xmp\_num\_nodes()}
       is always the same as that of {\ttfamily NUM\_IMAGES()}.

 \item 
  The result of {\tt xmp\_node\_num()} is always the same as that of {\tt THIS\_IMAGE()}.
 \item 
       In a {\tt read} statement, an io-unit that is an asterisk
       identifies an external unit that is preconnected for a sequential
       formatted input only on the image whose primary image index is one.

\end{itemize}



\section{{\tt coarray} Directive}
\subsection{Purpose and Form of the {\tt coarray} Directive}

The {\tt coarray} directive maps coarrays onto a node array and
the set of images that corresponds to the node array.
An image index determined by an image selector for a coarray
that appears in a {\tt coarray} directive
always indicates the on-node image index for the node array;
that is, the specified image corresponds to the node whose subscript
order value in the node array is the same as the image index.

A coarray appearing in a {\tt coarray} directive is an on-node coarray
of the node array that is specified in the {\tt coarray} directive.


\medskip


\begin{tabular}{llll}
 & & & \\
\verb![F]!&\verb|!$xmp| {\tt coarray on} {\it node-name} {\tt ::} {\it object-name-list}\\
 & & & \\
\verb![C]!&\verb|#pragma xmp| {\tt coarray on} {\it node-name} {\tt ::} {\it object-name-list}\\
 & & & 
\end{tabular}


\begin{itemize}
 \item An {\it object-name} shall be a name of a coarray declared 
       in the same scoping unit.

 \item The same {\it object-name} shall not appear more than once
       in {\tt coarray} directives in a scoping unit.

 \item If an {\it object-name} is a name of an allocatable object,
       the current set of images
       at the allocation and the deallocation of the
       object shall correspond to the node array specified as the {\it
       node-name}, and the current image index of each image shall 
       be the same as the subscript order value of the corresponding
       element of the node array.

 \item If an {\it object-name} is the name of an allocated allocatable
       dummy argument,
       the set of images onto which it is mapped
       shall be a subset of the set of images that has most recently
       allocated 
       the corresponding argument in the chain of argument associations.
       
 \item If an {\it object-name} is the name of a nonallocatable dummy argument
       whose ultimate argument has an allocatable attribute,
       the set of images onto which the {\it object-name} is mapped
       shall be a subset of the set of images that has most recently
       allocated 
       the corresponding argument in the chain of argument associations.

 \item The image index determined by an image selector for an on-node
       coarray shall be within the range of one to the size of the node
       array onto which the on-node coarray is mapped.

 \item 
THIS\_IMAGE(COARRAY[,DIM]) shall be invoked by
the image contained in the set of images onto which
the COARRAY argument is mapped
if the COARRAY argument appears in a {\tt coarray} directive.


\end{itemize}

\paragraph*{Note}
\begin{itemize}
 \item 
       The result value of THIS\_IMAGE(COARRAY)
       is the sequence of cosubscript
       values for the COARRAY argument that would specify the current image index of
       the invoking image, if the COARRAY argument does not appear in a
       {\tt coarray} directive.
       The result value of THIS\_IMAGE(COARRAY)
       is the sequence of cosubscript
       values for the COARRAY argument that would specify the on-node image index of the invoking image
for the node array onto which the COARRAY argument is mapped
       if the COARRAY argument appears in a {\tt coarray} directive.	      

 \item 

       The result value of THIS\_IMAGE(COARRAY,DIM) is
       the value
       of cosubscript DIM in the sequence of cosubscript values
       for the COARRAY argument that would specify the current image index of
       the invoking image if the COARRAY argument does not appear in a
       {\tt coarray} directive.
       The result value of THIS\_IMAGE(COARRAY,DIM) is
       the value
       of cosubscript DIM in the sequence of cosubscript values
       for the COARRAY argument that would specify the on-node image index of
       the invoking image for the node array
       onto which the COARRAY argument is mapped
       if the COARRAY argument appears in a {\tt coarray}
       directive.
\end{itemize}


\subsection{An Example of the {\tt coarray} Directive}


\begin{XFexample}
      module global
!$xmp nodes node(8)=**    
      real s[*]             ! The coarray s is always 
!$xmp coarray on node :: s  ! visible on node(1:8).
      end global

      program coarray
      use global
!$xmp task on node(5:8)     ! The executing node set 
        call sub            ! consists of node(5:8).
!$xmp end task
      end

      subroutine sub
      use global
      real, save :: a[*]    ! The images 1, 2, 3, and 4 
         :                  ! correspond to node(5:8), respectively.
      if(this_image().eq.1)then ! The value of the coarray a on node(5) 
         s[1] = a               ! defines that of the coarray s on node(1)
      endif
\end{XFexample}


\section{{\tt image} Directive}
\subsection{Purpose and Form of the {\tt image} Directive}

The {\tt image} directive specifies that 
an image index in the following executable statement
indicates the on-node image index
of the node array specified in the {\tt image} directive
unless the image index is determined by an image selector.


The {\tt image} directive also specifies that 
the execution of a {\tt sync all} statement performs a synchronization
of 
all of the images corresponding to the node array specified in the {\tt
image} directive.


\medskip


\begin{tabular}{llll}
 & & & \\
\verb![F]! & \verb|!$xmp| {\tt image} {\tt (} {\it node-name} {\tt )}\\
 & & & \\
\verb![C]! & \verb|#pragma xmp| {\tt image} {\tt (} {\it node-name} {\tt )}\\
 & & & 
\end{tabular}


\begin{itemize}
 \item An {\tt image} directive shall be followed by a sync all statement,
       an image control statement that contains {\it image-set}, or a reference to
       an intrinsic procedure that has {\tt IMAGES} argument.
\end{itemize}

\subsection{An Example of the {\tt image} Directive}

\begin{XFexample}
      module global
!$xmp nodes node(8)=**
      real s[*]             ! The coarray s is always visible 
!$xmp coarray on node :: s  ! on node(1:8).
      end global

      program image
      use global
!$xmp tasks
!$xmp task on node(1:4) 
        call subA  ! The executing node set consists of node(1:4).
!$xmp end task
!$xmp task on node(5:8) 
        call subB  ! The executing node set consists of node(5:8).
!$xmp end task
!$xmp end tasks
      end

      subroutine subA      
      use global
      real, save :: a[*] ! The images 1, 2, 3, and 4 
         :               ! correspond to node(1:4), respectively.
!$xmp image(node)        ! Synchronization between node(1:4) and
        sync images(5)   ! node(5).
      a = s[1]           ! a on node(1:4) is defined using
         :               ! the value of s on node(1).
      end subroutine	

      subroutine subB 
      use global
      real, save :: b[*] ! The images 1, 2, 3, and 4 
         :               ! correspond to node(5:8), respectively.
      if(this_image() .eq. 1)then ! The image 1 indicates node(5).
        s[1] = b      ! s on node(1) is defined using the value of 
                      ! b on node(5).
!$xmp   image(node)                ! Synchronization between 
          sync images((/1,2,3,4/)) ! node(5) and node(1:4).
      endif
         :
      end subroutine		
\end{XFexample}


\section{Image Index Translation Intrinsic Procedures}

XcalableMP supports intrinsic procedures to translate
image indices between different sets of images.

\subsection{Translation to the Primary Image Index}

{\bf xmp\_get\_primary\_image\_index(NUMBER,INDEX,PRI\_INDEX,NODE\_DESC)}

\begin{description}
 \item[Description.] Translate image indices to the primary image indices.

 \item[Class.] Subroutine.
	    
 \item[Arguments.] \hspace{1cm}
	    \begin{description}
	     \item[NUMBER] shall be a scalar of type default integer. It
			is an INTENT(IN) argument.
	     \item[INDEX] shall be a rank-one array of type default integer.
			The size of {\bfseries INDEX} shall be
			greater than or equal to the value of {\bfseries NUMBER}.
			It is an INTENT(IN) argument.
			The value of each element of {\bfseries INDEX} shall be
					within 
			the range one
			to the size of the node array specified 
			in {\bfseries NODE\_DESC} if {\bfseries NODE\_DESC} appears.
			The value of each element of {\bfseries INDEX} shall be
					within 
			the range one
			to the cardinality of the current set of images
			if {\bfseries NODE\_DESC} does not appear.
	     \item[PRI\_INDEX] shall be a rank-one array of type default integer.
			The size of {\bfseries PRI\_INDEX} shall be
			greater than or equal to the value of {\bfseries NUMBER}.
			It is an INTENT(OUT) argument.
			If {\bfseries NODE\_DESC} appears,
			{\bfseries PRI\_INDEX(i)} is assigned the primary image index
			corresponding to the element of the node array
			specified in {\bfseries NODE\_DESC}
			whose subscript order value is {\bfseries INDEX(i)}; otherwise,
			{\bfseries PRI\_INDEX(i)} is assigned the primary image index
			corresponding to the image whose
			current image index is {\bfseries INDEX(i)}.
	     \item[NODE\_DESC (optional)] shall be a descriptor of a
			node array. It is an INTENT(IN) argument.
			{\bfseries NODE\_DESC} shall appear in XcalableMP C.
	    \end{description}

 \item[Example.] 
In the following code fragment, the value of {\ttfamily index(1:4)} is {\ttfamily (/5,6,7,8/)}.
\begin{XFexample}
!$xmp nodes node(1:8)=**        ! A primary node array
!$xmp nodes subnode(4)=node(5:8)
      integer index(4)
      call xmp_get_primary_image_index&
           &(4,(/1,2,3,4/),index,xmp_desc_of(subnode))
\end{XFexample}
\end{description}


\subsection{Translation to the Current Image Index}
%{\bf XMP\_GET\_IMAGE\_INDEX(CURINDEX,INDEX,NUM,NODE\_DESC)}

{\bf xmp\_get\_image\_index(NUMBER,INDEX,CUR\_INDEX,NODE\_DESC)}

\begin{description}
 \item[Description.] Translate image indices to the current image indices.

 \item[Class.] Subroutine.
	    
 \item[Arguments.] \hspace{1cm}
	    \begin{description}
	     \item[NUMBER] shall be a scalar of type default integer. It is an INTENT(IN) argument.
	     \item[INDEX] shall be a rank-one array of type default integer.
			The size of {\bfseries INDEX} shall be
			greater than or equal to the value of {\bfseries NUMBER}.
			It is an INTENT(IN) argument.
			The value of each element of {\bfseries INDEX} shall be
					within 
			the range one
			to the size of the node array specified 
			in {\bfseries NODE\_DESC}.
	     \item[CUR\_INDEX] shall be a rank-one array of type default integer.
			The size of {\bfseries CUR\_INDEX} shall be
			greater than or equal to the value of {\bfseries NUMBER}.
			It is an INTENT(OUT) argument.
			If the current image index 
			corresponding to the element of the node-array
			specified in {\bfseries NODE\_DESC}
			whose subscript order value is {\bfseries INDEX(i)}
			exists,
			{\bfseries CUR\_INDEX(i)} is assigned the current image index;
			otherwise, {\bfseries CUR\_INDEX(i)} is assigned zero.
	     \item[NODE\_DESC] shall be a descriptor of a
			node array. It is an INTENT(IN) argument.
	    \end{description}

 \item[Example.] 
In the following code fragment, the value of {\ttfamily index(1:4)} is {\ttfamily (/1,2,3,4/)}.
\begin{XFexample}
!$xmp nodes node(1:8)=**        
      integer index(4)
!$xmp task on node(5:8)
        call xmp_get_image_index&
             &(4,(/5,6,7,8/),index,xmp_desc_of(node))
!$xmp end task
\end{XFexample}
\end{description}


\section{Examples of Communication between Tasks}

\begin{itemize}
 \item In the following program fragment, two tasks communicate
with each other with synchronization.
\begin{XFexample}
      module nodes
!$xmp nodes node(8)=**            ! A primary node array
      integer, parameter :: n=2
!$xmp nodes subnodeA(n)=node(1:n)      ! subnodeA is for taskA.
!$xmp nodes subnodeB(8-n)=node(n+1:8)  ! subnodeB is for taskB.
      endmodule

      module intertask
      use nodes
      real,save :: dA[*],dB[*]     
      endmodule

      use nodes
!$xmp tasks  
!$xmp task on subnodeA  ! The taskA is invoked on subnodeA.
        call taskA
!$xmp end task
!$xmp task on subnodeB  ! The taskB is invoked on subnodeB.
        call taskB
!$xmp end task
!$xmp end tasks
      end

      subroutine taskA
      use intertask 
         :
      me = this_image()    ! The value of me is i on subnodeA(i).
      if(me.eq.1)then
        call xmp_get_primary_image_index&  ! The value of iyouabs
               &(1,(/1/),iyouabs,subnodeB) ! is n+1.
!$xmp   image(node)                    ! Synchronization between
          sync images(iyouabs)         ! node(1) and node(n+1).
        call exchange(dA,dB,iyouabs)   
!$xmp   image(node)                    ! Synchronization between
          sync images(iyouabs)         ! node(1) and node(n+1).
      endif
      sync all                ! Synchronization within subnodeA.
      if(me.ne.1)dA = dA[1]            
      sync all                ! Synchronization within subnodeA.
       :
      end

      subroutine taskB
      use intertask
         :
      me = this_image()    ! The value of me is i on subnodeB(i).
      if(me.eq.1)then
        call xmp_get_primary_image_index&  ! The value of iyouabs
             &(1,(/1/),iyouabs,subnodeA)   ! is 1.
!$xmp   image(node)                    ! Synchronization between
          sync images(iyouabs)         ! node(n+1) and node(1).
        call exchange(dB,dA,iyouabs)   
!$xmp   image(node)                    ! Synchronization between
          sync images(iyouabs)         ! node(n+1) and node(1).
      endif
      sync all                ! Synchronization within subnodeB.
      if(me.ne.1)dB = dB[1]  
      sync all                ! Synchronization within subnodeB.

      end

      subroutine exchange(mine,yours,iput)
      use nodes
      real :: mine[*],yours[*]       ! mine and yours are always 
!$xmp coarray on node :: mine,yours  ! visible on node(1:8).

      yours[iput] = mine  ! node(1) puts mine to yours[n+1] and 
                          ! node(n+1) puts mine to yours[1].
      end
\end{XFexample}


 \item In the following program fragment, two tasks communicate
with each other without one-to-one synchronization.

\begin{XFexample}
!$xmp nodes node(8)=**       ! A primary node array
        :
!$xmp tasks                        
!$xmp   task on(node(1:n))
          call taskA(n)      ! The taskA is invoked on node(1:n)
!$xmp   end task
!$xmp   task on(node(n+1:8))
          call taskB(8-n)    ! The taskB is invoked on node(n+1:8)
!$xmp   end task
!$xmp end tasks
      end

      subroutine taskA(n)
      real,save :: yours[*],mine[*]
!$xmp nodes subnode(n)=*        ! An executing node array

      me = this_image()
      if(me.eq. NUM_IMAGES())then
         call xmp_get_primary_image_index(1,me,meabs) ! meabs=n.
         call exchange(yours,mine,meabs,meabs+1,NUM_IMAGES())
      endif
      sync all                  ! Synchronization within node(1:n).
      if(me.ne.NUM_IMAGES())mine = mine[NUM_IMAGES()] 
      sync all                  ! Synchronization within node(1:n).
      end

      subroutine taskB(m)
      real,save :: yours[*],mine[*]
!$xmp nodes subnode(m)=*        ! An executing node array

      me = this_image()
      if(me.eq.1)then
         call xmp_get_abs_image_index(1,me,meabs) ! meabs=n+1.
         call exchange(yours,mine,meabs,meabs-1,NUM_IMAGES())
      endif
      sync all                  ! Synchronization within node(n+1:8).
      if(me.ne.1)mine = mine[1]       
      sync all                  ! Synchronization within node(n+1:8).
      end

      subroutine exchange(yours,mine,meabs,iyouabs,nnodes)
      USE, INTRINSIC :: ISO_FORTRAN_ENV
      real :: yours[*],mine[*]
      real, save :: s[*]                       ! only for exchage.
      TYPE(LOCK_TYPE),save :: lock[*]          ! for lock.
!$xmp nodes subnode(nnodes)=*     ! An executing node array.
!$xmp nodes node(8)=**            ! The coarrays s and lock are 
!$xmp coarray on node :: s,lock   ! always visible on node(1:8).

      LOCK(lock[meabs])   ! node(n) puts yours[n] to s[n] and 
      s[meabs] = yours    ! node(n+1) puts yours[n+1] to s[n+1].
      UNLOCK(lock[meabs])

      LOCK(lock[iyouabs])  ! node(n) gets s[n+1] into mine[n] and 
      mine = s[iyouabs]    ! node(n+1) gets s[n] into mine[n+1].
      UNLOCK(lock[iyouabs])
      end
\end{XFexample}

\end{itemize}

%\section{Coarrays in XcalableMP}
%
%The specification of the coarray features in XcalableMP conforms to that
%of Fortran 2008 unless otherwise provided.
%%
%Each node in the entire node set is considered to correspond to each
%image of a program. Therefore the number of images is always equal to
%the size of the entire node set. The image index of a node is its node
%number of the entire node set.
%
%Declaring coarrays on an arbitrary node array, which may not correspond
%to the entire node set, is an open issue of {\XMP} and under
%discussion. Note that even if the coarray features of {\XMP} would be
%extended in the future version so that they could be declared on a
%subset of the entire node set, they are compatible with those of Fortran
%2008 as long as declared on the entire node set.
%
%Terms related to coarrays in {\XMP} (e.g. \Term{coshape}, coindex,
%cobound, \Term{cosubscript}, image, image index, etc.) are derived from
%that in Fortran 2008.

\section{{\tt [C]} Coarrays in {\XMPC}}
\label{154737_16Jan17}

%Described in the rest of this section is the coarray features for
%{\XMPC}.
This section describes the coarray features for {\XMPC}.

\subsection{{\tt [C]} Declaration of Coarrays}
\label{173311_31Oct14}

\subsubsection*{Synopsis}

Coarrays are declared in {\XMPC}.

\subsubsection*{Syntax}
\Syntax{coarray}

% \begin{tabular}{ll}
% \verb![C]! & {\it data-type} {\it variable} {\openb}{\tt
%      ,} {\it variable} {\closeb}{\tt ...} {\tt :} 
% {\it codimensions} \\
% \end{tabular}

\begin{tabular}{ll}
\verb![C]! & \mytextcolor{red}{{\it data-type} {\it variable} {\tt :} {\it codimensions}} \\
\end{tabular}

\vspace{0.5cm}

where {\it codimensions} is:

\vspace{0.3cm}
\hspace{0.5cm} {\openb}{\tt [}{\it int-expr}{\tt ]}...{\closeb}{\tt [*]}

\subsubsection*{Description}
For {\XMPC}, 
coarrays are declared with a colon and square bracket, where {\it
codimensions} specify the coshape of a variable.

Note that the {\tt coarray} directive for defining a coarray in the
XcalableMP specification 1.0 (page 49) is obsolete.

\subsubsection*{Restrictions}

\begin{itemize}
 \item A coarray {\it variable} must have a global scope.
\end{itemize}

\subsubsection*{Examples}
\Example{coarray}

\begin{XCexample}
#pragma xmp nodes p[16]
float x:[*];
\end{XCexample}

A variable {\it x} that has a global scope is declared as a coarray.

\subsection{{\tt [C]} Reference of Coarrays}

\subsubsection*{Synopsis}
A coarray can be directly referenced or defined by any node. 
The target node is specified using an extended notation in {\XMPC}.

\subsubsection*{Syntax}
\index{coarray reference}
\index{Syntax!coarray reference}

\begin{tabular}{ll}
\verb![C]! & {\it variable} {\tt :} {\tt [}{\it int-expr}{\tt ]}...\\
\end{tabular}

\subsubsection*{Description}

A sequence of {\tt [}{\it int-expr}{\tt ]}'s preceded by a colon in
{\XMPC} determines the image index for a coarray to be accessed.
Note that the image index in {\XMPC} is 0-origin while the image index in {\XMPF} is 1-origin.

A reference of coarrays can appear in the same place as an reference of
normal variables in the base languages.

\subsubsection*{Examples}
\Example{coarray}

In the following codes, 
the second image ([C] image index 1/[F] image index 2) gets all values of array {\tt B} 
on the first image ([C] image index 0/[F] image index 1) to array {\tt A} on the second image.
\vspace{0.3cm}

\index{Example!coarray reference}
\begin{minipage}{0.45\hsize}
\begin{center}
\begin{XCexample}
int A[100]:[*], B[100]:[*];

if(xmpc_this_image() == 1){
  A[:] = B[:]:[0];
}
\end{XCexample}
\end{center}
\end{minipage}
~
\begin{minipage}{0.45\hsize}
\begin{center}
\begin{XFexampleR}
integer :: A(100)[*], B(100)[*]

if (this_image() == 2) then
  A(:) = B(:)[1]
end if
\end{XFexampleR}
\end{center}
\end{minipage}

%\subsection{Declaration of Coarrays}
%
%\subsubsection*{Synopsis}
%
%Coarrays are declared in the same manner as in Fortran 2008 for {\XMPF},
%or with the {\tt coarray} directive for {\XMPC}.
%
%\subsubsection*{Syntax}
%\Syntax{coarray}
%
%\begin{tabular}{ll}
%%\verb![F]! & {\it type-declaration-stmt} {\openb}{\tt on} {\it
%% nodes-ref}{\closeb}\\
%%\verb![F]! & {\it codimension-stmt} {\openb}{\tt on} {\it nodes-ref}{\closeb}\\
%\verb![F]! & \verb|!$xmp coarray| {\it variable-name} {\openb}{\tt ,}
%     {\it variable-name} {\closeb}{\tt ...} {\tt on} {\it nodes-ref} \\
%\verb![C]! & \verb|#pragma xmp coarray| {\it variable-name} {\openb}{\tt
%     ,} {\it variable-name} {\closeb}{\tt ...} {\tt :} 
%{\it codimensions} {\openb}{\tt on} {\it nodes-ref}{\closeb}\\
%\end{tabular}
%
%\vspace{0.5cm}
%
%where {\it codimensions} is:
%
%\vspace{0.3cm}
%\hspace{0.5cm} {\openb}{\tt [}{\it int-expr}{\tt ]}...{\closeb}{\tt [*]}
%
%\subsubsection*{Description}
%
%For {\XMPF}, coarrays are declared in the same manner as in Fortran
%2008, which is not described in this document.
%%
%For {\XMPC}, coarrays are declared with the {\tt coarray} directive
%where {\it codimensions} specify the \Term{coshape} of a variable.
%%
%Terms related to coarrays in {\XMP} (e.g. coshape, coindex,
%cobound, \Term{cosubscript}, image, image index, etc.) are derived from
%that in Fortran 2008.
%
%In addition, the {\tt on} clause of the {\tt coarray} directive
%specifies a node array on each element of which the coarray object is
%allocated, and associates each {\it \Term{image}} of the coarray with a
%node of the node array so that an image having an {\it \Term{image
%index}} is associated with a node that have the same index of the node
%array in the Fortran's array element order.
%%
%If the {\tt coarray} directive is not specified in {\XMPF} or the {\tt
%on} clause of the {\tt coarray} directive is not specified in {\XMPC},
%then it is assumed that an implicit node array corresponding to the
%executing node set is specified.
%
%Note that an actual coarray and its associating dummy coarray must be
%declared on the same node set (see chapter \ref{chap:procedure}).
%
%\subsubsection*{Restrictions}
%
%\begin{itemize}
% \item \verb![F]! All of the variables appeared in a {\tt coarray}
%       directive must have the codimension attributes.
% \item If the coarray specified by {\it variable-name} is local to the
%       procedure, the node set specified by the {\tt on} clause must be
%       the executing node set of the procedure or its subset.
%% \item The node array specified by the {\tt on} clause of the {\tt
%%       coarray} directive for a procedure-local variable must
%%       correspond to the executing node set of the procedure.
%% \item \verb![F]! All of the variables declared by {\it
%%       type-declaration-stmt} in the first form with the {\tt on} clause
%%       must have the codimension attribute.
%% \item \verb![F]! At most one {\tt on} clause can be specified for a
%%       variable in a scoping unit.
%% \item The cosize of each declared coarray must be less than or equal to
%%       the size of the node set specified by the {\tt on} clause.
%\end{itemize}
%
%\subsubsection*{Examples}
%\Example{coarray}
%
%\begin{description}
%\item[Example 1]
%\hspace{\hsize}
%\begin{XFexample}
%!$xmp nodes w(50)
%      real wa(100)[*]
%      ...
%!$xmp tasks
%!$xmp task on w(1:30)
%      call task1 ( wa )
%!$xmp end task
%!$xmp task on w(32:50)
%      call task2 ( wa )
%!$xmp end task
%      ...
%      subroutine task1 ( aa )
%!$xmp nodes w(50)
%!$xmp nodes p(50) = *
%      real aa(100)[*]
%!$xmp coarray aa on w
%      real b(100)[*] 
%      ...
%      subroutine task2 ( aa )
%!$xmp nodes w(50)
%!$xmp nodes p(20) = *
%      real aa(100)[*]
%!$xmp coarray aa on w
%      real c(100)[*] 
%      ...
%\end{XFexample}
%
%	   The coarray {\tt wa} declared on the entire node set {\tt w}
%	   is passed as an actual argument to subroutines {\tt task1}
%	   and {\tt task2}.
%	   In these subroutines, the coarray object is referenced as the
%	   formal argument {\tt aa} and its declaration requires the {\tt
%	   on} clause to specify the entire node set, which is
%	   different from the executing node set.
%%	   By this declaration, the subroutine can
%%	   access {\tt wa}.
%	   Note that the coarrays {\tt b} and {\tt c} are
%	   local variables in the subroutines {\tt task1} and {\tt
%	   task2}, respectively, and therefore they are declared on the
%	   executing node set specified by {\tt p}.
%%	   Since {\tt b} is
%%	   mapped to {\tt w(1:30)} and {\tt c} is mapped onto {\tt
%%	   w(31:51)}, there is no way to access coarray {\tt c} ({\tt
%%	   b}) from {\tt task1} ({\tt task2}) because a subroutine can
%%	   access the entire node set or its subset as well as the
%%	   executing node set.
%
%\item[Example 2]
%\hspace{\hsize}
%\begin{Fexample}
%!$xmp nodes w1(200)
%      real one(100)[*]
%      real two(50)[20,*]
%\end{Fexample}
%
%%$
%
%	   The entire node set contains 200 nodes, on which coarrays {\tt
%	   one} and {\tt two} are declared. In this case, it is
%	   guaranteed that {\tt one(...)[i+(j-1)*20]} and {\tt
%	   two(...)[i,j]} reside in the same node, which is also the
%	   case for the following codes. Such a rule is compatible with
%	   that of Fortran 2008. The shape of the entire node set has
%	   nothing to do with the image of coarrays.
%
%\begin{Fexample}
%!$xmp nodes w2(20,10)
%      real one(100)[*]
%      real two(50)[20,*]
%\end{Fexample}
%
%\begin{Fexample}
%!$xmp nodes w3(10,5,4)
%      real one(100)[*]
%      real two(50)[20,*]
%\end{Fexample}
%
%\item[Example 2]
%\hspace{\hsize}
%\begin{XFexample}
%!$xmp nodes w1(200)
%!$xmp nodes w2(20,10)
%      real one(100)[*]
%      real two(50)[20,*]
%!$xmp coarray one, two on w1
%      real ichi(100)[*]
%      real ni(50)[20,*]
%!$xmp coarray ichi, ni on w2
%\end{XFexample}
%
%%In many application programs, neighbor communication
%%is an important communication pattern. The optimal mapping from
%%the logical node to the physical node depends on the number of dimensions. For
%%example, the optimal mapping for the communication between
%%{\tt one(...)[i]} and {\tt one(...)[i+1]} is different from the
%%mapping for the communication between {\tt two(...)[i,j]} and {\tt
%%  two(...)[i+1,j]}. 
%
%%In {\XMP}, several mappings can be described. In this example, the
%%mapping may be compatible with Coarray Fortran, and a different mapping
%%can be selected. By switching the runtime option, node sets {\tt w1} and
%%{\tt w2} can be different node sets. 
%
%	   The coarrays {\tt one} and {\tt two} are declared on the node
%	   set {\tt w1}, and therfore {\tt
%	   one(...)[i+(j-1)*20]} and {\tt two(...)[i,j]} reside in the
%	   same node. The coarrays {\tt ichi} and {\tt ni} are declared
%	   on a node set {\tt w2}, and {\tt
%	   ichi(...)[i+(j-1)*20]} and {\tt ni(...)[i,j]} reside in the
%	   same node.
%%
%	   However, it is not guaranteed either that {\tt one(...)[i]}
%	   and {\tt ichi(...)[i]} reside in the same node or that {\tt 
%	   two(...)[i,j]} and {\tt ni(...)[i,j]} reside in the same
%	   node.
%%
%	   Note that this point is {\it not} compatible with Fortran
%	   2008.
%
%\end{description}


%\subsection{Reference of Coarrays}
%
%\subsubsection*{Synopsis}
%
%A coarray can be directly referenced or defined by any node in the node
%set on which it is declared. The target node is specified in the same
%manner as in Fortran 2008 in {\XMPF} or using an extended notation in
%{\XMPC}.
%
%\subsubsection*{Syntax}
%\index{coarray reference}
%\index{Syntax!coarray reference}
%
%\begin{tabular}{ll}
%\verb![F]! & {\it variable} {\tt [} {\it int-expr} 
%{\openb}, {\it int-expr} {\closeb}... {\tt ]}\\
%\verb![C]! & {\it variable} {\tt :} {\tt [}{\it int-expr}{\tt ]}...\\
%\end{tabular}
%
%\subsubsection*{Description}
%
%A sequence of {\it int-expr}'s in a square bracket in {\XMPF} and a
%sequence of {\tt [}{\it int-expr}{\tt ]}'s preceded by a colon in
%{\XMPC} determine the image index for a coarray to be accessed.
%
%An reference of coarrays can appear in the same place as an reference of
%normal variables in the base languages.
%
%Note that a coarray cannot be directly accessed by a node outside the
%node set on which it is declared.%, unlike in Fortran 2008.
%
%\subsubsection*{Examples}
%\Example{coarray}
%
%In each of the following two codes, two coarrays {\tt A} and {\tt B} are
%declared on the executing node set, and each executing node gets whole
%of {\tt B} from the image 10 (that is, the tenth node of the executing
%node set) and copies it into the local storage for {\tt A}.
%
%\vspace{0.5cm}
%
%\index{Example!coarray reference}
%\begin{minipage}{0.45\hsize}
%\begin{center}
%\begin{Fexample}
%      integer A(10)[*], B(10)[*]
%
%
%      A(:) = B(:)[10]
%\end{Fexample}
%\end{center}
%\end{minipage}
%%
%\begin{minipage}{0.45\hsize}
%\begin{center}
%\begin{XCexampleR}
%int A[10], B[10];
%#pragma xmp coarray A, B : [*]
%
%A[:] = B[:]:[10];
%\end{XCexampleR}
%\end{center}
%\end{minipage}

\subsection{{\tt [C]} Synchronization of Coarrays} 
\subsubsection*{Synopsis}
{\XMPC} provides synchronization functions for coarrays.

\subsubsection*{Format}
\begin{tabular}{lll}
\verb![C]!&  {\tt void}& {\tt xmp\_sync\_all(int* {\it status})} \\
\verb![C]!&  {\tt void}& {\tt xmp\_sync\_memory(int* {\it status})} \\
\verb![C]!&  {\tt void}& {\tt xmp\_sync\_image(int {\it image}, int* {\it status})} \\
\verb![C]!&  {\tt void}& {\tt xmp\_sync\_images(int {\it num}, int* {\it image\_set}, int* {\it status})} \\
\verb![C]!&  {\tt void}& {\tt xmp\_sync\_images\_all(int* {\it status})}
\end{tabular}

\subsubsection*{Description}

\begin{itemize}
 \item {\tt xmp\_sync\_all} is equivalent to the {\tt sync all} statement in Fortran 2008.
 \item {\tt xmp\_sync\_memory} is equivalent to the {\tt sync memory} statement in Fortran 2008.
 \item A combination of {\tt xmp\_sync\_image}, {\tt xmp\_sync\_images}, and {\tt xmp\_sync\_images\_all} 
   is equivalent to the {\tt sync images} statement in Fortran 2008.
  \begin{itemize}
    \item {\tt xmp\_sync\_image} is to synchronize one image.
    \item {\tt xmp\_sync\_images} is to synchronize some images.
    \item {\tt xmp\_sync\_images\_all} is to synchronize all images.
  \end{itemize}
\end{itemize}

\subsubsection*{Arguments}
\begin{itemize}
  \item The argument {\it status} is defined with one of the follow symbolic constants.
  \begin{itemize}
    \item {\tt XMP\_STAT\_SUCCESS}
    \item {\tt XMP\_STAT\_STOPPED\_IMAGE}
  \end{itemize}
  If an execution of the function is successful, the {\it status} is defined using {\tt XMP\_STAT\_SUCCESS}.
  The condition where the {\it status} is defined using {\tt XMP\_STAT\_STOPPED\_IMAGE} is the same as 
  that where the {\it status} is defined using {\tt STAT\_STOPPED\_IMAGE} in Fortran 2008.
  These symbolic constants are defined in ``xmp.h.''
  If any other error condition occurs during the execution of these functions,
  the {\it status} is defined with a value that is different from the value of {\tt XMP\_STAT\_SUCCESS} and
  {\tt XMP\_STAT\_STOPPED\_IMAGE}.
  \item In {\tt xmp\_sync\_image}, the variable {\it image} determines a target image index.
  \item In {\tt xmp\_sync\_images}, the variable {\it num} is a number of target images, and 
        the variable {\it image\_set} is an array in which the target image set is defined.
\end{itemize}

%\subsection{{\tt [C] sync\_memory} Directive}
%
%\subsubsection*{Synopsis}
%
%In {\XMPC}, the \Directive{sync\_memory} directive is used to complete
%all memory operations on coarrays.
%
%\subsubsection*{Syntax}
%\Syntax{sync\_memory}
%
%\begin{tabular}{ll}
%\verb![C]! & \verb|#pragma xmp sync_memory| \\
%\end{tabular}
%
%\subsubsection*{Description}
%
%%The {\tt sync\_memory} directive ensures that any change of coarrays
%%on the image (or the executing node) is visible to other images (or
%%nodes) and any coarrays on the image can be referenced or defined by
%%other images.
%
%The {\tt sync\_memory} directive ensures that any changes of coarrays
%on the image (or the executing node) done by itself are visible to any
%images (or nodes) and any coarrays on the image can be referenced or
%defined by any images.
%%
%Note that no other {\XMP} directive ensures that.
%
%The {\tt sync\_memory} directive in {\XMPC} has a function equivalent
%to that of the {\tt sync memory} statement in {\XMPF}.
%
%\subsubsection*{Examples}
%\Example{sync\_memory}
%
%\begin{XCexample}
%// a, b, c, and d are coarrays
%
%a = ...;
%... = b;
%c:[p] = ...;
%... = d:[q];
%
%#pragma xmp sync_memory
%\end{XCexample}
%
%\vspace{0.5cm}
%
%When a node finishes executing the {\tt sync\_memory} directive in this
%code snippet at runtime, the following are ensured that:
%
%\begin{itemize}
% \item the local definition of a coarray {\tt a} is complete
%       and it can be referenced and defined by any other nodes;
% \item the local reference of a coarray {\tt b} is complete
%       and it can be defined by any other nodes;
% \item the remote definition of a coarray {\tt c} from the node is
%       complete and it can be referenced and defined by any nodes;
%       and
% \item the remote reference of a coarray {\tt d} from the nodes is
%       complete and it can be defined by any nodes.
%\end{itemize}

%\subsubsection*{Restrictions}
%
%\begin{itemize}
% \item \verb![C]! The variable specified by {\it variable} must be a
%       coarray.
% \item \verb![C]! The number of {\tt [}{\it int-expr}{\tt ]} must be
%       equal to the corank of the variable specified by {\it variable}.
% \item \verb![C]! The value of a each {\it int-expr} must be within the
%       cobounds for its codimension.
% \item \verb![C]! The sequence of {\tt [}{\it int-expr}{\tt ]} must
%       specify an image index value that is not greater than the cosize
%       of the variable specified by {\it variable}.
%\end{itemize}


%\section{Coarray in the {\C} language}
%
%{\XMP} adopts coarray notation. In order to use coarray notation in {\C},
%we propose some extensions of the language.

%\subsection{Declarations and reference of coarray}
%
%A coarray is declared by the coarray directive in {\C}.
%
%\Syntax{coarray}
%\begin{tabular}{ll}
%\verb![C]! & \verb|#pragma xmp| {\tt coarray} {\it coarray-dimension}
%  {\tt ::} {\it array-variable} \\
%\verb![C]! & \verb|#pragma xmp| {\tt coarray} {\it array-variable}
%{\it coarray-dimension} \\
%\end{tabular}
%
%The coarray directive declares a coarray with the coarray dimension.
%
%\subsubsection*{Examples}
%\Example{coarray}
%
%\hspace{\hsize}
%\begin{XCexample}
%int A[10], B[10];
%#pragma xmp coarray [*]: A, B
%
%double p[100][100];
%#pragma xmp coarray p[20][*]
%\end{XCexample}
%
%The coarray object is referenced in the following expression:
%
%\begin{tabular}{ll}
%\hspace{0.5cm} & {\it scalar-variable} {\tt :} {\it image-index} \\
% & {\it array-section-expression} {\tt :} {\it image-index} \\
%\end{tabular}
%
%This expression indicates the co-array on the image indicated by {\it image-index}.
%
%\hspace{\hsize}
%\begin{XCexample}
%      A[:] = B[:]:[10];    // copy from B on image 10 to A
%\end{XCexample}

\section{Directives for the Local-view Programming}
\label{154830_16Jan17}

\subsection{{\tt [F] local\_alias} Directive}
\label{172700_31Oct14}

\subsubsection*{Synopsis}

In {\XMPF}, the {\tt \Directive{local\_alias}} directive declares a
local data object as an alias to the local section of a mapped
array.

\subsubsection*{Syntax}
\Syntax{local\_alias}

\begin{tabular}{ll}
\verb![F]! & \verb|!$xmp| {\tt local\_alias} {\it local-array-name}
\verb|=>| {\it global-array-name} \\
%\verb![C]! & \verb|#pragma xmp| {\tt local\_alias} {\it local-array-name}
%\verb|=>| {\it global-array-name} \\
\end{tabular}

\subsubsection*{Description}

%This directive declares that a local array specified by
%{\it local-array-name} is a ``\Term{local alias}'' to the global array
%specified by {\it global-array-name}.
The LOCAL\_ALIAS directive associates a non-mapped array with
an explicitly mapped array.
The non-mapped array is an associating local array and
the explicitly mapped array is an associated global array.
The shape of the associating local array is the same as
that of the node-local portion of the associated global array
including the shadow area.
Each element of the associating local array shares the 
same storage unit in array-element order with that of the node-local portion 
of the associated global array.
An associating local array and the corresponding global array always have
the same allocation status.
An associating local array always has the dynamic type
and type parameter values of the corresponding 
associated global array.

%The shape of a local alias is the same as that of the local section of
%the global array that is owned by each node.
%Note that the local part includes the shadow area.
%An associating local array is defined when the corresponding 
%associated global array is defined. 


An associating local array may be a coarray.
An associating local array that is a coarray is
an on-node coarray of the node array onto which the corresponding
associated global array is mapped.
All specifications and restrictions on coarrays are also applied to an associating local
array that is a coarray, with the exception that
an associating local array is always declared with
\mytextcolor{red}{\it assumed-shape-spec-list} of the same rank as the associated global array.
In particular, a processor shall ensure that an associating local array 
that is a coarray has
the same bounds on all the images corresponding to the node array
onto which the corresponding associated global array is mapped.
The mapping attributes that are allowed for an associated global array are
processor dependent.

Note that the base language {\Fort} is extended so that a deferred-shape
array that is neither an allocatable array nor an array pointer is
declared if it is specified as a 
{\it local-array-name} in the {\tt local\_alias} directive.

In {\XMPC}, the {\tt address-of} operator is applied to global data
substitutes for the {\tt local\_alias} directive (see \ref{sec:pointer
to global data}).


\subsubsection*{Restrictions}

\begin{itemize}
 \item A {\it global-array-name} shall be the name of an explicitly
       mapped array declared in the same scoping unit. 

 \item A {\it local-array-name} shall be the name of a non-mapped array
       declared in the same scoping unit. 

 \item A {\it local-array-name} shall not be a dummy argument.

 \item An associating local array shall have the declared type and type parameters of
       the corresponding associated global array.

 \item An associating local array shall be declared with
       \mytextcolor{red}{\it assumed-shape-spec-list} of the
       same rank as the corresponding associated global array.

 \item A {\it local-array-name} shall appear in a {\tt coarray}
       directive in the same scoping unit and the {\it node-name} in the
       {\tt coarray} directive 
       shall be the name of the node array onto which the
       associated global array is mapped.

 \item 
If an associated global array is a dummy argument and corresponds to
an associating local array that is a coarray, 
the corresponding effective argument shall be
an explicitly mapped array or a subobject of 
an explicitly mapped array whose name appears
in a LOCAL\_ALIAS directive, %as a {\it global-array-name} 
and the corresponding associating local array shall be a coarray.

 \item If a dummy argument is a coarray 
       and the corresponding ultimate argument is a coarray appearing in
       a LOCAL\_ALIAS directive, % as a {\it local-array-name},
       the dummy argument shall appear in a COARRAY directive
       with a node array corresponding to
       a subset of the set of images that corresponds to the node array
       onto which the ultimate argument is mapped.
%
% \item The array specified by {\it local-array-name} must not be aligned
%       by an {\tt align} directive.
% \item The array specified by {\it global-array-name} must be aligned
%       by an {\tt align} directive.
% \item The data type and rank of the array specified by {\it
%       local-array-name} must be the same as those of the array
%       specified by {\it global-array-name}.
% \item The array specified by {\it local-array-name} must be a
%       deferred-shape array, which means that it must be declared with 
%       a {\it deferred-shape-spec-list} in a type declaration statement
%       or a {\tt DIMENSION} statement.
% \item \verb![C]! The array specified by {\it local-array-name} must be
%       a pointer.
\end{itemize}

\subsubsection*{Examples}
\Example{local\_alias}

\begin{description}

\item[Example 1]
\hspace{\hsize}

\begin{XFexample}
!$xmp nodes n(4)
!$xmp template :: t(100)
!$xmp distribute (block) onto n :: t

      real :: a(100)
!$xmp align (i) with t(i) :: a
!$xmp shadow (1) :: a

      real :: b(:)
!$xmp local_alias b => a
\end{XFexample}

The array {\tt a} is distributed by block onto four nodes.
The node {\tt n(2)} has its local section of 25elements ({\tt
a(25:50)}) with shadow areas of size one on both the upper and lower 
bounds. The local alias {\tt b} is an array of 27 elements
({\tt b(1:27)}) on {\tt n(2)}. The table below shows the correspondence
of each element of {\tt a} to that of {\tt b} on {\tt n(2)}.

\begin{center}
\begin{tabular}{|c|c|}\hline
{\tt a}      & {\tt b} \\ \hline\hline
lower shadow & 1       \\ \hline
26           & 2       \\ \hline
27           & 3       \\ \hline
28           & 4       \\ \hline
\dots        & \dots   \\ \hline
50           & 26      \\ \hline
upper shadow & 27      \\ \hline
\end{tabular}
\end{center}

\item[Example 2]
\hspace{\hsize}
\begin{XFexample}
!$xmp nodes n(4)
!$xmp template :: t(100)
!$xmp distribute (cyclic) onto n :: t

      real :: a(100)
!$xmp align (i) with t(i) :: a

      real :: b(0:)
!$xmp local_alias b => a
\end{XFexample}

An array {\tt a} is distributed cyclically onto four nodes.
Node {\tt n(2)} has its local section of 25 elements ({\tt
a(2:100:4)}).
%with shadow areas of size one on the upper and lower 
%bounds.
The lower bound of local alias {\tt b} is declared to
be zero. As a result, {\tt b} is an array of size 25 whose lower
bound is zero ({\tt b(0:24)}) on {\tt n(2)}. The table below shows the
correspondence of each element of a to that of {\tt b} on {\tt n(2)}.

\begin{center}
\begin{tabular}{|c|c|}\hline
{\tt a}      & {\tt b} \\ \hline\hline
2            & 0       \\ \hline
6            & 1       \\ \hline
10           & 2       \\ \hline
\dots        & \dots   \\ \hline
98           & 24      \\ \hline
\end{tabular}
\end{center}

\item[Example 3]
\hspace{\hsize}
\Example{template\_fix}
\begin{XFexample}
!$xmp nodes n(4)
!$xmp template :: t(:)
!$xmp distribute (block) onto n :: t

      real, allocatable :: a(:)
!$xmp align (i) with t(i) :: a

      real :: b(:)[*]
!$xmp local_alias b => a

      …

!$xmp template_fix :: t(128)

      allocate (a(128))

      if (me < 4) b(4) = b(4)[me +1]
\end{XFexample}

Because the global array {\tt a} is an allocatable array, its local alias
{\tt b} is not defined when the subroutine starts execution. {\tt b} is
defined when {\tt a} is allocated at the {\tt allocate}
statement. Note that {\tt b} is declared as a coarray, and can therefore
be accessed in the same manner as a normal coarray.

\end{description}


%\subsection{{\tt async} Construct}
%
%\subsubsection*{Synopsis}
%
%The {\tt \Directive{async}} construct allows coarrays to be referenced
%asynchronously.
%
%\subsubsection*{Syntax}
%\Syntax{async}
%
%\begin{tabular}{ll}
%\verb![F]! & \verb|!$xmp| {\tt async ( {\it async-id} )} \\
%\verb![C]! & \verb|#pragma xmp| {\tt async ( {\it async-id} )} \\
%\end{tabular}
%
%\subsubsection*{Description}
%
%An {\tt async} directive and the following statement
%including coarray references constitute an {\tt async} construct.
%
%Coarrays appeared in an {\tt async} construct are referenced
%asynchronously and therefore statements following it may be executed
%before the operation is complete.
%
%\subsubsection*{Restriction}
%
%\begin{itemize}
% \item {\it async-id} must be an integer expression.
%\end{itemize}


\subsection{{\tt post} Construct}

\subsubsection*{Synopsis}

The {\tt \Directive{post}} construct, in combination with the {\tt
\Directive{wait}} construct, specifies the point-to-point
synchronization.

\subsubsection*{Syntax}
\Syntax{post}

\begin{tabular}{ll}
\verb![F]! & \verb|!$xmp| {\tt post ( {\it nodes-ref}, {\it tag} )}\\
\verb![C]! & \verb|#pragma xmp| {\tt post ( {\it nodes-ref}, {\it tag} )}\\
\end{tabular}

\subsubsection*{Description}

%This construct prohibits statements that follow the construct from being
%executed until the execution of all statements preceding a matching
%\Directive{post} construct is completed on the node specified by {\it
%node-ref}.

This construct ensures that the execution of statements that precede
it is completed before statements that follow the matching
\Directive{wait} construct start are executed.

A {\tt \Directive{post}} construct issued with the arguments of {\it
nodes-ref} and {\it tag} on a node (called a {\it \Term{posting node}})
dynamically matches at most one {\tt \Directive{wait}} construct issued
with the arguments of the posting node (unless omitted) 
and the same value as {\it tag} (unless omitted) by the node specified
by {\it nodes-ref}.

\subsubsection*{Restriction}

\begin{itemize}
 \item {\it nodes-ref} must represent one node.
 \item {\it tag} must be an expression of type default integer, in
       {\XMPF}, or type {\tt int}, in {\XMPC}.
\end{itemize}

\subsubsection*{Example}
\Example{post}
\Example{wait}

\begin{description}

\item[Example 1]
\hspace{\hsize}

\begin{minipage}{0.45\hsize}
\begin{center}
\begin{XFexample}
      S1
!$xmp post (p(2), 1)
\end{XFexample}
\end{center}
\end{minipage}
%
\begin{minipage}{0.45\hsize}
\begin{center}
\begin{XFexample}
!$xmp wait (p(1), 1)
      S2
\end{XFexample}
\end{center}
\end{minipage}

It is assumed that the code of the left is executed by the node {\tt
p(1)}, while that on  the right is executed by node {\tt p(2)}.

The {\tt post} construct on the left matches the {\tt wait} construct on
the right because their {\it nodes-ref}s represent each other and both
{\it tags}'s have the same value of one.
%
These constructs ensure that no statement in {\tt S2} is executed by
{\tt p(2)} until the execution of all statements in {\tt S1} is
completed by {\tt p(1)}.

\item[Example 2]
\hspace{\hsize}

\begin{XFexample}
!$xmp wait
      S3
\end{XFexample}

It is assumed that this code is executed by node {\tt p(2)}.

The {\tt post} construct in the left code in Example 1 may match
this {\tt wait} construct because both {\it nodes-ref} and
{\it tag} are omitted in this {\tt wait} construct.

\end{description}

\subsection{{\tt wait} Construct}

\subsubsection*{Synopsis}

The {\tt \Directive{wait}} construct, in combination with the {\tt
\Directive{post}} construct, specifies a point-to-point
synchronization.

\subsubsection*{Syntax}
\Syntax{wait}

\begin{tabular}{ll}
\verb![F]! & \verb|!$xmp| {\tt wait} {\openb}{\tt (} {\it nodes-ref}
 {\openb}, {\it tag}{\closeb} {\tt )}{\closeb}\\
\verb![C]! & \verb|#pragma xmp| {\tt wait} {\openb}{\tt (} {\it nodes-ref} 
 {\openb}, {\it tag}{\closeb} {\tt )}{\closeb}\\
\end{tabular}

\subsubsection*{Description}

This construct prohibits statements that follow from
being executed until the execution of all statements preceding a
matching \Directive{post} construct is completed on the node specified
by {\it node-ref}.

A {\tt \Directive{wait}} construct that is issued with the arguments of {\it nodes-ref} and {\it tag} 
on a node (called a {\it \Term{waiting node}})
dynamically matches a {\tt \Directive{post}} construct issued with the
arguments of the waiting node and the same value as {\it tag} by the
node specified by {\it nodes-ref}.

If {\it tag} is omitted, then the {\tt \Directive{wait}} construct can
match a {\tt \Directive{post}} construct that is issued with the
arguments of the waiting node and any tag by the node specified by {\it nodes-ref}.
%
If both {\it tag} and {\it nodes-ref} are omitted, then the {\tt
\Directive{wait}} construct can match a {\tt \Directive{post}}
construct that is issued with the arguments of the waiting node and any
tag on any node.

\subsubsection*{Restriction}

\begin{itemize}
 \item {\it nodes-ref} must represent one node.
 \item {\it tag} must be an expression of type default integer, in
       {\XMPF}, or type {\tt int}, in {\XMPC}.
\end{itemize}

\subsection{{\tt [C]} {\tt lock}/{\tt unlock} Construct}
\subsubsection*{Synopsis}
The {\tt \Directive{lock}}/{\tt \Directive{unlock}} constructs are equivalent to the 
{\tt lock}/{\tt unlock} statements in Fortran 2008.

\subsubsection*{Syntax}
\Syntax{lock}
\begin{tabular}{ll}
\verb![C]! & \verb!#!include \verb!<!xmp.h\verb!>! \\
\verb![C]! & {\tt xmp\_lock\_t} {\it lock-object}
{\openb}{\tt ,} {\it lock-object} {\closeb}{\tt ...} \\
\verb![C]! & \verb|#pragma xmp| {\tt lock} {\tt (}{\it lock-object}{\tt )}
{\openb} {\tt acquired\_lock} {\tt (}{\it success}{\tt )} {\closeb}
{\openb} {\tt stat} {\tt (}{\it status}{\tt )} {\closeb}\\
\verb![C]! & \verb|#pragma xmp| {\tt unlock} {\tt (}{\it lock-object}{\tt )}
{\openb} {\tt stat} {\tt (}{\it status}{\tt )} {\closeb}
\end{tabular}

\vspace{5mm}
Please note the following points:
\begin{itemize} \vspace{-2mm}
 \item The type {\tt xmp\_lock\_t} is defined in ``xmp.h''. \vspace{-3mm}
 \item The variable {\it lock-object} is a coarray. \vspace{-3mm}
 \item The variable {\it success} is an expression of type {\tt int}. \vspace{-3mm}
 \item The variable {\it status} is an expression of type {\tt int}. 
\end{itemize}

\subsubsection*{Description}
%The {\tt \Directive{lock}} construct, in combination with the {\tt \Directive{unlock}} construct,
%is used to create a critical section on a target node.
%The critical section is a range of a code which can be executed by only a node creates the critical section.
%The critical section is started by the {\tt \Directive{lock}} construct,
%and the critical section is ended by the {\tt \Directive{unlock}} construct.
The {\tt \Directive{lock}} construct, in combination with the {\tt \Directive{unlock}} construct,
is used to control a {\it lock-object}.
The {\it lock-object} must be defined as a coarray to control it on a target node.
The {\it lock-object} must be an expression of type {\tt xmp\_lock\_t}, 
which is an opaque object defined in ``xmp.h''.

If the {\tt acquired\_lock} clause is not used in the {\tt \Directive{lock}} construct and 
the {\it lock-object} is locked,
the node stops at the {\tt \Directive{lock}} construct until the {\it lock-object} is unlocked by a different node.
If the {\tt acquired\_lock} clause is used in the {\tt \Directive{lock}} construct and 
the {\it lock-object} is locked by a different node,
the node does not stop at the {\tt \Directive{lock}} construct and the
variable {\it success} is defined with the value false;
{\tt \Directive{lock}} construct leaves the {\it lock-object}
unchanged.
If the {\tt acquired\_lock} clause is used in the {\tt \Directive{lock}}
construct and the {\it lock-object} is unlocked,
the variable {\it success} is defined with the value true.

The {\it status} is defined with one of the follow symbolic constants 
when executing the {\tt \Directive{lock}}/{\tt \Directive{unlock}}
construct.

\begin{itemize}
\item {\tt XMP\_STAT\_SUCCESS} \vspace{-2mm}
\item {\tt XMP\_STAT\_LOCKED} \vspace{-2mm}
\item {\tt XMP\_STAT\_UNLOCKED} \vspace{-2mm}
\item {\tt XMP\_STAT\_LOCKED\_OTHER\_IMAGE}
\end{itemize}

If the execution of the {\tt \Directive{lock}}/{\tt \Directive{unlock}}
construct is successful, 
the {\it status} is defined with {\tt XMP\_STAT\_SUCCESS}.
The condition where the {\it status} is defined with 
{\tt XMP\_STAT\_LOCKED}, {\tt XMP\_STAT\_UNLOCKED}, or {\tt XMP\_STAT\_}
{\tt LOCKED\_OTHER\_IMAGE} 
is the same as that where the {\it status} is defined with
{\tt STAT\_LOCKED}, {\tt STAT\_UNLOCKED}, or {\tt STAT\_LOCKED\_OTHER\_IMAGE} in Fortran 2008.
These symbolic constants are defined in ``xmp.h''.
If any other error condition occurs during the execution of these
constructs, 
the {\it status} is defined with a value that is different from the
value of {\tt XMP\_STAT\_SUCCESS}, {\tt XMP\_STAT\_LOCKED}, 
{\tt XMP\_STAT\_UNLOCKED}, and {\tt XMP\_STAT\_LOCKED\_OTHER\_IMAGE}.


\subsubsection*{Example}
\begin{XCexample}
#include "xmp.h"

xmp_lock_t lock_obj:[*];
int A:[*], B;
#pragma xmp nodes p[2]
...
#pragma xmp lock(lock_obj:[2])
  if(xmp_node_num() == 1){
    A:[2] = B;
  }
#pragma xmp unlock(lock_obj:[2])
\end{XCexample}

%\subsection{{\tt critical} Construct}
%
%\subsubsection*{Synopsis}
%
%The {\tt \Directive{critical}} construct restricts execution of the
%associated structured block to a single node at any given time.
%
%\subsubsection*{Syntax}
%\Syntax{critical}
%
%\begin{tabular}{ll}
%\verb![F]! & \verb|!$xmp| {\tt critical} \\
%& {\it block} \\
%& \verb|!$xmp| {\tt end critical} \\
%& \\
%\verb![C]! & \verb|#pragma xmp| {\tt critical} \\
%& {\it block} \\
%\end{tabular}
%
%%\subsubsection*{Restriction}
%
%\subsubsection*{Description}
%
%This construct specifies a {\it critical region}, at the beginning of
%which a node waits until no other node is executing it. 
%
%%\subsubsection*{Examples}
%%\Example{critical}
%%
%%\begin{XFexample}
%%!$xmp critical
%%      ...
%%!$xmp end critical
%%\end{XFexample}
