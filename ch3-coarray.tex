\chapter{Support for local-view programming}

\XMP adopts coarray notations as
an extension of languages for local-view programming. In the case of
\Fort as the base language, most coarray notations are compatible with
that of \CAF (CAF), except that the {\tt task} constructs
are used for task parallelism.

This line was inserted by Sakagami for svn test.


\section{Coarray notation of  \XMP}

The \Term{coarray} is declared and referenced as in CAF. In addition, the
notations are extended in \XMP so that the coarray declarations
can be followed by the {\tt on} clause ({\tt on} {\it nodes-ref}). In this
case, the images of the coarray are allocated on the node set specified by
{\it nodes-ref}. If the {\tt on} clause is not specified, then the images of
the coarray are allocated on the executing node set as a default.

\subsubsection*{Examples}
\Example{nodes}
\Example{task}
\Example{tasks}
\Example{end task}
\Example{end tasks}

\begin{description}
\item[Example 1]
\hspace{\hsize}
\begin{Fexample}
!$xmp nodes w(50)
      real wa(100)[*]
      ...
!$xmp tasks
!$xmp task on w(1:30)
      call task1 ( wa )
!$xmp end task
!$xmp task on w(32:50)
      call task2 ( wa )
!$xmp end task
      ...
      subroutine task1 ( aa )
!$xmp nodes w(50)
!$xmp nodes p(50) = *
      real aa(100)[*] on w
      real b(100)[*] 
      ...
      subroutine task2 ( aa )
!$xmp nodes w(50)
!$xmp nodes p(20) = *
      real aa(100)[*] on w
      real c(100)[*] 
      ...
\end{Fexample}

Coarray {\tt wa} mapped onto the entire node set {\tt w} is passed as an
argument to subroutines {\tt task1} and {\tt task2}. In these
subroutines, the argument is referenced by the parameter {\tt aa}. In this
case, the declaration of the coarray for the parameter {\tt aa} requires an on
clause to specify the executing node set, which is different from the
entire node set. By this declaration, the subroutine can access {\tt
  wa}. Note that coarrays {\tt b} and {\tt c} are local
variables in subroutines {\tt task1} and {\tt task2}, respectively, and
these coarrays are mapped onto the executing node set indicated by {\tt
  p}. Since {\tt b} is mapped to {\tt w(1:30)} and {\tt c} is mapped
onto {\tt   w(31:51)}, there is no way to access coarray {\tt c}
({\tt b}) from {\tt task1} ({\tt task2}) because a subroutine can
access the entire node set or its subset as well as the executing node set. 

\item[Example 2]
\hspace{\hsize}
\begin{Fexample}
!$xmp nodes w1(20)
      real one(100)[*]
      real two(50)[20,*]
\end{Fexample}

The entire node set has 200 nodes, onto which coarray one and two are
mapped. In this case, it is guaranteed that {\tt one(...)[i+(j-1)*20]}
and {\tt two(...)[i,j]} are mapped into the same node. This mapping rule 
is compatible with Coarray Fortran. This is also the case for the following
code. That is, the shape of the entire node set has nothing to coarray.  

\begin{Fexample}
!$xmp nodes w2(20,10)
      real one(100)[*]
      real two(50)[20,*]
      ...
OR
!$xmp nodes w3(10,5,4)
      real one(100)[*]
      real two(50)[20,*]
      ...
\end{Fexample}

\item[Example 2]
\hspace{\hsize}
\begin{Fexample}
!$xmp nodes w1(200)
!$xmp nodes w2(20,10)
      real one(100)[*]
      real two(50)[20,*]
      real ichi(100)[*] on w2
      real ni(50)[20,*] on w2
\end{Fexample}

In many application programs, neighbor communication
is an important communication pattern. The optimal mapping from
the logical node to the physical node depends on the number of dimensions. For
example, the optimal mapping for the communication between
{\tt one(...)[i]} and {\tt one(...)[i+1]} is different from the
mapping for the communication between {\tt two(...)[i,j]} and {\tt
  two(...)[i+1,j]}. 

In \XMP, several mappings can be described. In this example, the
mapping may be compatible with Coarray Fortran, and a different mapping
can be selected. By switching the runtime option, node sets {\tt w1} and
{\tt w2} can be different node sets. Coarrays {\tt one} and {\tt two}
are mapped to node set {\tt w1} (that is, the entire node set), and
{\tt one(...)[i+(j-1)*20]} and {\tt two(...)[i,j]} are mapped to the
same node. Coarrays {\tt ichi} and {\tt ni} are mapped onto a different
node set {\tt w2}, and {\tt ichi(...)[i+(j-1)*20]} and {\tt
  ni(...)[i,j]} are mapped onto the same physical node. However, it is not
guaranteed that a pair of {\tt one(...)[i]} and {\tt ichi(...)[i]} and a
pair of {\tt two(...)[i,j]} and {\tt ni(...)[i,j]} are
mapped into the same node. Note that this point is not compatible with Coarray Fortran. 

\end{description}

\section{Coarray in the \C language}

\XMP adopts coarray notation. In order to use coarray notation in \C,
we propose some extensions of the language.

\subsection{Array section notation}

Array section notation is a notation to describe the part of array that
is adapted in Fortran90. In \C, an array section has the following form:

\begin{tabular}{ll}
\hspace{0.5cm} & {\it array-name} '[' [{\it lower-bound}] ':' [{\it
  upper-bound}] [':' {\it step}] ']' ...
\end{tabular}

An array section is built from subsets of the elements of an array
object -- those associated with a selected subset of the index range
attached to the object. The {\it upper-bound} and the {\it lower-bound} 
specify the range of array elements of an array object. Either the
{\it upper-bound} or the {\it lower-bound} can be omitted in the index
range of a section, in which case they default to the lowest or
highest values taken by the index of the array. Therefore, {\tt A[:]} is a section
containing the whole of {\tt A}. If the {\it step} is specified,  
then the elements of an array section are every �e�estep"-th element in the
specified range. For example, {\tt B[1:10:3]} is an array section of
size 4 containing every third element of {\tt B} with indices between
1 and 10 (i.e., indices 1, 4, 7, 10). Collectively, ranges specified by
{\it upper-bound}, {\it lower-bound}, and {\it step} are referred to as
triplets. For multi-dimensional arrays, some dimensions can be
subscripted with a normal scalar expression, and some dimensions can be ``sectioned'' with triplets.  

\subsubsection*{Example}

Array {\tt A} is declared by

\begin{tabular}{ll}
\hspace{0.5cm} & {\tt int A[100];} \\
\end{tabular}

Then:

\begin{tabular}{lll}
\hspace{0.5cm} & {\tt A[10:19]} & array section of 10 elements from A[10] to
A[19] \\
 & {\tt A[10:]} & array section of 90 elements from A[10] to A[99] \\
 & {\tt A[:9]} & array section of 10 elements from A[0] to A[9] \\
 & {\tt A[10:19:2]} & array section of 5 elements from A[10] to A[18]
 by step 2 \\
 & {\tt A[:]} & entire array of A \\
\end{tabular}

\subsection{Assignment of array sections}

Array-valued expressions can be used by array section in assignments.

\begin{tabular}{ll}
\hspace{0.5cm} & {\tt {\it array-section1} = {\it array-section2}} \\
\end{tabular}

The expression on the right-hand side of the assignment must be
conformable with the array variable on the left-hand side. Thus, both sides must have the same shape, i.e., the same number of dimensions and size of each dimension.

\subsubsection*{Examples}

\hspace{\hsize}
\begin{Cexample}
int A[10];
int B[5];
    ...
A[5:9] = B[0:4]; // copy the elements from A[5] to A[9],
                 // to the elements from B[0} to B[4]
\end{Cexample}

\subsection{Declarations and reference of coarray}

A coarray is declared by the coarray directive in \C.

\Syntax{coarray}
\begin{tabular}{ll}
\verb![C]! & \verb|#pragma xmp| {\tt coarray} {\it coarray-dimension}
  {\tt ::} {\it array-variable} \\
\verb![C]! & \verb|#pragma xmp| {\tt coarray} {\it array-variable}
{\it coarray-dimension} \\
\end{tabular}

The coarray directive declares a coarray with the coarray dimension.

\subsubsection*{Examples}
\Example{coarray}

\hspace{\hsize}
\begin{Cexample}
int A[10], B[10];
#pragma xmp coarray [*]: A, B

double p[100][100];
#pragma xmp coarray p[20][*]
\end{Cexample}

The coarray object is referenced in the following expression:

\begin{tabular}{ll}
\hspace{0.5cm} & {\it scalar-variable} {\tt :} {\it image-index} \\
 & {\it array-section-expression} {\tt :} {\it image-index} \\
\end{tabular}

This expression indicates the co-array on the image indicated by {\it image-index}.

\hspace{\hsize}
\begin{Cexample}
      A[:] = B[:]:[10];    // copy from B on image 10 to A
\end{Cexample}
