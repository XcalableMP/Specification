
\section{Directive for local view programming}

\subsection{{\tt local\_alias} directive}

\subsubsection*{Synopsis}

The {\tt \Directive{local\_alias}} directive declares the local data
object as an alias to the data of a distributed array allocated in each
node.

\subsubsection*{Syntax}
\Syntax{local\_alias}

\begin{tabular}{ll}
\verb![F]! & \verb|!$xmp| {\tt local\_alias ::} {\it local-array-name}
\verb|=>| {\it global-array-name} \\
\verb![C]! & \verb|#pragma xmp| {\tt local\_alias ::} {\it local-array-name}
\verb|=>| {\it global-array-name} \\
\end{tabular}

\subsubsection*{Description}

This directive declares that the local array specified by
{\it local-array-name} is a "local alias" to the global array specified by
{\it global-array-name}.

The shape of a local alias is the same as that of the local part of the
corresponding global array that is owned by each node. Note that the
local part includes the shadow area.

A local alias becomes defined when the corresponding global array
becomes defined. If the corresponding global array is statically
allocated, then the local alias is always defined in its scoping unit;
if not, the local alias is not defined until the corresponding global
array is aloocated.

\subsubsection*{Restriction}

\begin{itemize}
 \item The array specified by {\it local-array-name} must not be aligned
       by an {\tt align} directive.
 \item The array specified by {\it global-array-name} must be aligned
       by an {\tt align} directive.
 \item The data type and rank of the array specified by {\it
       local-array-name} must be the same as those of the array
       specified by {\it global-array-name}.
 \item \verb![F]! The array specified by {\it local-array-name} must be
       a deferred-shape array.
 \item \verb![C]! The array specified by {\it local-array-name} must be
       a pointer.
\end{itemize}

\subsubsection*{Examples}

\begin{description}
\item[Example 1]
\hspace{\hsize}
\begin{Fexample}
!$xmp nodes n(4)
!$xmp template :: t (100)
!$xmp distribute (cyclic) onto n :: t

      real :: a (100)
!$xmp align (i) with t(i) :: a
!$xmp shadow (1) :: a

      real :: b(0:)
!$xmp local_alias :: b => a
\end{Fexample}

The node {\tt n(2)} has a local array of twenty-five elements ({\tt
a(2:100:4)}) with shadow areas of size one on the lower and upper
bounds. The lower bound of the local alias {\tt b} is declared to
be zero. As a result, {\tt b} is an array of twenty-seven elements
({\tt b(0:26)}) on {\tt n(2)}. The table below shows the correspondence
of each element of a to that of {\tt b}.

\begin{center}
\begin{tabular}{|c|c|}\hline
{\tt a}      & {\tt b} \\ \hline\hline
lower shadow & 0       \\ \hline
2            & 1       \\ \hline
6            & 2       \\ \hline
10           & 3       \\ \hline
\dots        & \dots   \\ \hline
98           & 25      \\ \hline
upper shadow & 26      \\ \hline
\end{tabular}
\end{center}

\item[Example 2]
\hspace{\hsize}
\begin{Fexample}
!$xmp nodes n(4)
!$xmp template :: t(:)
!$xmp distribute (block) onto n :: t

      real , allocatable :: a(:)
!$xmp align (i) with t(i) :: a

      real :: b(:)[*]
!$xmp local_alias :: b => a

      �c

!$xmp template_fix :: t(128)

      allocate (a(128))

      if (me < 4) b(4) = b(4)[me +1]
\end{Fexample}

Since the global array {\tt a} is an allocatable array, its local alias
{\tt b} is not defined when the subroutine starts execution. {\tt b} is
defined when {\tt a} is allocated at the {\tt allocate}
statement. Note that {\tt b} is declared as an coarray and therefore can be
accessed in the same way as a normal coarray.

\end{description}

\subsection{Post directive}

\subsection{Wait directive}

\subsection{Critical directive}

