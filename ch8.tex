\chapter{OpenMP in {\XMP} Programs}
\label{chap:openmp}

The usage of OpenMP directives in {\XMP} programs is subjected to
the following basic rule.

\begin{itemize}
 \item {\XMP} directives and the invocation of an {\XMP}
       intrinsic/built-in procedure should be single-threaded, and they
       may therefore be placed in the sequential part, or one of
       the {\tt single}, {\tt master}, or {\tt critical} regions
       that are closely nested inside a {\tt parallel} region whose
       parent thread is the initial thread;

 \item with the exception that the {\XMP}'s {\tt loop} directive that
       controls a loop can be placed immediately inside the OpenMP's
       parallel loop directive ({\tt parallel do} for Fortran and
       {\tt parallel for} for C), which controls the identical loop.
\end{itemize}

The behavior of coarray references in a {\tt parallel} region is
implemetation-defined.

\subsubsection*{Examples}
\index{Example!OpenMP in XcalableMP programs}

Assume that the following codes are placed in the sequential part of
the program.

\begin{XCexample}
#pragma omp parallel for 
for (...){
  #pragma xmp barrier  // NG because not single-threaded
}
\end{XCexample}

\begin{XCexample}
#pragma omp parallel for 
for (...){
  #pragma omp single 
  {
    #pragma xmp barrier  // OK because single-threaded
                         // (inside a single region)
  }
}
\end{XCexample}

\begin{XCexample}
#pragma omp parallel for
#pragma xmp loop  // OK because immediately nested
for (...){
  ...
}
\end{XCexample}

\begin{XCexample}
#pragma xmp loop  // OK because single-threaded (not nested)
#pragma omp parallel for
for (...){
  ...
}
\end{XCexample}

\begin{XCexample}
#pragma xmp loop  // OK because single threaded (not nested)
for (...){
  #pragma omp parallel for
  for (...) { ... }
}
\end{XCexample}

\begin{XCexample}
#pragma omp parallel for 
for (...){
  #pragma xmp loop  // NG because not immediately nested
  for (...) { ... }
}
\end{XCexample}
