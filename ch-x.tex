\chapter{Programming Interface for {\MPI} and {\OMP}}

   This section describes the programming interface for {\MPI} and {\OMP},
   which are widely used for parallel programming for cluster computing.
   Users can introduce {\MPI} and {\OMP} functions to {\XMP} using the interface.   

\section{{\MPI} Interface}

   {\XMP} provides the following user API functions to mix {\MPI} functions with {\XMP}

\begin{itemize}
\item {\tt xmp\_get\_mpi\_comm}
\item {\tt xmp\_init\_mpi}
\item {\tt xmp\_finalize\_mpi}
\end{itemize}

\subsection{\tt xmp\_get\_mpi\_comm}

\subsubsection*{Format}

\begin{tabular}{lll}

\verb![C]!&  {\tt int}& {\tt xmp\_get\_mpi\_comm(nodes-name)}\\

\verb![F]!&  {\tt integer function}& {\tt xmp\_get\_mpi\_comm(nodes-name)}
\end{tabular}

\subsubsection*{Synopsis}
   xmp\_get\_mpi\_comm returns the integer value with the associated communicator to which
   nodes-name belongs. xmp\_mpi\_comm returns the executing MPI communicator when nodes-name is omitted.

\subsubsection*{Arguments}
   nodes-name is the name of a node set.

\subsection{\tt xmp\_init\_mpi}

\subsubsection*{Format}

\begin{tabular}{lll}

\verb![C]!&  {\tt void}& {\tt xmp\_init\_mpi(int *args, char ***argv)}\\

\verb![F]!&  {\tt }& {\tt xmp\_init\_mpi()}
\end{tabular}

\subsubsection*{Synopsis}

   xmp\_init\_mpi initializes MPI execution environment.

\subsubsection*{Arguments}

   In C, argc and argv, command-line arguments, should be given to xmp\_init\_mpi.


\subsection{\tt xmp\_finalzie\_mpi}

\subsubsection*{Format}

\begin{tabular}{lll}

\verb![C]!&  {\tt void}& {\tt xmp\_finalize\_mpi(void)}\\

\verb![F]!&  {\tt }& {\tt xmp\_finalize\_mpi()}
\end{tabular}

\subsubsection*{Synopsis}

   xmp\_finalize\_mpi terminates MPI execution enviroment.

\subsubsection*{Arguments}

   none.

\subsection*{Example}
\begin{Cexample}
#include <stdio.h>
#include "mpi.h"
#include "xmp.h"

#pragma xmp nodes p(4)

int main(int argc, char *argv[]) {
  xmp_init_mpi(&argc, &argv)

  int rank, size;
  MPI_Comm_rank(MPI_COMM_WORLD, &rank);
  MPI_Comm_size(MPI_COMM_WORLD, &size);

#pragma xmp task on p(2:3)
{
  MPI_Comm comm = xmp_get_mpi_comm(); // get the MPI communicator of p(2:3)

  int rank, size;
  MPI_Comm_rank(comm, &rank);
  MPI_Comm_size(comm, &size);
}

  xmp_finalize_mpi();

  return 0;
}
\end{Cexample}


\section{{\OMP}}

\chapter{Interface to Numerical Libraries}

   This section describe XcalableMP interfaces to existing MPI parallel library, 
   in order to develop good productive and high performance XcalableMP programs 
   using existing MPI parallel libraries.
   
\section{ScaLAPACK}

   This subsection shows main characteristics in case of developing
   XcalableMP programs using XcalableMP interfaces to ScaLAPACK library routines.

\begin{itemize}
\item The name of XcalableMP interface prefixes "XMP\_" or "xmp\_" 
      to ScaLAPACK library routine name.
\item Execution flow is the following:\\
      XMP programs $\to$ XMP interface $\to$ native ScaLAPACK library routines
\item If arguments of subroutines have array descriptor,
      we replace the argument with "xmp\_desc\_of".
      Then we don't need array descriptor. 
\item For ScaLAPACK library routine having descriptor array as argumet,
      the XcalableMP interface routine have BLACS context handle including 
      the descriptor array as new argument.
\item The blacs\_exit routine is unnecessary, because XcalableMP program executes process 
      corresponding to MPI\_Finalize routines.
\item A valid value of the argument "order" of BLACS routine "blacs\_gridinit" is 
      only column-major. 
\end{itemize}
