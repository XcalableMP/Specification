\chapter{Programming Interface for {\MPI} and {\OMP}}

   This chapter describes the programming interface for {\MPI} and {\OMP},
   which are widely used for parallel programming for cluster computing.
   Users can introduce {\MPI} and {\OMP} functions to {\XMP} using the interface.   

\section{{\MPI} Interface}

   {\XMP} provides the following user API functions to mix {\MPI}
   functions with {\XMP}.

\begin{itemize}
\item {\tt xmp\_get\_mpi\_comm}
\item {\tt xmp\_init\_mpi}
\item {\tt xmp\_finalize\_mpi}
\end{itemize}

\subsection{\tt xmp\_get\_mpi\_comm}

\subsubsection*{Format}

\begin{tabular}{lll}
\verb![F]!&  {\tt integer function}& {\tt xmp\_get\_mpi\_comm()}\\
\verb![C]!&  {\tt MPI\_Comm}& {\tt xmp\_get\_mpi\_comm(void)}
\end{tabular}

\subsubsection*{Synopsis}

   {\tt xmp\_get\_mpi\_comm} returns the handle of the communicator
   associated with the executing node set. 

\subsubsection*{Arguments}

none.

\subsection{\tt xmp\_init\_mpi}

\subsubsection*{Format}

\begin{tabular}{lll}
\verb![F]!&  {\tt }& {\tt xmp\_init\_mpi()}\\

\verb![C]!&  {\tt void}& {\tt xmp\_init\_mpi(int *args, char ***argv)}
\end{tabular}

\subsubsection*{Synopsis}

   {\tt xmp\_init\_mpi} initializes the MPI execution environment.

\subsubsection*{Arguments}

   In {\XMPC}, the command-line arguments {\tt argc} and {\tt argv}
   should be given to {\tt xmp\_init\_mpi}.


\subsection{\tt xmp\_finalzie\_mpi}

\subsubsection*{Format}

\begin{tabular}{lll}
\verb![F]!&  {\tt }& {\tt xmp\_finalize\_mpi()}\\

\verb![C]!&  {\tt void}& {\tt xmp\_finalize\_mpi(void)}
\end{tabular}

\subsubsection*{Synopsis}

   {\tt xmp\_finalize\_mpi} terminates the MPI execution enviroment.

\subsubsection*{Arguments}

   none.

\subsection*{Example}
\begin{XCexample}
#include <stdio.h>
#include "mpi.h"
#include "xmp.h"

#pragma xmp nodes p(4)

int main(int argc, char *argv[]) {
  xmp_init_mpi(&argc, &argv)

  int rank, size;
  MPI_Comm_rank(MPI_COMM_WORLD, &rank);
  MPI_Comm_size(MPI_COMM_WORLD, &size);

#pragma xmp task on p(2:3)
{
  MPI_Comm comm = xmp_get_mpi_comm(); // get the MPI communicator of p(2:3)

  int rank, size;
  MPI_Comm_rank(comm, &rank);
  MPI_Comm_size(comm, &size);
}

  xmp_finalize_mpi();

  return 0;
}
\end{XCexample}


\section{{\OMP}}

   Thread-level parallelism is needed to program multi-core cluster system.
   Users can use some {\OMP} features to parallelize loops in
   thread level with the {\tt threads} clause of the {\tt loop} directive.
   No direct use of {\OMP} directives in {\XMP} code is allowed.

\subsection*{Syntax}

\begin{tabular}{ll}
\verb![F]! & \verb|!$xmp| {\tt loop} {\openb} \verb|(| {\it loop-index}
 {\openb}, {\it loop-index}{\closeb}... \verb|)| {\closeb} \\
 & \hspace{3cm}{\tt on} \{{\it nodes-ref} $\vert$ {\it template-ref}\}
     {\openb} {\it reduction-clause} {\closeb}...
     {\openb} {\it threads-clause} {\closeb} \\
 & {\it do-loops} \\
 & \\
\verb![C]! & \verb|#pragma xmp| {\tt loop} {\openb} \verb|(| {\it
     loop-index} {\openb}, {\it loop-index}{\closeb}... \verb|)|
     {\closeb} \\
 & \hspace{3cm}{\tt on} \{{\it nodes-ref} $\vert$ {\it template-ref}\}
     {\openb} {\it reduction-clause} {\closeb}...
     {\openb} {\it threads-clause} {\closeb} \\
 & {\it for-loops} \\
\end{tabular}

\vspace{0.3cm}

where {\it threads-clause} is:

\vspace{0.3cm}

\begin{tabular}{ll}
 & {\tt threads} {\openb} {\it omp-clause} {\closeb} \\
\end{tabular}

\vspace{0.3cm}

and {\it omp-clause} is one of:

\vspace{0.3cm}

\begin{tabular}{ll}
 & {\tt num\_threads(} {\it num-thread} {\tt )}\\
 & {\tt private(} {\it list} {\tt )}\\
 & {\tt firstprivate(} {\it list} {\tt )}\\
 & {\tt lastprivate(} {\it list} {\tt )}\\
\end{tabular}

\subsection*{Description}

   {\OMP} clauses such as {\tt num\_threads} can be specified in {\tt
   threads} clause.
   The {\XMP} compiler generates internally {\OMP} directives from the
   {\tt loop} directive and the {\tt threads} clause.
   Note that no {\tt reduction} need to be specified in the {\tt
   threads} clause because it is inherited from the {\tt reduction}
   clause in the {\tt loop} directive.

\subsection*{Example}

   This example calculates the total sum of an array.
   A {\tt threads} clause is given to the {\tt loop} directive to
   parallelize the loop statement in both process and
   thread level. 
   The {\tt reduction} clause in the {\tt loop} directive is also
   applied to the {\OMP} directive which is generated by the {\XMP}
   compiler.

\begin{XCexample}
#include <stdio.h>
#include "xmp.h"
#define N 1024

#pragma xmp nodes p(*)
#pragma xmp template t(0:N-1)
#pragma xmp distribute t(block) onto p
#pragma xmp align a[i] with t(i)

int main(void) {
  . . . // initialize a[]

  int sum = 0;
#pragma xmp loop on t(i) reduction(+:sum) threads num_threads(4)
  for (int i = 0; i < N; i++) {
    sum += a[i];
  }

  return 0;
}
\end{XCexample}


\chapter{Interface to Numerical Libraries}

   This chapter describes the XcalableMP interfaces to existing MPI
   parallel libraries, which is effective to achieve high productivity
   and performance ot {\XMP} programs.
   
\section{ScaLAPACK}

   This section shows a way how ScaLAPACK library routines are used in
   {\XMP} programs.

\begin{itemize}
\item The name of XcalableMP interface prefixes "{\tt XMP\_}" or "{\tt xmp\_}" 
      to ScaLAPACK library routine name.
\item Execution flow is the following:\\
      XMP programs $\to$ XMP interface $\to$ native ScaLAPACK library routines
\item If arguments of subroutines have array descriptor,
      we replace the argument with "{\tt xmp\_desc\_of}".
      Then we don't need array descriptor. 
\item For ScaLAPACK library routine having descriptor array as argument,
      the XcalableMP interface routine have BLACS context handle including 
      the descriptor array as new argument.
\item The blacs\_exit routine is unnecessary, because XcalableMP program executes process 
      corresponding to MPI\_Finalize routines.
\item A valid value of the argument "order" of BLACS routine "{\tt blacs\_gridinit}" is 
      only column-major. 
\end{itemize}

\subsection*{Example}
\begin{XFexample}
      program xmptdgesv

      double precision a(1000,1000)
      double precision b(1000)
      integer ipiv(2*1000,2)
!$xmp nodes p(2,2)
!$xmp nodes q(2)=p(1:2,1)
!$xmp template t(1000,1000)
!$xmp template t1(2*1000,2)
!$xmp template s(1000)
!$xmp distribute t(block,block) onto p
!$xmp distribute t1(block,block) onto p
!$xmp distribute s(block) onto q
!$xmp align a(i,j) with t(i,j)
!$xmp align ipiv(i,j) with t1(i,j)
!$xmp align b(i) with s(i)

      integer i,j,ictxt,myrow,mycol
      integer m=1000,n=1000,nprow=2,npcol=2
      integer icontxt=-1,iwhat=0
      integer nrhs=1,ia=1,ja=1,ib=1,jb=1,info
      character*1 order

      ...
      
      order="C"

      ...

      call blacs_get(icontxt,iwhat,ictxt)
      call blacs_gridinit(ictxt,order,nprow,npcol)
      call blacs_gridinfo(ictxt,nprow,npcol,myrow,mycol)

      ...

!$xmp loop (i,j) on t(i,j)
      do i=1,m
         do j=1,n
            a(i,j) = ...
         end do
      end do

      ...
 
!$xmp loop on s(i)
      do i=1,m
         b(i)= ...
      end do

      call xmp_pdgesv(n,nrhs,a,ia,ja,xmp_desc_of(a),ipiv,
                      b,ib,jb,xmp_desc_of(b),ictxt,info);

      ...

      call blacs_gridexit(ictxt);

      stop
      end
\end{XFexample}