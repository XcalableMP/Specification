\section{\Directive{nodes} Directive}

\subsubsection*{Synopsis}

%The {\tt nodes} directive declares a node array with a name, a shape,
%and some attributes.

The {\tt nodes} directive declares a named node array.

\Syntax{nodes}
\subsubsection*{Syntax}

\begin{tabular}{ll}
\verb![F]!&\verb|!$xmp| {\tt nodes} {\it nodes-decl} {\openb},
 {\it nodes-decl} {\closeb}...\\
& \\
\verb![C]!&\verb|#pragma xmp| {\tt nodes} {\it nodes-decl} {\openb},
 {\it nodes-decl} {\closeb}...\\
\end{tabular}

%\begin{tabular}{ll}
%\verb![F]!&\verb|!$xmp| {\tt nodes} {\it nodes-name} \verb|(|
%     {\it nodes-spec} {\openb}, {\it nodes-spec}
%     {\closeb}... \verb|)| \\
%\verb![F]!&\verb|!$xmp| {\tt nodes} {\it nodes-name} \verb|(|
%     {\it nodes-spec} {\openb}, {\it nodes-spec}
%     {\closeb}... \verb|)| {\tt = *}\\
%\verb![F]!&\verb|!$xmp| {\tt nodes} {\it nodes-name} \verb|(|
%     {\it nodes-spec} {\openb}, {\it nodes-spec}
%     {\closeb}... \verb|)| {\tt =} {\it nodes-ref}\\
%& \\
%\verb![C]!&\verb|#pragma xmp| {\tt nodes} {\it nodes-name}
%     \verb|(| {\it nodes-spec} {\openb}, {\it nodes-spec}
%     {\closeb}... \verb|)| \\
%\verb![C]!&\verb|#pragma xmp| {\tt nodes} {\it nodes-name}
%     \verb|(| {\it nodes-spec} {\openb}, {\it nodes-spec}
%     {\closeb}... \verb|)| {\tt = *} \\
%\verb![C]!&\verb|#pragma xmp| {\tt nodes} {\it nodes-name}
%     \verb|(| {\it nodes-spec} {\openb}, {\it nodes-spec}
%     {\closeb}... \verb|)| {\tt =} {\it nodes-ref} \\
%\end{tabular}

\vspace{0.3cm}

where {\it nodes-decl} is one of:

\vspace{0.3cm}

\begin{tabular}{ll}
 \hspace{0.5cm} & {\it nodes-name} \verb|(| {\it nodes-spec} {\openb},
 {\it nodes-spec} {\closeb}... \verb|)| \\
 \hspace{0.5cm} & {\it nodes-name} \verb|(| {\it nodes-spec} {\openb},
     {\it nodes-spec} {\closeb}... \verb|)| {\tt = *} \\
 \hspace{0.5cm} & {\it nodes-name} \verb|(| {\it nodes-spec} {\openb},
     {\it nodes-spec} {\closeb}... \verb|)| {\tt =} {\it nodes-ref} \\
\end{tabular}

\vspace{0.3cm}

and {\it nodes-spec} must be one of:

\vspace{0.3cm}

\begin{tabular}{ll}
 \hspace{0.5cm} & {\it int-expr} \\
 \hspace{0.5cm} & {\tt *} \\
\end{tabular}


\subsubsection*{Description}

The {\tt nodes} directive declares a node array that corresponds to a
node set.

The first form of the {\tt nodes} directive is used to declare a node
array that corresponds to the entire node set.
%The second and third forms declare a new node array with a name, a
%dimension, and a size in order to reference a set of nodes.
The second form is used to declare a node array that corresponds to the
executing node set.
%The ``{\tt *}'' symbol specifies the current executing node set.
The third form is used to declare a node array that corresponds to the
node set specified by {\it nodes-ref}.

%If {\it map-type} is specified as {\tt regular}, then the order of nodes in
%the node array follows that of the {\Fort} array. Therefore, in the first
%form, the node number is used to order nodes in the node array with
%{\Fort} array ordering. In the second and third forms, the nodes are
%ordered according to the sequence association with referenced nodes.  
%
%If no {\it map-type} is specified, then the ordering nodes in the node array are
%system dependent. It is desirable to order the nodes in order to make use of
%the network topology for efficient communication. 

If {\it node-size} in the last dimension is ``{\tt *}'', then the size
of the node array is automatically adjusted according to the total size
of the entire node set in the first form, the executing node set in the
second form, or the referenced node set in the third form.

\subsubsection*{Restrictions}

\begin{itemize}
%\item {\it nodes-name} is an identifier in class (1) and must not
%  conflict with other names in class (1).
\item {\it nodes-name} must not conflict with any other local name in
      the same scoping unit.
\item \verb![F]! The second form cannot be used in either the main
      program or a module.
\item {\it nodes-spec} can be ``{\tt *}'' only in the last dimension.
\item {\it nodes-ref} must not reference {\it nodes-name} either
      directly or indirectly.
\item If no {\it nodes-spec} is ``{\tt *}'', then the product
      of all {\it nodes-spec} must be equal to the total size of the
      entire node set in the first form, the executing node set in the
      second form, or the referenced node set in the third form.
%
%      The referenced node set must consist of all nodes in the first form,
%      the executing node set in the second form, and the node set
%      referenced by {\it nodes-ref} in the third form. 
\item {\it nodes-subscript} in {\it nodes-ref} must not be ``{\tt *}''.
\end{itemize}

\subsubsection*{Examples}
\Example{nodes}

The following are examples of the first and the third forms appeared in
the main program. Since the node array {\tt p}, which corresponds to the
entire node set, is declared to be of size 16, this program must be
executed by 16 nodes.

%Since the declaration of node array {\tt p} specifies
%16 nodes as its size, this program must be executed with 16 nodes.

%Since {\tt regular} is not specified, it is not guaranteed that {\tt
%Ar(1)} and {\tt p(3)} are the same node, and the node number of {\tt
%z(1,1)} is 1.

\vspace{0.5cm}

\begin{minipage}{0.45\hsize}
\begin{center}
\begin{Fexample}
      program main
!$xmp nodes p(16)
!$xmp nodes q(4,*)
!$xmp nodes r(8)=p(3:10)
!$xmp nodes z(2,3)=(1:6)
      ...       
      end program 
\end{Fexample}
\end{center}
\end{minipage}
%
\begin{minipage}{0.45\hsize}
\begin{center}
\begin{CexampleR}
int main() {
#pragma xmp nodes p(16)
#pragma xmp nodes q(4,*)
#pragma xmp nodes r(8)=p(3:10)
#pragma xmp nodes z(2,3)=(1:6)
    ...
}
\end{CexampleR}
\end{center}
\end{minipage}

\vspace{0.5cm}

%Example using the regular option. Since node array {\tt p} is declared
%without the regular option, it is not guaranteed that {\tt p(1), p(2)}
%have the node number 1, 2, ... and so on. The node array {\tt q} with the 
%regular option has the order in which
%{\tt q(1,1), q(2,1), q(3,1), q(4,1), q(1,2), ...} have node numbers
%1,2,3,4,5, ... In node array z with the regular option,
%{\tt z(1,1), z(2,1), z(1,2), z(2,2), z(1,3), z(2,3), ...} have the
%node numbers 1, 2, 3, 4, 5, 6, ...
%
%\begin{Fexample}
%      program main
%!$xmp nodes p(16)
%!$xmp nodes(regular) q(4,*)
%!$xmp nodes(regular) r(8)=p(3:10)
%!$xmp nodes(regular) z(2,3)=(1:6)
%      ...
%      end program
%\end{Fexample}

The following is an example of a node declaration in a procedure.
Since {\tt p} is declared in the second form to be of size 16 and
correspond to the executing node set, the invocation of the {\tt foo}
function must be executed by 16 nodes.
%
The node array {\tt q} is declared in the first form and corresponds to
the entire node set. The node array {\tt r} is declared as a subset of
{\tt p}, and {\tt x} as a subset of {\tt q}.

%The declaration for the node array {\tt q} of the first form
%declares the node array for the entire node set. The node array {\tt r}
%is a subset of {\tt p}, and the node array of {\tt x} is a subset of
%{\tt q}.

\begin{Fexample}
      function foo()
!$xmp nodes p(16)=*
!$xmp nodes q(4,*)
!$xmp nodes r(8)=p(3:10)
!$xmp nodes x(2,3)=q(1:2,1:3)
      ...
      end function
\end{Fexample}


\subsection{Node Reference}

\subsubsection*{Synopsis}

The \Term{node reference} expression is used to reference a subset of a
node set.

\subsubsection*{Syntax}
\Syntax{node reference}

A node reference {\it nodes-ref} is specified by either node number
or the name of a node array.

\begin{center}
\begin{tabular}{lll}
{\it nodes-ref} & {\bf is} & \verb|(| {\it nodes-subscript} \verb|)|\\ 
                & {\bf or} & {\it nodes-name} {\openb}\verb|(| {\it nodes-subscript}
	 {\openb}, {\it nodes-subscript} {\closeb}... \verb|)|{\closeb} \\
\end{tabular}
\end{center}
%
\vspace{0.3cm}
%
where {\it nodes-subscript} must be one of:

\hspace{\hsize}

\begin{tabular}{ll}
 \hspace{0.5cm} & {\it int-expr} \\
 \hspace{0.5cm} & {\it triplet} \\
 \hspace{0.5cm} & {\tt *} \\
\end{tabular}

%\begin{center}
%\begin{tabular}{ll}
%{\it nodes-ref} & {\it node-number-ref} $\vert$ {\it named-nodes-ref} \\
%{\it node-number-ref} & {\it node-number} $\vert$ ([{\it
%     node-number}]:[{\it node-number}][:{\it int-expr}]) \\
%& {\it node-number} is a positive number. \\
%{\it named-nodes-ref} & {\it nodes-name} [ ( {\it nodes-subscript}
%[,  ...] ) ] \\
%{\it nodes-subscript} & {\it int-expr} $\vert$ {\it triplet} $\vert$ {\tt *} \\
%\end{tabular}
%\end{center}

\subsubsection*{Description}

Node reference by node number represents one node specified by the node 
number, or a node set specified by a triplet that represents a set of
node numbers.

Node reference by name represents a node set specified by the name of a
node array or its subarray.

%The subscript of the subarray of a node array must be either an integer,
%a triplet, or ``{\tt *}''. The notation of the subarray using a triplet
%in the subscript is the same as that in {\Fort}. 

Specifically, the ``{\tt *}'' symbol apperared as {\it nodes-subscript}
in a dimension of {\it nodes-ref} is interpreted by each node at runtime
as its position (coordinate) in the dimension of the referenced node
array.
%The ``{\tt *}'' symbol in {\it nodes-subscript} in a subarray of a
%node array specifies a subscript associated with the executing node in
%the node array of the executing node set.
%
Thus, a node reference {\tt p($s_1$, ..., $s_{k-1}$, *, $s_{k+1}$, ...,
$s_n$)} is interpreted as {\tt p($s_1$, ..., $s_{k-1}$, $j_k$,
$s_{k+1}$, ..., $s_n$)} on the node {\tt p($j_1$, ..., $j_{k-1}$, $j_k$,
$j_{k+1}$, ..., $j_n$)}.

%Thus, the following node is referenced by name with the $k$-th subscript
%``{\tt *}'':
%
%\begin{center}
%{\tt p($s_1$, ..., $s_{k-1}$, *, $s_{k+1}$, ..., $s_n$)} 
%\end{center}
%where, with the exception of $s_k$, subscripts $s_i$ must not be ``{\tt *}'', 
%is evaluated at the node 
%\begin{center}
%{\tt p($j_1$, ..., $j_{k-1}$, $j_k$, $j_{k+1}$, ..., $j_n$)} 
%\end{center}
%where $j_i$ is an integer, in
%\begin{center}
%{\tt p($s_1$, ..., $s_{k-1}$, $j_k$, $s_{k+1}$, ..., $s_n$)}.
%\end{center}

Note that ``{\tt *}'' can be used only as the node reference in
the {\tt on} clause of some executable directives.

%This node reference composes the node set using nodes with the $k$-th
%subscript $j_k$. The same rule is applied even if more than two
%subscripts are ``{\tt *}''. This notation can be used only in the node
%reference of the on clause in executable directives. 

\subsubsection*{Examples}

Assume that {\tt p} is the name of a node array and that {\tt m} is an
integer variable.

\begin{itemize}
\item As a target node array in the {\tt distribute} directive

\Example{distribute}
\begin{tabular}{l}
\verb|!$xmp distribute a(block) onto p| \\
\end{tabular}%$

\item To specify a node set to which the declared node array corresponds
      in the second form of the {\tt nodes} directive

\Example{nodes}
\begin{tabular}{l}
\verb|!$xmp nodes r(2,2,4) = p(1:4,1:4)| \\
\verb|!$xmp nodes r(2,2,4) = (1:16)| \\
\end{tabular}

\item To specify a node array that corresponds to the executing node set
      of a task in the {\tt task} directive

\Example{task}
\begin{tabular}{l}
\verb|!$xmp task on p(1:4,1:4)| \\
\verb|!$xmp task on (1:16)| \\
\verb|!$xmp task on p(:,*)| \\
\verb|!$xmp task on (m)| \\
\end{tabular}

\item To specify a node array with which the iterations of the loop are
      aligned in the {\tt loop} directive

%In the {\tt loop} directive, sets of executing nodes are specified
%      for the iterations.

\Example{loop}
\begin{tabular}{l}
\verb|!$xmp loop (i) on p(lb(i):lb(i+1)-1)| \\
\end{tabular}%$

\item To specify a node array that corresponds to the executing node set
      in the {\tt barrier} and the {\tt reduction} directive

%In {\tt barrier} directive and the {\tt reduction} directive,
%executing nodes are specified. 

\Example{barrier}
\Example{reduction}
\begin{tabular}{l}
\verb|!$xmp barrier on p(5:8)| \\
\verb|!$xmp reduction (+:a) on p(*,:)| \\
\end{tabular}

\item To specify the source node and the node array that corresponds to
      the executing node set in the {\tt bcast} directive 

%In the {\tt bcast} directive, a source node and executing nodes are specified.

\Example{bcast}
\begin{tabular}{l}
\verb|!$xmp bcast b from p(k) on p(:)| \\
\end{tabular}
\end{itemize}

%\subsubsection*{Examples}
%\Example{nodes}
%\Example{tasks}
%\Example{task}
%\Example{end task}
%\Example{end tasks}
%
%\begin{minipage}{0.45\hsize}
%\begin{center}
%\begin{Fexample}
%      subroutine caller
%!$xmp nodes p(1000)
%      real a(100,100)
%      ...
%!$xmp tasks
%!$xmp  task on p(1:500)
%        call task1(a)
%!$xmp  end task
%!$xmp  task on p(501:800)
%        call task1(a)
%!$xmp  end task
%!$xmp  task on p(801:1000)
%        call task1(a)
%!$xmp  end task
%!$xmp end tasks
%      ...
%      end do
%\end{Fexample}
%\end{center}
%\end{minipage}
%\begin{minipage}{0.45\hsize}
%\begin{center}
%\begin{FexampleR}
%      subroutine task1(a)
%      ...
%!$xmp nodes q(*)
%      real a(100,100)
%      ...
%      end subroutine
%\end{FexampleR}
%\end{center}
%\end{minipage}
%\vspace{1cm}


\subsection{Correspondence between Node Arrays}

If one node array and the other have the same shape and correspond to
the same node set, an element of the one node array and the element of
the other are assigned to the same node;
%
otherwise, correspondence between any two node arrays is not specified.
