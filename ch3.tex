\chapter{Base Language Extensions in {\XMPC}}
\label{chap:Base Language Extensions in {\XMPC}}

%This chapter describes base language extensions in {\XMPC} that is other
%than coarrays described in Chapter \ref{chap:Support for the Local-view
%Programming}.

This chapter describes base language extensions in {\XMPC} that are not
described in any other chapters.

\section{Array Section Notation}
\label{173437_31Oct14}

\subsubsection*{Synopsis}

The array section notation is a notation to describe a part of an array, 
which is adapted in Fortran.

\subsubsection*{Syntax}
\index{array section in XMP/C}
\index{Syntax!array section in XMP/C}

\begin{tabular}{llll}
\verb![C]! & {\it array-section} & {\bf is} & {\it array-name}{\tt [} \{
 {\it triplet} $\vert$ {\it int-expr} \} {\tt ]}...
\end{tabular}

\vspace{0.5cm}

%where {\it triplet} must be one of:
where {\it triplet} is:

\vspace{0.3cm}

\begin{tabular}{ll}
 \hspace{0.5cm} & {\openb}{\it base}{\closeb} {\tt :}
  {\openb}{\it length}{\closeb} {\openb}{\tt :} {\it step}{\closeb}\\
% \hspace{0.5cm} & {\tt :} \\
\end{tabular}

\subsubsection*{Description}

In {\XMPC}, the base language C is extended so that a part of an array,
i.e., an array section, can be put in an {\it array assignment
statement}, which is described in \ref{sec:Array assignment statements
in C}, and some {\XMP} constructs. An array section is built from a
subset of the elements of an array, which is specified by this notation
including at least one {\it triplet}.

When {\it step} is positive, the {\it triplet} specifies a set of
subscripts that is a regularly spaced integer sequence of length {\it
length} beginning with {\it base} and proceeding in increments of {\it
step} up to the largest.
%
When {\it step} is negative, the {\it triplet} specifies a set of
subscripts that is a regularly spaced integer sequence of length {\it
length} beginning with {\it base} and proceeding in increments of {\it
step} down to the smallest.

%{\it lower-bound} and {\it upper-bound} specify an index range of array
%elements. {\it lower-bound} and/or {\it upper-bound} can be omitted, in
%which case they default to the lower and/or the upper bound of the
%array. Therefore, {\tt A[:]} is a section containing the whole of {\tt
%A}.
%%
%If {\it step} is specified, then the elements of an array section are
%every ``step''-th element in the range specified by {\it lower-bound}
%and/or {\it upper-bound}. For example, {\tt B[1:10:3]} is an array 
%section of size 4 containing every third element of {\tt B} with indices
%between 1 and 10 (i.e., indices 1, 4, 7, 10).

When {\it base} is omitted, it is assumed to be 0. When {\it length}
is omitted, it is assumed to account for the remainder of the array
dimension. When {\it step} is omitted, it is assumed to be 1.


% When {\it step} is omitted, it is assumed to be ``1''.
% %
% When all of {\it base}, {\it length} and {\it step} is omitted, it is
% assumed that {\it base} is ``0'', {\it length} is the size of the
% dimension of the array, and {\it step} is ``1''.

An array section can be considered as a virtual array containing the set
of elements from the original array, which is determined by all possible
subscript lists that are specified by the sequence of {\it triplets} or
{\it int-expr}'s in square brackets.

\subsubsection*{Restrictions}

\begin{itemize}
 \item \verb![C]! Each of {\it base}, {\it length}, and {\it step} must
       be an integer expression.
% \item \verb![C]! When {\it step} is positive, {\it lower-bound} must be
%       greater than or equal to the lower bound and {\it upper-bound}
%       must be smaller than or equal to the upper bound of the dimension
%       of the array specified by {\it array-name}.
% \item \verb![C]! When {\it step} is negative, {\it lower-bound} must be
%       smaller than or equal to the upper bound and {\it upper-bound}
%       must be greater than or equal to the lower bound of the dimension
%       of the array specified by {\it array-name}.
 \item \verb![C]! {\it length} must be greater than zero.
 \item \verb![C]! {\it step} must not be zero.
\end{itemize}

\subsubsection*{Example}
\index{Example!array section in XMP/C}

Assuming that an array {\tt A} is declared by the following statement,

\vspace{0.3cm}

\begin{tabular}{ll}
\hspace{0.5cm} & {\tt int A[100];} \\
\end{tabular}

\vspace{0.3cm}

\hspace{-0.55cm}some array sections can be specified as follows:

\vspace{0.3cm}

\begin{tabular}{lll}
\hspace{0.5cm} & {\tt A[10:10]} & array section of 10 elements from {\tt
 A[10]} to {\tt A[19]} \\
 & {\tt A[10:]} & array section of 90 elements from
		  {\tt A[10]} to {\tt A[99]}\\
 & {\tt A[:10]} & array section of 10 elements from {\tt A[0]} to {\tt
	 A[9]} \\
 & {\tt A[10:5:2]} & array section of 5 elements from {\tt A[10]} to
	 {\tt A[18]} by step 2 \\
 & {\tt A[:]} & the whole of {\tt A} \\
\end{tabular}

\section{Array Assignment Statement}
\label{sec:Array assignment statements in C}

\subsubsection*{Synopsis}

An array assignment statement copies a value into each element of
an array section.

%Array-valued expressions can be used by array section in assignments.

\subsubsection*{Syntax}
\index{array assignment in XMP/C}
\index{Syntax!array assignment in XMP/C}

\begin{tabular}{ll}
\verb![C]! & {\it array-section} {\openb}{\tt :}{\tt [}{\it int-expr}{\tt
 ]}...{\closeb} {\tt =} {\it expression}{\tt ;}\\
\end{tabular}

% \begin{tabular}{ll}
% \verb![C]! & {\it array-section} {\openb}{\tt :}{\tt [}{\it int-expr}{\tt
%  ]}...{\closeb} {\tt =} \{ {\it variable} {\openb}{\tt :}{\tt [}{\it
%  int-expr}{\tt ]}...{\closeb} $\vert$ {\it int-expr \}}{\tt ;} \\
% \end{tabular}

\subsubsection*{Description}


The value of each element of the result of the right-hand side expression is
assigned to the corresponding element of the array section on the
left-hand side.
%
When an operator or an elemental function (see section
\ref{094142_25Sep13}) is applied to array sections in the right-hand side
expression, it is evaluated to an array section that has the same shape
as that of the operands or arguments, and each element of which is the
result of the operator or function applied to the corresponding element
of the operands or arguments. A scalar object is assumed to be an array
section that has the same shape as that of the array section(s), and
where each element has its value.

% When the right-hand side is an array section, the value of each element of it is
% assigned to the corresponding element of the left-hand side array 
% section. When the right-hand side is an integer expression, its value is assigned to
% each element of the left-hand side array section.
% The right-hand side and/or the left-hand side data can have cosubscripts.

Note that an array assignment is a statement, and therefore cannot
appear as an expression in any other statements.

\subsubsection*{Restrictions}

\begin{itemize}
 \item \verb![C]! any array section appearing in the right-hand side expression and
	   the left-hand side must have the same shape, i.e., the same number of
	   dimensions and size of each dimension.
 % \item \verb![C]! When the right-hand side is an array section, the left-hand side and the right-hand side
 %       must have the same shape, i.e., the same number of dimensions and
 %       size of each dimension.
 \item \verb![C]! If {\it array-section} on the left-hand side is followed by
       ``{\tt :}{\tt [}{\it int-expr}{\tt ]}...'', it must be a coarray.
 % \item \verb![C]! If {\it variable} on the right-hand side is followed by
 %       ``{\tt :}{\tt [}{\it int-expr}{\tt ]}...'', it must be a coarray.
\end{itemize}

\subsubsection*{Examples}
\index{Example!array assignment in XMP/C}

An array assignment statement in the fourth line copies the elements
{\tt B[0]} through {\tt B[4]} into the elements {\tt A[5]} through {\tt
A[9]}.

\hspace{\hsize}
\begin{XCexample}
int A[10];
int B[5];
    ...
A[5:5] = B[0:5]; 
\end{XCexample}


\section{Built-in Functions for Array Section}
\index{built-in functions of XMP/C}

Some built-in functions are defined that can accept one or more array
sections as arguments. In addition, some of them are array-valued.
%
Such array-valued functions can appear in the right-hand side of an
array assignment statement, and should be preceded by the {\tt array}
directive if the array section is distributed.

All of the built-in functions for array sections are described in
Sections \ref{094142_25Sep13} and \ref{112125_19Sep13}.


\section{Pointer to Global Data}
\label{sec:pointer to global data}

\subsection{Name of Global Array}

The name of a global array is considered to represent an abstract entity
in the {\XMP} language. It is not interpreted as the pointer to the array,
while the name of a local array is.

However, the name of a global array that appears in an expression is
evaluated to the base address of its local section on each node. The
pointer can be operated on each node as if it were a normal (local)
pointer.

\subsection{Address-of Operator}
\index{address-of operator}

The result of the address-of operator (``{\tt \&}'') applied to an
element of a global array is the pointer to the corresponding element of
its local section. Note that the value of the result pointer is defined
only on the node that owns the element. The pointer can be operated on
the node as if it were a normal (local) pointer.

As a result, for a global array {\tt a}, {\tt a} and {\tt \&a[0]} are
not always evaluated to the same value.

\section{Dynamic Allocation of Global Data}
\label{sec:Dynamic Allocation of Global Data in C}

In {\XMPC}, it is possible to allocate global arrays at runtime.
%
Such an allocation is done by performing the following steps.
%
\begin{enumerate}
 \item Declare a pointer to an object of the type of the global array to
       be allocated.
 \item Align the pointer with a template as if it were an array.
 \item Allocate a block of memory of the global size using the {\tt xmp\_malloc}
       library procedure, and assign the return value to the
       pointer on each node.
\end{enumerate}

\index{Example!dynamic allocation in XMP/C}
\Intrinsic{xmp\_malloc}
\Example{template\_fix}
\Example{xmp\_malloc}
\Example{xmp\_desc\_of}
\begin{XCexample}
#pragma nodes p(NP1,NP2)
#pragma xmp template t(:,:)
#pragma xmp distribute t(block,block) onto p

float (*pa)[N2];
#pragma xmp align pa[i][j] with t(i,j)
#pragma xmp template_fix t(0:N1-1,0:N2-1)

pa = (float (*)[N2])xmp_malloc(xmp_desc_of(pa), N1, N2);
\end{XCexample}

\section{Descriptor-of Operator}
\label{sec:Descriptor of Global Data in C}
\index{xmp\_desc\_of@{\tt xmp\_desc\_of}}

%When distribution of Global data is defined, query function which have 
%operator {\tt xmp\_desc\_of} as argument can return some descriptor
%information. 

The \Term{descriptor-of operator} (``{\tt xmp\_desc\_of}'') is
introduced as a built-in operator in {\XMPC}.

The result of the descriptor-of operator applied to {\XMP} entities such
as node arrays, templates, and global arrays is their {\it
\Term{descriptor}}, which can be used in various ways, including as an
argument of some inquiry procedures. The type of the result, {\tt
xmp\_desc\_t}, is implementation-defined, and is defined in the {\tt
xmp.h} header file in {\XMPC}.

For the {\tt xmp\_desc\_of} intrinsic function in {\XMPF}, refer to
section \ref{subsec: xmp_desc_of}.

%For details of the {\tt xmp\_desc\_of} library procedure, refer to
%Chapter \ref{chap:Intrinsic and library procedures}.
