\chapter{DRAFT: Coarray Features}
\label{chap:DRAFT: Coarray Features}

%-------------------------------------------------------%
% ENVIRONMENT 
%-------------------------------------------------------%
\newcommand{\onlyF}{{\tt [F]}} 
\newcommand{\onlyC}{{\tt [C]}} 

\newenvironment{Constraints}{\subsubsection*{Constraints}
 \begin{enumerate}}{\end{enumerate}}
\newenvironment{Constraints F}{\subsubsection*{Constraints {\onlyF}}
 \begin{enumerate}}{\end{enumerate}}
\newenvironment{Constraints C}{\subsubsection*{Constraints {\onlyC}}
 \begin{enumerate}}{\end{enumerate}}

\newenvironment{Restriction}{\subsubsection*{Restriction}
 \begin{itemize}}{\end{itemize}}

\newcommand{\NEW}[1]{\mytextcolor{red}{#1}}

%-------------------------------------------------------%
% PREFACE
%-------------------------------------------------------%

\framebox[0.9\textwidth][c]{
\rule[-5mm]{0mm}{10mm}
This chapter is a proposal document to be added before 
Section~\ref{chap:Support for the Local-view Programming}.
}

\bigskip


%-- COARRAY AS THE LOCAL-VIEW
For the local-view programming, {\XMP} supports the coarray features
as a part of the language specifications.
%-- SPECIFICATION RANGE OF THE XMP COARRAY
{\XMPF} contains all coarray features defined in the standard Fortran~2008
(ISO/IEC 1539-1:2010) with few incompatibility described in 
Section~\ref{sec:Compatibility with the Fortran Standard}
and includes some important intrinsic procedures defined in 
the standard Fortran~2015.
Also {\XMPC} contains the coarray features which was designed based on
the ones of {\XMPF}.


%-------------------------------------------------------%
\section{Introduction for Coarrays}
\label{sec:Introduction for Coarrays}
%-------------------------------------------------------%

%=======================================
\subsubsection*{Image and image index}
%=======================================
The local-view programming model is a Single Program Multiple Data (SPMD) model.
Each replication of the program is called an {\bf image}.
Every image has a different {\bf image index}, which is an integer number 
between one and the number of images.
The number of images is not determined until the program execution.

In {\XMP}, a virtual array of the whole images with any number of dimensions 
is called an {\bf image array}. 
Each array element of an image array is corresponding to an image index
in the array element order of Fortran.
The extent of the final (outermost) dimension is not determined until
the program execution because it depends on the number of images.

The images are mapped one-by-one to the execution nodes.
The correspondence between images and nodes is defined in 
Chapter~\ref{chap:Interoperability of Global- and Local-views}.
Inquire functions about the image index and the number of images 
are described in Section~\ref{sec:Intrinsic Procedures}.


%=======================================
\subsubsection*{Coarray}
%=======================================
%-- what is coarray
A {\bf coarray} is an object that has a corresponding image array.
Unlike a usual object (non-coarray), a coarray is allowed to be
referred from other images. 
Each image has its own coarrays and can reference and define 
coarrays on all images each other.

%-- coshape
The shape of the image array corresponding to a coarray is 
called a {\bf coshape} of the coarray.
A coshape is specified with a {\it coarray-spec} in the declaration of the 
coarray variable or the coarray pointer (described later).
%-- corank, cobounds, coextend
The number of dimensions of a coshape is called a {\bf corank}.
For each dimension of a coshape, the lower and upper bounds and the extent
are called {\bf lower and upper cobounds} and {\bf coextent}, respectively.


%-- [F] category of coarray
{\onlyF} A {\it coarray-spec} can appear in a type declaration statement 
and in a component definition statement.
Entities declared with a {\it coarray-spec} are categorized into three:
\begin{itemize}
\item A {\bf static coarray} is a coarray that is not a dummy argument and is
non-allocatable. 
It must have the SAVE attribute explicitly or implicitly.
\item A {\bf dummy coarray} is a coarray that is a dummy argument and is 
non-allocatable.
The actual argument corresponding to a dummy coarray must be a static coarray,
a dummy coarray, an allocatable coarray, or a subobject of them.
\item An {\bf Allocatable coarray} is a coarray that is allocatable.
An ultimate component\footnote
{A component is an ultimate component of the structure if it is of 
a basic type or is allocatable or a pointer.
An ultimate component of a component is an ultimate component of the structure,
recursively.}
of a non-coarray structure can also be an allocatable coarray.
\end{itemize}

%-- [C] category of coarray
{\onlyC} A {\it coarray-spec} can appear in a {\it declaration} and 
in a {\it parameter-declaration}. 
Entities declared with a {\it coarray-spec} are categorized into three:
\begin{itemize}
\item A {\bf static coarray} is a coarray that is not a dummy argument and
is of a basic, structure or array type. If it is an array, the array 
element must be of a basic, structure or array type.
It must have the {\tt static} or {\tt extern} storage class.
\item A {\bf dummy coarray} is a coarray that is a dummy argument and
is of a basic, structure or array type. If it is an array, the array 
element must be of a basic, structure or array type.
The actual argument corresponding to a dummy coarray must be the name
of a static corray, dummy coarray, or a coarray pointer.
\item A {\bf coarray pointer} is a pointer to a coarray called a 
{\bf target coarray}.
An ultimate component of a non-coarray structure can also be a
coarray pointer.
\end{itemize}

%-- life span of coshape
A static coarray is allocated previously and is static during the program 
execution. 
The coshape of a static coarray is explicitly specified in the declaration of the 
variable.
%
An allocatable coarray is dynamically allocated and deallocated at
the ALLOCATE and DEALLOCATE statements. 
A coarray pointer is dynamically allocated and freed by the intrinsic 
functions. 
The coshapes of them are determined at the allocation time and are retained 
until the deallocation/freeing time without regard to the scoping units.
%
A dummy coarray is an allocated object. 
The coshape of a dummy coarray is explicitly re-specified in the declaration of the
variable. Even if the corresponding actual argument is an allocatable coarrray or
a coarray pointer, the specification of the coshape is valid during the execution
in the scope.
%


\subsubsection*{Coarray Container}

A non-coarray structure object can have ultimate (leaf) components as 
allocatable coarrays (in {\XMPF}) or as coarray pointers (in {\XMPC}).
The strucure object is called a {\bf coarray container} in {\XMP}.

{\onlyF} A coarray container must be a scalar, may not be a pointer
or an allocatable, may not be a coarray, and may not be a function result.

{\onlyC} A coarray container may not be a coarray and may not be a function
result.


\subsubsection*{Cosubscript}
A coarray on the different image can be referenced and defined
by referring the coarray with {\bf cosubscripts}, which
is an array element of an image array.
See Section~\ref{sec:Allocation and Deallocation of Coarrays} for the detail.



%-------------------------------------------------------%
\section{Declaration of Coarrays}
\label{sec:Declaration of Coarrays}
%-------------------------------------------------------%
%**************************************************
\subsubsection*{Static and allocatable coarrays}
A coarray is either a \NEW{\bf static coarray} or an \NEW{\bf allocatable coarray}.
%
A static coarray is declarad with an explicit specification of the coshape.
If it is a dummy argument, the coshape is redefined by the specification during 
the execution in the scope.
%
An allocatable coarray is declared with an deferred specification of the coshape.
which are determined when the data object is allocated.
If it is a dummy argument, the coshape is inherited from the actural argument.

{\onlyF} 
A static coarray does not have the ALLOCATABLE attribute.
A procedure-local static coarray must have the SAVE attribute 
unless it is a dummy argument.
An allocatable coarray has the ALLOCATABLE attribute.

{\onlyC} 
A static coarray is of a basic type, a {\tt struct} type, or an array.
An allocatable coarray is not a named variable and the target of a coarray pointer.
Intrinsic function {\tt xmp\_coarray\_malloc} allocates an allocatable coarray,
determines the coextents, and returns the address to be assigned to the coarray pointer.

%**************************************************



%============================================
\subsection{Declaration statement of coarray}
\label{sec:Declaration statement of coarray}
%============================================

\subsubsection*{Synopsis}
Declarations of a variable are extended to add the codimension attribute.

{\onlyF}
The {\it type-declaration-stmt} is extended with {\it coarray-spec}.
An {\it object-name} declared with a {\it coarray-spec} is a coarray.

{\onlyC} 
the {\it init-declarator} of the {\it declaration} and 
the {\it parameter-declaration} are extended with a {\it coarray-spec}.
If the {\it declarator} of them does not start with {\tt *},
the {\it identifier} declared with a {\it coarray-spec} is a coarray.
Else, 
the {\it identifier} declared with a {\it coarray-spec} is a coarray pointer
that may point an unnamed coarray.


\subsubsection*{Syntax \onlyF}
{\it coarray-spec} is adopted in the type declaration statement, as shown below.
As a kind of {\it attr-spec}, the CODIMENSION attribute specifier is added.
Besides, {\it entry-decl} is extended with {\it coarray-spec}.
Underlined parts are addition to the Fortran 90 standard.

\begin{center}
 \begin{tabular}{lll}
  {\it type-declaration-stmt} &  {\bf is} & 
  {\it declaration-type-spec} 
  {\openb}\/{\openb}\/, {\it attr-spec}\/{\closeb}\/{\tt ...} {\tt ::} {\closeb}\/
  {\it entity-decl-list}\\
  \\
  {\it attr-spec} & {\bf is} & {\tt ...}\\
  & {\bf or} & {\tt ...}\\
  & \multicolumn{2}{l}{\tt ...}\\
  & \underline{\bf or} & 
    \underline{{\tt CODIMENSION} {\tt [} {\it coarray-spec} {\tt ]}}\\
  \\
  {\it entity-decl} & {\bf is} & {\it object-name}
  {\openb}\/{\tt (} {\it array-spec}\/ {\tt )}{\closeb}\/
  \underline{{\openb}\/{\tt [} {\it coarray-spec}\/ {\tt ]}{\closeb}\/} ~~{\bsquare}
  \\
  \multicolumn{3}{r}{\hfill{\bsquare}~~
  {\openb}\/{\tt *} {\it char-length} {\closeb}\/
  {\openb}\/{\it initialization} {\closeb}}
  \\
  & {\bf or} & {\it function-name} {\openb}\/{\tt *} {\it char-length} {\closeb}
 \end{tabular}
\end{center}
%
{\it coarray-spec} is defined as follows:
%
\begin{center}
 \begin{tabular}{lll}
  {\it coarray-spec} & {\bf is} & {\it explicit-coshape-spec} \\
                     & {\bf or} & {\it deferred-coshape-spec}
 \end{tabular}
\end{center}
%
{\it explicit-coshape-spec} and {\it deferred-coshape-spec} is defined in 
Section~\ref{sec:Static coarray} and
Section~\ref{sec:Allocatable coarray}, respectively.


\subsubsection*{Syntax \onlyC}
{\it coarray-spec} is adopted in {\it declaration},
as an underlined part shown below:
%
\begin{center}
 \begin{tabular}{lll}
  {\it declaration} & {\bf is} & {\it declaration-specifiers} 
    {\openb}\/{\it init-declarator-list}\/{\closeb} {\tt ;} \\
  {\it init-declarator} & {\bf is} & {\it declarator}
    \underline{{\openb}\/{\tt :} {\it coarray-spec} {\closeb}}
    {\openb} {\tt =} {\it initializer} {\closeb} \\
  \\
  {\it parameter-declaration} & {\bf is} & {\it declaration-specifiers}
    {\it declarator}\/
    \underline{{\openb}\/{\tt :} {\it coarray-spec} {\closeb}}\\
  & {\bf or} & {\it declaration-specifiers}
    {\openb} {\it abstract-declarator} {\closeb}\\
 \\
  {\it declarator} & {\bf is} &
    {\openb} {\it pointer} {\closeb} {\it direct-declarator} \\
  \\
  {\it pointer} & {\bf is} &
    {\tt *} {\openb} {\it type-qualifier} {\closeb}\/{\tt ...}\\
  & {\bf or} &
    {\tt *} {\openb} {\it type-qualifier} {\closeb}\/{\tt ...} {\it pointer}\\
 \end{tabular}
\end{center}
%
\begin{center}
 \begin{tabular}{lll}
  {\it direct-declarator} & {\bf is} & {\it identifier} \\
  & {\bf or} & {\tt (} {\it declarator}\/ {\tt )} \\
  & {\bf or} & {\it direct-declarator} 
      {\tt [} 
         {\openb}\,{\it type-qualifier}\,{\closeb}\/{\tt ...}
         {\openb}\,{\it assignment-expression}\,{\closeb}\/
      {\tt ]} \\
  & {\bf or} & {\it direct-declarator} 
      {\tt [}
         {\tt static} {\openb}\,{\it type-qualifier}\,{\closeb}\/{\tt ...}
         {\it assignment-expression}
      {\tt ]} \\
  & {\bf or} & {\it direct-declarator} 
      {\tt [}
         {\it type-qualifier}\,{\tt ...} {\tt static}
         {\it assignment-expression}
      {\tt ]} \\
  & {\bf or} & {\it direct-declarator} 
      {\tt [}
         {\openb}\,{\it type-qualifier}\,{\closeb}\/{\tt ...} {\tt *}
      {\tt ]} \\
  & {\bf or} & {\it direct-declarator} 
      {\tt (} {\it parameter-type-list}\/ {\tt )} \\
  & {\bf or} & {\it direct-declarator} 
      {\tt (} {\openb} {\it identifier-list} {\closeb}\/ {\tt )}\\
 \end{tabular}
\end{center}
%
{\it coarray-spec} is defined as follows:
%
\begin{center}
 \begin{tabular}{lll}
  {\it coarray-spec} & {\bf is} & {\it explicit-coshape-spec}\\
                     & {\bf or} & {\it deferred-coshape-spec}\\
 \end{tabular}
\end{center}
%

\begin{Constraints F}
\item A coarray shall be a dummy argument or have the ALLOCATABLE or SAVE 
attribute.\footnote
{In other words, a local coarray to the procedure should be a SAVE'd or allocatable
variable unless it is a dummy argument.}

\item A coarray shall not be a function result.

\item A coarray shall not be a named constant or a pointer. 

\item A coarray shall not be a {\it common-block-object}
or a {\it equivalence-object}.

%-- COMMENT OUT. This is not in F90.
% \item The VOLATILE attribute shall not be specified for a coarray that is 
% associated by use or host association. %-- C560
%
% \item Within a BLOCK construct, the VOLATILE attribute shall not be specified
% for a coarray that is not a construct entity of that construct.
%

\end{Constraints F}

\begin{Constraints C}
\item A coarray shall be a dummy argument or have the {\tt static} or {\tt extern} 
storage class.\footnote
{Conversely, a coarray may not have the {\tt auto} or {\tt register} 
storage class.}

\item A coarray shall not be a function. 
A coarray pointer shall not be a pointer to a function.

\item A coarray shall not be of an {\tt enum} or {\tt union} type.

\item A {\it declaration-specifiers} of a {\it declaration} or 
{\it parameter-declaration} shall not contain the {\tt volatile} type qualifier.\footnote
{Because it is difficult to allow the access to the coarray from outside of the 
language system.
The {\tt const} qualifier here asserts the coarray data is not be modified 
like INTENT(IN) of Fortran.}

\item For a coarray pointer, the most right {\it pointer} of a {\it declarator} 
shall not contain the {\tt volatile} or {\tt restrict} type qualifier.\footnote
{Because it is difficult to permit .......}

\item For an array coarray, the {\it type-qualifier}s and the {\tt static} keyword
apearing inside the outermost (the most left) brackets shall meeet the following conditions.
 \begin{itemize}
 \item The {\tt const} qualifier
 \end{itemize}

\end{Constraints C}


\subsubsection*{Description}

The entity that is declared
by a {\it type-declaration-statement} (in {\XMPF}) or 
by a {\it declaration} (in {\XMPC})
with {\it coarray-spec} is a coarray variable.

{\onlyF} A coarray variable is a coarray itself.
A coarray is a scalar or an array data object and is of a basic or derived type.
The specification of {\it coarray-spec} in {\it entity-decl} 
overrides the specification of {\it coarray-spec} in {\it attr-spec}
if both are specified.

{\onlyC} \NEW{A coarray in {\XMPC} is a single or sequence of a data object 
each of which is of a basic type or of a {\tt struct} type.}
A coarray variable of a basic type or a {\tt struct} type is a coarray 
as a single data object.
A coarray variable of an array or of a pointer is a (nested) pointer 
to a coarray.

A coarray can be initialized. 
Each image can initialize coarrays on the image and cannot initialize
any coarrays on the other images.


{\onlyF} A type declaration statement 
with the specification of the codimension attribute
defines the coshape of the specified variable.

{\onlyC} If the specified variable is of a basic type or of a {\tt struct} type,
a declaration with the specification of the codimension attribute
defines the coshape of the specified variable.
If the specified variable is of an array or of a pointer,
a declaration with the specification of the codimension attribute
defines the coshape of the coarray corresponding to the specified variable.


A declaration of a coarray has either an {\it explicit-coshape-spec} 
or a {\it deferred-coshape-spec} as the {\it coarray-spec}.
A static coarray (Section~\ref{sec:Static coarray}) is declarad 
with an {\it explicit-coshape-spec} and
an allocatable coaray (Section~\ref{sec:Allocatable coarray}) is declarad 
with a {\it deferred-coshape-spec}.



{\onlyF} 
An explicit coshape specifies the corank and the cobounds, 
except the upper cobound of the final (outermost) dimension.
For each dimension, 
\(($coextent$) = ($upper cobound$) - ($lower cobound$) + 1.\)
A deffered coshape specifies the corank and does not specify cobounds.

{\onlyC}
An explicit coshape specifies the corank and the coextents,
except the extent of the final (outermost) dimension.
The lower cobound is always zero and the upper cobound is the same as 
the coextent for each dimension.
A deffered coshape specifies the corank and does not specify coextents.


The corank and coextents of an explicit coshape is explicitly declared
for the scope.
An explicit coshape of a coarray has a constant corank and 
constant coextents in the scope.
A deferred coshape of a coarray has a constant corank and
unknown coextents 

%A coarray is either an {\bf explicit-coshape coarray} or a 
%{\bf deferred-coshape coarray}.

A coshape of a coarray is either explicit or deferred in the scope.
%
The corank and coextents of the explicit coshape
are specified in the declaration of the coarray.
They are determined at compile 

The coextents of the explicit coshape are specified in the declaration of
the coarray and determined at compile time or at the begining of the scope.
%
The coextents of the deferred coshape are determined when the coarray 
is allocated. 


explicitly declared 
and is constant in the scope.
The coextents of the deferred coshape are 
It is 


The corank of the deferred coshape is 

An explicit-coshape coarray has a coshape whose corank and coextents 

The coextents of a coarray are explicitly declarad 

 in the scope, and
a differed coshape has 





\paragraph*{NOTE}
In {\XMPC}, after the following declaration:
\begin{verbatim}
            static double a1[10]:[*][4];
\end{verbatim}
{\tt a1} is not a coarray but a pointer to the coarray on this image.
The coarray is referred as {\tt a1[:]} using the notation defined in 
Chapter~\ref{chap:Base Language Extensions in {\XMPC}}.
In contrast, for a basic type variable {\tt v1} and a {\tt struct} 
variable {\tt s1}:
\begin{verbatim}
            static int b1:[*][4];
            static struct {int n; double a;} s1:[*][4];
\end{verbatim}
variables {\tt b1} and {\tt s1} are the coarrays.



\begin{Restriction}
\item 
The sum of the rank\footnote{the number of dimensions} and 
corank\footnote{the number of codimensions} of a coarray 
shall not exceed fifteen.

\end{Restriction}

\subsubsection*{Example}
See Examples in Sections~\ref{sec:Static coarray} and 
\ref{sec:Allocatable coarray}.


%============================================
\subsection{Static coarray}
\label{sec:Static coarray}
%============================================

\subsubsection*{Synopsis}

A static coarray is a coarray that has a static data and coshape during execution of the program.

An explicit-coshape coarray is a coarray that has its corank and cobounds,
which are declared with a {\it coarray-spec} that is an {\it explicit-coshape-spec}
for the coarray variable.

\subsubsection*{Syntax \onlyF}

\begin{quote}
 \begin{tabular}{lll}
  {\it explicit-coshape-spec} & {\bf is} & 
     {\openb} 
       {\openb} {\it lower-cobound} {\tt :}{\closeb} {\it upper-cobound}{\tt ,}
     {\closeb}{\tt ...}
     {\openb} {\it lower-cobound} {\tt :}{\closeb} {\tt *} \\
  {\it lower-cobound} & {\bf is} & {\it specification-expr} \\
  {\it upper-cobound} & {\bf is} & {\it specification-expr}
 \end{tabular}
\end{quote}


\subsubsection*{Syntax \onlyC}

\begin{quote}
 \begin{tabular}{lll}
  {\it explicit-coshape-spec} & {\bf is} & 
     {\tt [}\,{\tt *}\,{\tt ]}
     {\openb\/} {\tt [}\,{\it coextent\/}\,{\tt ]} {\closeb\/}{\tt ...} \\
  {\it coextent} & {\bf is} & {\it assignment-expression} \\
 \end{tabular}
\end{quote}

\begin{Constraints F}
\item 
A nonallocatable coarray shall have a {\it coarray-spec} that is an 
{\it explicit-coshape-spec}.

\item
A lower-cobound or upper-cobound that is not a constant expression shall appear 
only in a subprogram, BLOCK construct, or interface body.

\end{Constraints F}

\begin{Constraints C}
\item
If a {\it coarray-spec} appearing in an {\it init-declarator} or 
a {\it parameter-declaration} is an {\it explicit-coshape-spec}, 
the {\it declarator} followed by the {\it coarray-spec} shall be
in the following format:\footnote
{Keyword {\tt static} and {\it assignment-expression}s in ``{\tt [ ]}''
are not yet taken into account!}

\begin{quote}
 \begin{tabular}{lll}
  {\it declarator} & {\bf is} & {\it identifier} 
    {\openb\/} {\tt [} {\it assignment-expression\/} {\tt ]} {\closeb\/}{\tt ...}
 \end{tabular}
\end{quote}

\item
A {\it coextent} that is not a constant expression shall appear 
only in a block scope or function prototype scope.

\end{Constraints C}


\subsubsection*{Description}

{\onlyF}
Cobounds of an explicit-coshape coarray are determined on entry to the procedure
if they are not constant expressions.
If the lower cobound is omitted, the default value is 1.
The upper cobound shall not be less than the lower cobound.
A cosubscript of the coarray in that codimension shall not be less than the lower
cobound or greater than the upper cobound.

{\onlyC}
Coextents of an explicit-coshape coarray are determined on entry to the scope 
if they are not constant expressions.
A cosubscript of the coarray in that codimension shall be less than the coextent
and shall not be less than 0.


%============================================
\subsection{Allocatable coarray}
\label{sec:Allocatable coarray}
%============================================

\subsubsection*{Synopsis}

A deferred-coshape coarray is a coarray that has its corank 
declared by a {\it deferred-coshape-spec}.
Its cobounds are determined by allocation or argument association.

\subsubsection*{Syntax}

\begin{quote}
 \begin{tabular}{llll}
  {\onlyF} & {\it deferred-coshape-spec} & {\bf is} & 
    {\openb}\/{\tt :,} {\closeb}\/{\tt ...} {\tt :} \\
  \\
  {\onlyC} & {\it deferred-coshape-spec} & {\bf is} &
%%    \{{\tt [ ]}\}{\tt ...}
    {\tt [ ]} {\tt ...}
 \end{tabular}
\end{quote}

\begin{Constraints F}
\item 
A coarray with the ALLOCATABLE attribute shall have a {\it coarray-spec}
that is a {\it deferred-coshape-spec}.

\end{Constraints F}


\begin{Constraints C}
\item 
If a {\it coarray-spec} appearing in an {\it init-declarator} or 
a {\it parameter-declaration} is an {\it deferred-coshape-spec}, 
the {\it declarator} followed by the {\it coarray-spec} shall be
a pointer.

\end{Constraints C}


\subsubsection*{Description}

{\onlyF}
The cobounds of an unallocated allocatable coarray are undefined.
The cobounds of an allocated allocatable coarray are those specified 
when the coarray is allocated.

{\onlyC}


%============================================
\subsection{Volatile {\onlyF} Attribute/{\onlyC} Type Qualifier}
\label{sec:Volatile}
%============================================

C560 The VOLATILE attribute shall not be specified for a coarray that is accessed by use (11.2.2) or host (16.5.1.4) association.  %-- 5.3.19 VOLATILE attribute

C561 Within a BLOCK construct (8.1.4), the VOLATILE attribute shall not be specified for a coarray that is not a construct entity (16.4) of that construct.  %-- 5.3.19 VOLATILE attribute

If the target of a pointer is a coarray, the pointer shall have the VOLATILE attribute if and only if the coarray has the VOLATILE attribute.  %-- 7.2.2.3-8


%============================================
\subsection{Coarray Container}
\label{sec:Coarray Container}
%============================================

{\onlyF}
A derived type that has a scalar coarray component is called 
a \NEW{\bf coarray container type}.
A derived type that has a scalar component of a coarray container type 
is also called a coarray container type.
A scalar object of a coarray container type is called 
a \NEW{\bf coarray container}.

{\onlyC}
A {\tt struct} type that has a coarray component is called 
a \NEW{\bf coarray container type}.
A {\tt struct} type that has a component of a coarray container type 
is also called a coarray container type.
A object of a coarray container type is called 
a \NEW{\bf coarray container}.

\begin{Constraints F}
\item A coarray container shall be a dummy argument or have 
the ALLOCATABLE or SAVE attribute.\footnote
{In other words, a local coarray container to the procedure should be a SAVE'd or 
allocatable structure unless it is a dummy argument.}

\item A coarray container shall be a nonpointer nonallocatable scalar, 
shall not be a coarray, and shall not be a function result. 
\end{Constraints F}

\begin{Constraints C}
\item A coarray container shall be a dummy argument or have the {\tt static} 
or {\tt extern} storage class.\footnote
{Conversely, a coarray container may not have {\tt auto} storage class.}

\item A coarray container shall not be a coarray.

\end{Constraints C}



%-------------------------------------------------------%
\section{Argument Association}
\label{sec:Argument Association}
%-------------------------------------------------------%

\begin{Constraints F}
\item An entity with the INTENT(OUT) attribute shall not be
an allocatable coarray or a coarray container.
%-- ORIGINALLY IN C541:
% or have a subobject that is an allocatable coarray.
%--

\item An entity with the VALUE attribute shall not be a 
coarray container. %-- C557

\item A procedure that has a coarray dummy argument 
shall have an explicit interface if it is referenced. %-- 12.4.2.2-1

\end{Constraints F}


\subsubsection*{Example}

{\onlyF}
{\tt a1} and {\tt a2} are explicit-shape coarrays and
{\tt a3} and {\tt a4} are assumed-shape coarrays.
These coarrays must have explicit coshapes.
Because subroutine {\tt foo} has coarray dummy arguments,
the explicit interface must be visible to subroutine {\tt caller}.

{\onlyC}
{\tt a1} and {\tt a2} are specified size of coarrays and
{\tt a3} and {\tt a5} are unspecified size of coarrays.
These coarrays must have explicit coshapes.
Because function {\tt foo} has coarray dummy arguments,
the prototype definition must be visible to function {\tt caller}.

\begin{center}
 \begin{minipage}{0.48\hsize}
  \begin{XFexample}
module moo
 integer,parameter :: m=5
end module moo

!-- CALLER --
subroutine caller
 interface
  subroutine foo(n, a1, a2, a3, a4)
   use moo
   integer n
   real a1(10,5)[*], a2(n,m)[*]
   real a3(10,*)[*], a4(0:n-1,0:*)[*]
  end subroutine foo
 end interface
 real,save :: a(10,5)[*]
 call foo(10, a, a, a, a)
end subroutine caller

!-- SUBROUTINE --
subroutine foo(n, a1, a2, a3, a4)
 use moo
 integer n
 real a1(10,5)[*], a2(n,m)[*]
 real a3(10,*)[*], a4(0:n-1,0:*)[*]
 ...
end subroutine caller
  \end{XFexample}
 \end{minipage}
%
 \begin{minipage}{0.48\hsize}
  \begin{XCexampleR}

int const m = 5;

/*-- PROTOTYPE --*/
void foo(int n,
         float a1[5][10]:[*],
         float a2[m][n]:[*],
         float a3[][10]:[*],
         float a5[][*]:[*]);

/*-- CALLER --*/
void caller() {
  float static a[5][10];

  foo(10, a, a, a, a);
}

/*-- FUNCTION --*/
void foo(int n,
         float a1[5][10]:[*],
         float a2[m][n]:[*],
         float a3[][10]:[*],
         float a5[][n]:[*]) {
  ...

}
  \end{XCexampleR}
 \end{minipage}
\end{center}



%-------------------------------------------------------%
\section{Memory Allocation of Coarrays}
\label{sec:Memory Allocation of Coarrays}
%-------------------------------------------------------%

%============================================
\subsection{[F] Allocation of allocatable coarray}
\label{sec:Allocation of allocatable coarray}
%============================================

TBD

%============================================
\subsection{[C] Allocation of coarray pointer}
\label{sec:Allocation of coarray pointer}
%============================================

A coarray pointer is a pointer to a coarray object that is
called a target coarray.
A target coarray is allocated and freed with library functions
{\tt xmp\_comalloc} and {\tt xmp\_cofree}, respectively.
A coarray pointer retains the address of an allocated target coarray.
To avoid aliasing between coarrays, a coarray pointer is not allowed to 
point to any named coarrays or its subobjects,
nor to point the same target coarray or its subobjects that is pointed from
the other coarray pointer.


\subsubsection*{Example}

The first line of the code fragment shown below 
declares coarray pointer {\bf y} pointing to an array coarray 
of double type. 
The second line allocates a target coarray of size 
{\tt sizeof(double)*10*20} with the first-dimension coextent 4.
The third line frees the coarray and the value of {\bf y} becomes
invalid.

\begin{center}
 \begin{minipage}{0.70\hsize}
  \begin{XCexampleR}
  double (*y)[10]:[][];
  y = xmp_comalloc(sizeof(double)*10*20, 4);
  xmp_cofree(y);
  \end{XCexampleR}
 \end{minipage}
\end{center}


%-------------------------------------------------------%
\section{Referrence and Definition to Remote Coarrays}
\label{sec:Referrence and Definition to Remote Coarrays}
%-------------------------------------------------------%

TBD
%\begin{Constraints C}
\item A coarray variable shall not be pointer-assigned.
\end{Constraints C}

{\onlyF}
A cosubscript of the coarray in that codimension shall not be less than the lower
cobound or greater than the upper cobound.

{\onlyC}
A cosubscript of the coarray in that codimension shall be less than the coextent
and shall not be less than 0.


\subsubsection*{Note}
In {\XMPC}, after the following declaration:
\begin{verbatim}
            static double a1[10]:[*][4];
\end{verbatim}
{\tt a1} is not a coarray but a pointer to the coarray on this image.
The coarray is referred as {\tt a1[:]} using the notation defined in 
Chapter~\ref{chap:Base Language Extensions in {\XMPC}}.
In contrast, for a basic type variable {\tt v1} and a {\tt struct} 
variable {\tt s1}:
\begin{verbatim}
            static int b1:[*][4];
            static struct {int n; double a;} s1:[*][4];
\end{verbatim}
variables {\tt b1} and {\tt s1} are the coarrays.


%-------------------------------------------------------%
\section{Synchronization and Error Handling}
\label{sec:Synchronization and Error Handling}
%-------------------------------------------------------%

TBD

%-------------------------------------------------------%
\section{Intrinsic Procedures}
\label{sec:Intrinsic Procedures}
%-------------------------------------------------------%

TBD

% Intrinsic function {\tt xmp\_coarray\_malloc} allocates an allocatable 
% coarray, determines the coextents, and returns the address to be assigned 
% to the coarray pointer.


%-------------------------------------------------------%
\section{Compatibility with the Fortran Standard}
\label{sec:Compatibility with the Fortran Standard}
%-------------------------------------------------------%

TBD
