
\section{Glossary}
\subsection*{\Term{node}}
A compute node,
which may have several cores sharing main memory, has its own local
memory. in distributed memory system. Each node is connected via
network. An \XMP program begins as a single thread of execution
in each node.

\subsection*{\Term{node number}}
Unique number assigned to nodes in entire
node. The number starts from 1, larger than or equal to 1 and less
than and equal to the number of nodes. Note that mapping from node
number to MPI rank is decided by the system. The image index of
coarray mapping to the entire nodes is equal to the node number. 

\subsection*{\Term{node set}}
A set of nodes

\subsection*{\Term{entire node set}, \Term{entire node}}
All nodes
executing the program, or a set of the nodes. The entire node set is
decided when staring the program.

\subsection*{\Term{executing node set}, \Term{executing nodes}}
A node set executing a certain region of program. The executing node
set which execute a whole program is a entire node set. The executing
node set of a task is a node set which executes the task. 

\subsection*{\Term{node array}}
A multi-dimensional array containing nodes. Node array have a
name and shape as it attributes.

\subsection*{\Term{executing node array}}
Node array which contain executing node.

\subsection*{\Term{task}}
A specific instance of executable code and
its data environments executed in a set of nodes. In context of
program text, a set of statement executed by a set of nodes. A task
can be nested, and a nested task is executed as a subtask of outer
task.

\subsection*{\Term{template}}
A dummy array used to express an index space associated
with an array. It is also used to describe a iteration space of
loop. Template has a name, dimension and lower and upper bound of each
dimension as attributes.

\subsection*{\Term{replicated execution}}
Execution of the same
code in different nodes. If the state at starting point is same and
the execution has only local side-effect, the local state in each node
remains same. 

\subsection*{\Term{data mapping}}
The combination of alignment and
distribution attributes used to describe how a data object is
allocated to nodes.

\subsection*{\Term{work mapping}}
Assignment of iterations to nodes in
parallel loop, and tasks to nodes.

\subsection*{\Term{image index}}
A number assigned to
each images of coarray. The value is equal to or larger than 1. 
Note that the image index of the coarray mapping to entire nodes is
identical to the node number.

\subsection*{\Term{local}}
Execution of a program has
side-effect only on data in the node. In this case, no communication
to other nodes takes place. 

\subsection*{\Term{non-local}}
Execution of a program requires
communication to other nodes, and has some side-effect to other
nodes. 

\subsection*{\Term{global data}}
Data declared as a distributed array and shared by
nodes.

\subsection*{\Term{local data}}
Data is allocated in each node, and referenced only
within the node.* collective  An operation must be executed by every
node in the executing node set to perform an operation working
together.

\subsection*{\Term{distribution}}
The partition of the index space of a data
object among a set of nodes according to a given pattern. The
{\tt distribute} directive is used to mapping element of a template onto a
set of nodes. 

\subsection*{\Term{alignment}}
An attribute of a data object that establishes
the relationship between data objects for distribution. The {\tt align}
directive is used to describe the correspondence of the element of data
and template. 

\subsection*{\Term{shadow}}
A data area is used to keep neighbor elements
temporarily in distributed array. The shadow is an attribute of
distributed array, declared by {\tt shadow} directive and updated by
{\tt reflect} directive.

