\newcommand{\namelistlabel}[1]{\mbox{#1}\hfil}
\newenvironment{namelist}[1]{%
\begin{list}{}
       {\let\makelabel\namelistlabel
        \settowidth{\labelwidth}{#1}
        \setlength{\leftmargin}{1.3\labelwidth}
}
}{%
\end{list}}

\newcommand{\gitem}[1]{\item[{\parbox[b]{3.3cm}{\raggedleft \bf \Term{#1}}}]}


\section{Glossary}\label{sec:glossary}

\subsection{Language Terminology}

\begin{namelist}{entire node setxxxx}

\gitem{base language}

 A programming language that serves as the foundation of the {\XMP}
 specification.

\gitem{base program}

 A program written in a base language.

\gitem{\XMPF}

 The {\XMP} specification for a base language {\Fort}, abbreviated as
 XMP/F.

\gitem{\XMPC}

 The {\XMP} specification for a base language {\C}, abbreviated as
 XMP/C.

\gitem{structured block}

 For C, an executable statement, possibly compound, with a single entry
 at the top and a single exit at the bottom, or an {\XMP} construct.
 For Fortran, a block of executable statements with a single entry at
 the top and a single exit at the bottom, or an {\XMP} construct.

\gitem{procedure}

 A generic term used to refer to ``procedure'' (including subroutine and
 function) in {\XMPF} and ``function'' in {\XMPC}.

\gitem{directive}

 In {\XMPF}, a comment, and in {\XMPC}, a {\tt \#pragma}, that specifies
 {\XMP} program behavior.

\gitem{declarative directive}

 An {\XMP} directive that may only be placed in a declarative context. A
 declarative directive has no associated executable user code, but
 instead has one or more associated user declarations.

\gitem{executable directive}

 An {\XMP} directive that is not declarative; it may be placed in an
 executable context.

\gitem{construct}

 An {\XMP} executable directive (and for Fortran, the paired {\tt end}
 directive, if any) and the associated statement, loop or structured
 block, if any.

%, not
%including the code in any called routines; i.e., the lexical extent of an
%executable directive.

\gitem{global construct}

 A construct that is executed collectively and synchronously by every
 node in the current executing node set. Global constructs are further 
 classified into two groups of {\it \Term{global communication
 constructs}}, such as {\tt gmove}, {\tt barrier}, etc., which specify
 communication or synchronization, and {\it \Term{work mapping
 constructs}}, such as {\tt loop}, {\tt array} and {\tt tasks}, which
 specify parallelization of loops, array assignments or tasks.

\gitem{template}

 A dummy array that represents an index space to be distributed onto a
 node set, which serves as the ``template'' of parallelization in
 {\XMP} and can be considered to abstract, for example, a set of grid
 points in the grid method or particles in the particle method.
%
 A template is used in an {\XMP} program to specify the data and work
 mapping.
%
 Note that the lower bound of each dimension of a template is one in
 both {\XMPF} and {\XMPC}.

%A dummy array used to express an index space associated
%with an array. Template is also used to describe the iteration space of
%a loop. A template has a name, a dimension, and an upper and  lower
%bound for each dimension as attributes.

\gitem{data mapping}

 Allocating elements of an array to nodes in a node set by specifying
 with the {\tt align} directive that the array is aligned with a
 distributed template.

%\gitem{work}

\gitem{work mapping}

 Assigning each of the iterations of a loop, the elements of an array
 assignment, or the tasks to nodes in a node set. Such work mapping is
 specified by aligning it with a template or distributing it onto a
 node set.

%\subsection*{\Term{data mapping}}
%The combination of the alignment and
%distribution attributes used to describe how a data object is
%allocated to nodes.
%
%\subsection*{\Term{work mapping}}
%Assignment of iterations to nodes in
%a parallel loop and tasks to nodes.

\gitem{global}

 A data or a work is {\it global} if and only if there is one or more
 replicated instances of it each of which is shared by the executing
 nodes.

\gitem{local}

 A data or a work is {\it local} if and only if there is a replicated
 instance of it on each of the executing nodes.

\gitem{global-view model}

 A model of programming or parallelization, on which parallel programs
 are written by specifying how to map global data and works onto nodes.

\gitem{local-view model}

 A model of programming or parallelization, on which parallel programs
 are written by specifying how each node owns local data and does local
 works.

%Execution of a program has side-effects only on the data in the node. In
%this case, no communication with other nodes occurs .

%\subsection*{\Term{non-local}}
%
%Execution of a program requires
%communication with other nodes and has side-effects with respect to other
%nodes. 

\end{namelist}


\subsection{Node Terminology}

\begin{namelist}{entire node setxxxx}

%\gitem{physical node}
%
% A computing component of a distributed-memory multicomputer, which has
% its own main memory and is connected with each other via an
% interconnect. A node may contain multiple cores sharing the main
% memory.

%In a distributed memory system, a computation node, which may have
%several cores sharing main memory, has its own local memory. Each node
%is connected through a network. An \XMP program begins as a single
%thread of execution in each node.

\gitem{node}

 An execution entity managed by the {\XMP} runtime system, which has its
 own memory and can communicate with other nodes. A node can execute one
 or more threads concurrently.

%\gitem{logical node}
%
% An execution entity managed by the {\XMP} runtime system, which has its
% own memory and can communicate with other nodes. A logical node can
% execute one or more threads concurrently.

\gitem{node set}

 A totally ordered set of nodes.

\gitem{entire node set}

 A node set that contains all of the nodes participating in the
 execution of an {\XMP} program.

\gitem{primary node set}

 An entire node set that is specified explicitly or implicitly, and is
 the current executing node set at the beginning of the program
 execution.

\gitem{executing node set}

 A node set that contains all of the nodes participating in the
 execution of a procedure, a statement, a construct, etc. of an
 {\XMP} program is called its executing node set.
%
 This term is used in this document to represent the {\it current
 executing node set} unless it is ambiguous.
%
 Note that the executing node set of the main routine is the
 primary node set.

\gitem{current executing node set}

 An executing node set of the current context, which is managed by the
 {\XMP} runtime system.
%
 The current executing node set can be modified by the {\tt task}, {\tt
 array}, or {\tt loop} constructs. 

\gitem{executing node}

 A node in the executing node set.

\gitem{node array}

 An {\XMP} entity of the same form as a Fortran array that represents a
 node set in XcalableMP programs. Each element of a node array
 represents a node in the corresponding node set. A node array is
 declared by the {\tt nodes} directive. Note that the lower bound of
 each dimension of a node array is one in both {\XMPF} and {\XMPC}.



\gitem{non-primary node array}

A node array declared without ``={\it node-ref}'', ``=**'', or ``=*''
in a NODES directive.
A non-primary node array corresponds to all the nodes at the invocation of
a program, and also corresponds to all the images at the invocation of
a program.


\gitem{primary node array}

 A node array declared with the rhs of a node reference by ``{\tt **}''
 representing the primary node set.
 A primary node array corresponds to all the nodes at the invocation of
 a program, and also corresponds to all the images at the invocation of
 a program.

\gitem{executing node array}

 A node array declared with the rhs of a node reference by ``{\tt *}''
 representing the executing node set.
 An executing node array corresponds to the executing node set, and also
 corresponds to the current set of images at the evaluation of the
 declaration of the node array.

\gitem{parent node set}

 The parent node set of a node set is the last executing node set, which
 encountered the innermost {\tt task}, {\tt loop}, or {\tt array}
 construct that is being executed.

\gitem{node number}
A unique number assigned to each node in a node set, which starts from
one and corresponds to its position within the node set which is
totally ordered.

%\gitem{node number}
%
% A number assigned to a node, which is associated with a node array and
% determined according to Fortran's array element order (i.e. row-major).

%\subsection*{\Term{node number}}
%
%A unique number assigned to each of the nodes in the entire node
%set. The number starts from 1, larger than or equal to 1 and less than
%and equal to the number of nodes.Note that the mapping from the node
%number to the MPI rank is decided by the system. The image index of 
%the coarray mapping to the entire node set is equal to the node number. 
%
%\subsection*{\Term{entire node set}, \Term{entire set of nodes}}
%All nodes executing the program, or a set of these nodes. The entire node set is
%decided when staring the program.
%
%\subsection*{\Term{executing node set}, \Term{executing nodes}}
%A node set executing a certain region of a program. The executing node
%set that executes an entire program is the entire node set. The executing
%node set of a task is the node set that executes the task. 
%
%\subsection*{\Term{node array}}
%A multi-dimensional array containing nodes. The node array has a
%name and shape as it attributes.
%
%\subsection*{\Term{executing node array}}
%Node array that contains the executing node.

\end{namelist}


\subsection{Data Terminology}

\begin{namelist}{entire node setxxxx}

\gitem{variable}

 A named data storage block, whose value can be defined and redefined
 during the execution of a program. Note that {\it variables} include
 array sections.
% and that they do not include array names for the C
% base language.

\gitem{global data}

 An array that is aligned with a template. Elements of a global data are
 distributed onto nodes according to the distribution of the
 template. As a result, each node owns a part of a global data (called a
 {\it local section}), and can access directly it but cannot those on
 the other nodes.

%Data declared as a distributed array and shared by nodes.

\gitem{local data}

 Data that is not global. Each node owns a replica of a local data,
 and can access directly it but cannot those on the other nodes. Note
 that the replicas of a local data do not always have the same value.

%Data is allocated in each node and is referenced only
%within the node.

\gitem{replicated data}

 A data whose storage is allocated on multiple nodes. A replicated data
 is either a local data or a global data replicated by an {\tt align}
 directive.

\gitem{distribution}

 Assigning each element of a template to nodes in a node set in a
 specified manner. In the broad sense, it means that of an array, a
 loop, etc.

% The partition of the index space of a data object among a set of nodes
% according to a given pattern. The {\tt distribute} directive is used to 
% map the elements of a template onto a set of nodes.

\gitem{alignment}

 Associating each element of an array, a loop, etc. with an element of
 the specified template. An element of the aligned array, a loop,
 etc. is necessarily mapped to the same node as its associated element
 of the template.

% An attribute of a data object that establishes the relationship between
% data objects for distribution. The {\tt align} directive is used to
% describe the correspondence of the element of the data and the
% template.

\gitem{local section}

 A section of a global data that is allocated as an array on each node
 at runtime.
%
 The local section of a global data includes its shadow objects.

\gitem{shadow}

 An additional area of the local section of a distributed array, which
 is used to keep elements to be moved in from neighboring
 nodes.

% A data area used to keep neighbor elements temporarily in a
% distributed array.
% Shadow is an attribute of a distributed array that
% is declared by the {\tt shadow} directive and is updated by the {\tt
% reflect} directive.

\end{namelist}


\subsection{Work Terminology}

\begin{namelist}{entire node setxxxx}

\gitem{task}

 A specific instance of executable codes that is defined by the {\tt
 task} construct and executed by a node set specified by its {\tt on}
 clause.

%A specific instance of executable code and
%its data environments executed in a set of nodes. In the context of
%the program text, a set of statement executed by a set of nodes. A task
%can be nested, and a nested task is executed as a subtask of an outer task.

%\subsection*{\Term{replicated execution}}
%
%Execution of the same code in different nodes. If the state at the
%starting point is the same and the execution has only local
%side-effects, then the local state in each node remains the same.

%\gitem{collective}
%
%      A construct is {\it collective} if and only if it must be executed 
%      synchronously and with the exact same directive and associated
%      statements by every nodes in the executing node set. The behavior
%      of an {\XMP} program is not specified if a collective construct is
%      not executed {\it collectively}.

%An operation must be executed by every
%node in the executing node set in order to perform an operation together.

\end{namelist}


\subsection{Communication and Synchronization Terminology}

\begin{namelist}{entire node setxxxx}

\gitem{communication}

 A data movement among nodes. Communication in {\XMP} occurs only when
 the programmer instruct it explicitly with a global communication
 construct or a coarray reference.

\gitem{reduction}

 A procedure of combining variables from each node in a specified manner
 and returning the result value. A reduction always involves
 communication.
%
 A reduction is specified by either the {\tt on} clause of the {\tt
 loop} construct or the {\tt reduction} construct.

\gitem{synchronization}

 Synchronization is a mechanism to ensure that multiple nodes do not
 execute specific portions of a program at the same
 time. Synchronization among any number of nodes is specified by the
 {\tt barrier} construct and that between two nodes by the {\tt post} and
 {\tt wait} constructs.

\gitem{asynchronous communication}

 Communication that does not block and returns before it is
 complete. Thus statements that follow it can overtake it. An
 asynchronous communication is specified by the {\tt async} clause of
 global communication constructs or the {\tt async} directive for
 a coarray reference.

\end{namelist}

\subsection{Local-view Terminology}

\begin{namelist}{entire node setxxxx}

\gitem{local alias}

 An alias to the local section of a global data, that is, a distributed
 array. A local alias can be used in {\XMP} programs in the same way as
 normal local data.

%\gitem{coarray}
%
% A special local data that can be accessed directly by other nodes with
% a specific notation (i.e. the {\it image index} corresponding to the
% target node in the square brackets) added to the end of the array
% reference syntax.
%%
% Every coarray is associated explicitly or implicitly with a node
% array and allocated on each node of the node array.
%
% The coarray feature of {\XMP} is based on that of the Fortran 2008
% standard.

\gitem{current set of images}
 The current set of images is a set of images determined by the most
 lately executed {\itshape task-directive} in the TASK directive
 constructs that are not completed if any TASK directive constructs are
 being executed. The current set of images is all the images
 at the invocation of a program
 if there are no TASK directive constructs that are not completed.

\gitem{image}

An instance of an XcalableMP program.
Each image uniquely corresponds to a node.

\gitem{image index}

An integer value identifying an image in a set of images.

In {\XMPC}, the lower cobound in each axis is one by default and
taking account of the cobound,
the cosubscript list in an image selector determines the
image index in the same way that a subscript list in an array element
determines the subscript order value in Fortran,
taking account of the bounds.

\end{namelist}
