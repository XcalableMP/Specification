%% \documentclass{article}
%% \usepackage{amsmath,amssymb,graphicx,verbatim,url,listings}%,stmaryrd
%% %\usepackage[varg]{txfonts}
%% \usepackage[all]{xy}

%% \begin{document}

\chapter{Memory Consistency Model}

%loop construct is synchronous.  But practically OK?

%array construct is synchronous.  But semantically OK?

This chapter explains a memory consistency model that XcalableMP adopts.

Memory consistency models specify rules about multiple data accesses
to memories.  Since XcalableMP is an extension of the base languages,
its memory consistency model is also defined to be their extension,
that is, XcalableMP adopts all the rules that base languages do.

In addition, XcalableMP adopts some rules.  XcalableMP provides
\emph{global view} that the base languages do not.  In global view,
\emph{global communication constructs} are used to access distributed
data.  Furthermore, distributed data can be accessed through
designating data in local view.  Conversely, non-distributed data can
be accessed through designating distributed data by using global
communication constructs in global view.  These, of course, are not
considered under the memory consistency models of the base language.
Therefore, XcalableMP memory consistency model adopts additional rules
about global communication constructs.

Recall that global communication constructs are collective as
described in Section~\ref{sec:glossary}.

\section{Synchronous Communications}

The constructs \texttt{reflect}, \texttt{gmove} (and its following
assignment statement), \texttt{reduction}, and \texttt{bcast} are
synchronous if \texttt{async} is not specified.  It means that
executions of these constructs guarantee completions of data
synchronizations, that is, global constructs read data that are
written by statements that are previously executed, and their
following statements and global constructs read data written by global
constructs.  This is guaranteed even if the distributed data are
accessed in local view.

For example, in the following code, the assignment statement \texttt{l(1)=1} is
guaranteed to be completed before the \texttt{gmove} construct is
executed.  Therefore, the value of \texttt{g(6)} must be \texttt{1}.
Furthermore, since the \texttt{gmove} construct completes before the
assignment statement \texttt{x=m(2)} is executed, the value of \texttt{x} on
\texttt{p(1)} must be \texttt{1}.
\begin{center}
\begin{XFexample}
!$xmp nodes p(2)
!$xmp template t(10)
!$xmp distribute (block) onto p :: t
      integer :: g(10), h(10)
!$xmp align (i) with t(i) :: g, h
      integer :: l(10), m(10)
!$xmp local_alias l => g
!$xmp local_alias m => h
      integer x

      l(1)=1
!$xmp gmove
      h(2)=g(6)
      x=m(2)
\end{XFexample}
\end{center}

Similarly, these are also guaranteed when an integer \texttt{1} is
stored at \texttt{g(6)} through a coarray \texttt{l} as follows:
\begin{center}
\begin{XFexample}
!$xmp nodes p(2)
!$xmp template t(10)
!$xmp distribute (block) onto p :: t
      integer :; g(10), h(10)
!$xmp align (i) with t(i) :: g, h
      integer :; l(10), m(10)
!$xmp local_alias l => g
!$xmp local_alias m => h
!$xmp coarray on p :: l(:)[*]
      integer x

      l(1)[2]=1
!$xmp gmove
      h(2)=g(6)
      x=m(2)
\end{XFexample}
\end{center}

Remark that synchronous global constructs do not guarantee completions
of data are written by asynchronous global constructs that are
previously executed.  For example, in the following code, the value of
\texttt{x} may be \texttt{1} or \texttt{2}.

\begin{center}
\begin{XFexample}
!$xmp nodes p(2)
!$xmp template t(10)
!$xmp distribute (block) onto p :: t
      integer :; g(10), h(10)
!$xmp align (i) with t(i) :: g, h
      integer :; l(10), m(10)
!$xmp local_alias l => g
!$xmp local_alias m => h
!$xmp coarray on p :: l(:)[*]
      integer x

      l(1)[2]=1
!$xmp gmove async(1)
      g(7)=g(1)+1
!$xmp gmove
      h(2)=g(7)
      x=m(2)
!$xmp wait_async(1)
\end{XFexample}
\end{center}

\section{Asynchronous Communications}

The constructs \texttt{reflect}, \texttt{gmove} (and its following
assignment statement), \texttt{reduction}, and \texttt{bcast} are asynchronous
if \texttt{async}s are specified.  Completions of data read and
written by these global constructs are not guaranteed until
\texttt{wait\_async}s are executed.

For example, in the following code, the assignment statement \texttt{l(1)=1} may
not be completed before the \texttt{gmove} construct is executed.
Therefore, the value of \texttt{g(6)} is undefined.  Moreover, since
the \texttt{gmove} construct may not complete before the assignment statement
\texttt{x=m(2)} is executed, the value of \texttt{x} on \texttt{p(1)}
is undefined even if the value of \texttt{g(6)} is \texttt{1}.  This
is also the case with coarrays.
\begin{center}
\begin{XFexample}
!$xmp nodes p(2)
!$xmp template t(10)
!$xmp distribute (block) onto p :: t
      integer :: g(10), h(10)
!$xmp align (i) with t(i) :: g, h
      integer :: l(10), m(10)
!$xmp local_alias l => g
!$xmp local_alias m => h
      integer x

      l(1)=1
!$xmp gmove async(1)
      h(2)=g(6)
      x=m(2)
!$xmp wait_async(1)
\end{XFexample}
\end{center}

The \texttt{wait\_async(\textit{async-id})} guarantees a completion of
a global construct that has \textit{async-id}.  However, it does not
guarantee that an effect of an assignment statement that follows the global
construct is kept.

For example, in the following code, the value of \texttt{x} is
undefined, that is, may be \texttt{1}, \texttt{2}, or anything else.
\begin{center}
\begin{XFexample}
!$xmp nodes p(2)
!$xmp template t(10)
!$xmp distribute (block) onto p :: t
      integer :: g(10), h(10)
!$xmp align (i) with t(i) :: g, h
      integer :: l(10), m(10)
!$xmp local_alias l => g
!$xmp local_alias m => h
      integer x

      l(1)=1
!$xmp gmove async(1)
      h(2)=g(6)
      m(2)=2
!$xmp wait_async(1)
      x=m(2)
\end{XFexample}
\end{center}



%\end{document}


